\documentclass[14pt,dvipdfmx,uplatex]{beamer}
\usetheme{Madrid}
\setbeamertemplate{footline}[page number]{}
\beamertemplatenavigationsymbolsempty
\usepackage{mypresentation}
\usepackage{fvextra}
\usepackage{lucidabr}
\usepackage{hologo}
%\AtBeginShipoutFirst{\special{pdf:tounicode EUC-UCS2}}
%\usepackage{minted}
\usepackage{tikz}
\usepackage{tcolorbox}
\usetikzlibrary{arrows}
\usetikzlibrary{shapes}
\usetikzlibrary{decorations}
\usetikzlibrary{positioning}

\usepackage[noalphabet]{pxchfon}
\definecolor{ikkonzome}{rgb}	{	0.9961	,	0.7569	,	0.7373	}
\definecolor{ishitake}{rgb}	{	0.9961	,	0.6941	,	0.7059	}
\definecolor{momo}{rgb}	        {	0.9961	,	0.6824	,	0.8039	}
\definecolor{kobai}{rgb}	{	0.9412	,	0.4235	,	0.5569	}
\definecolor{nakabeni}{rgb}	{	0.9451	,	0.2706	,	0.4941	}
\definecolor{sakura}{rgb}	{	0.9333	,	0.8353	,	0.8353	}
\definecolor{arazome}{rgb}	{	0.9725	,	0.7216	,	0.7843	}
\definecolor{usubeni}{rgb}	{	0.8471	,	0.4471	,	0.5451	}
\definecolor{hisame}{rgb}	{	0.7451	,	0.4039	,	0.4039	}
\definecolor{toki}{rgb}	        {	0.9569	,	0.6431	,	0.6353	}
\definecolor{sakuranezumi}{rgb}	{	0.6941	,	0.6039	,	0.6078	}
\definecolor{sango}	{rgb}	{	0.8471	,	0.4157	,	0.3725	}
\definecolor{akane}	{rgb}	{	0.7529	,	0.0118	,	0.3451	}
\definecolor{choshun}{rgb}	{	0.7490	,	0.5255	,	0.5255	}
\definecolor{karakurenai}{rgb}	{	0.7373	,	0.0118	,	0.2667	}
\definecolor{enji}{rgb}	        {	0.6275	,	0.0863	,	0.3176	}
\definecolor{keshiaka}{rgb}	{	0.6275	,	0.4275	,	0.4275	}
\definecolor{kokiake}{rgb}	{	0.5059	,	0.0706	,	0.2549	}
\definecolor{jinzamomi}{rgb}	{	0.9098	,	0.4549	,	0.4157	}
\definecolor{mizugaki}{rgb}	{	0.7294	,	0.5529	,	0.4784	}
\definecolor{umenezumi}{rgb}	{	0.5882	,	0.3922	,	0.3882	}
\definecolor{suoko}{rgb}        {	0.5843	,	0.2667	,	0.2431	}
\definecolor{akabeni}{rgb}	{	0.8039	,	0.0784	,	0.3725	}
\definecolor{shinshu}{rgb}	{	0.6431	,	0.0353	,	0.0000	}
\definecolor{azuki}{rgb}	{	0.5255	,	0.0235	,	0.0000	}
\definecolor{ginshu}{rgb}	{	0.7490	,	0.2824	,	0.0588	}
\definecolor{ebicha}{rgb}	{	0.4549	,	0.2706	,	0.2627	}
\definecolor{kuriume}{rgb}	{	0.5843	,	0.3804	,	0.4314	}
\definecolor{akebono}{rgb}	{	0.8902	,	0.5961	,	0.4941	}
\definecolor{hanezu}{rgb}	{	0.7882	,	0.5961	,	0.5373	}
\definecolor{sangoshu}{rgb}	{	0.8196	,	0.5059	,	0.4471	}
\definecolor{shozyohi}{rgb}	{	0.7686	,	0.0000	,	0.0000	}
\definecolor{shikancha}{rgb}	{	0.5569	,	0.3294	,	0.1882	}
\definecolor{kakishibu}{rgb}	{	0.6745	,	0.4078	,	0.3333	}
\definecolor{benikaba}{rgb}	{	0.7137	,	0.3373	,	0.2941	}
\definecolor{benitobi}{rgb}	{	0.6196	,	0.3176	,	0.2706	}
\definecolor{benihihada}{rgb}	{	0.5020	,	0.3137	,	0.2353	}
\definecolor{kurotobi}{rgb}	{	0.3176	,	0.2000	,	0.1490	}
\definecolor{benihi}{rgb}	{	0.8235	,	0.4745	,	0.1922	}
\definecolor{terigaki}{rgb}	{	0.8118	,	0.4627	,	0.1804	}
\definecolor{ake}{rgb}	        {	0.7804	,	0.3098	,	0.1725	}
\definecolor{edocha}{rgb}	{	0.6863	,	0.4353	,	0.2941	}
\definecolor{bengara}{rgb}	{	0.6392	,	0.1569	,	0.0196	}
\definecolor{hihada}{rgb}	{	0.5412	,	0.3412	,	0.2353	}
\definecolor{shishi}{rgb}	{	0.8549	,	0.6863	,	0.5961	}
\definecolor{araishu}{rgb}	{	0.9294	,	0.4902	,	0.4549	}
\definecolor{akago}{rgb}	{	0.8118	,	0.5765	,	0.4275	}
\definecolor{tokigaracha}{rgb}	{	0.7922	,	0.5255	,	0.3686	}
\definecolor{otan}{rgb}	        {	0.8157	,	0.4157	,	0.2235	}
\definecolor{komugi}{rgb}	{	0.8157	,	0.6549	,	0.5098	}
\definecolor{rakuda}{rgb}	{	0.6784	,	0.5255	,	0.4118	}
\definecolor{tsurubami}{rgb}	{	0.6275	,	0.4392	,	0.3961	}
\definecolor{ama}{rgb}	        {	0.7765	,	0.6902	,	0.5843	}
\definecolor{nikkei}{rgb}	{	0.7216	,	0.4667	,	0.3725	}
\definecolor{renga}{rgb}	{	0.6902	,	0.3765	,	0.3098	}
\definecolor{sohi}{rgb}   	{	0.8078	,	0.5098	,	0.2078	}
\definecolor{enshucha}{rgb}	{	0.6706	,	0.4275	,	0.1608	}
\definecolor{karacha}{rgb}	{	0.5765	,	0.4235	,	0.1490	}
\definecolor{kabacha}{rgb}	{	0.6353	,	0.3725	,	0.1569	}
\definecolor{sodenkaracha}{rgb}	{	0.5216	,	0.3490	,	0.1373	}
\definecolor{suzumecha}{rgb}	{	0.4745	,	0.3176	,	0.1255	}
\definecolor{kurikawacha}{rgb}	{	0.4078	,	0.2745	,	0.1098	}
\definecolor{momoshiocha}{rgb}	{	0.3490	,	0.2353	,	0.0902	}
\definecolor{tobi}{rgb}	        {	0.4353	,	0.3098	,	0.1412	}
\definecolor{kurumizome}{rgb}	{	0.6667	,	0.5333	,	0.3333	}
\definecolor{kaba}{rgb}	        {	0.8196	,	0.4588	,	0.1294	}
\definecolor{korosen}{rgb}	{	0.5059	,	0.3843	,	0.1608	}
\definecolor{kogecha}{rgb}	{	0.3333	,	0.2549	,	0.1098	}
\definecolor{kokikuchinashi}{rgb}	{	0.8078	,	0.5922	,	0.3490	}
\definecolor{araigaki}{rgb}	{	0.8157	,	0.5529	,	0.3176	}
\definecolor{taisha}{rgb}	{	0.6353	,	0.4196	,	0.2078	}
\definecolor{akashirotsurubami}{rgb}	{	0.8078	,	0.6118	,	0.4157	}
\definecolor{tonocha}{rgb}	{	0.5922	,	0.4039	,	0.2000	}
\definecolor{sencha}{rgb}	{	0.5255	,	0.3529	,	0.1490	}
\definecolor{sharegaki}{rgb}	{	0.8549	,	0.7098	,	0.4863	}
\definecolor{ko}{rgb}	        {	0.9529	,	0.8275	,	0.6627	}
\definecolor{usugaki}{rgb}	{	0.8627	,	0.7294	,	0.5294	}
\definecolor{koji}{rgb}	        {	0.8275	,	0.6039	,	0.2706	}
\definecolor{umezome}{rgb}	{	0.8667	,	0.7373	,	0.4235	}
\definecolor{beniukon}{rgb}	{	0.8549	,	0.5843	,	0.2549	}
\definecolor{chojicha}{rgb}	{	0.5569	,	0.4000	,	0.1098	}
\definecolor{kenpozome}{rgb}	{	0.3059	,	0.2667	,	0.0627	}
\definecolor{biwacha}{rgb}	{	0.7333	,	0.5294	,	0.1922	}
\definecolor{kohaku}{rgb}	{	0.8039	,	0.5961	,	0.2471	}
\definecolor{usuko}{rgb}	{	0.8745	,	0.7373	,	0.4824	}
\definecolor{kuchiba}{rgb}	{	0.8157	,	0.6157	,	0.3216	}
\definecolor{kincha}{rgb}	{	0.7725	,	0.4902	,	0.2000	}
\definecolor{chozizome}{rgb}	{	0.5686	,	0.3961	,	0.1608	}
\definecolor{kitsune}{rgb}	{	0.6392	,	0.4431	,	0.1804	}
\definecolor{hushizome}{rgb}	{	0.5804	,	0.4353	,	0.1608	}
\definecolor{kyara}{rgb}	{	0.4510	,	0.3373	,	0.1255	}
\definecolor{susutake}{rgb}	{	0.4588	,	0.3529	,	0.1176	}
\definecolor{shirocha}{rgb}	{	0.7529	,	0.6588	,	0.4118	}
\definecolor{odo}{rgb}	        {	0.7137	,	0.6039	,	0.3137	}
\definecolor{ginsusutake}{rgb}	{	0.5608	,	0.4745	,	0.2392	}
\definecolor{kigaracha}{rgb}	{	0.7490	,	0.6196	,	0.2745	}
\definecolor{kobicha}	{rgb}	{	0.5020	,	0.4118	,	0.1765	}
\definecolor{usuki}	{rgb}	{	0.8549	,	0.7804	,	0.4275	}
\definecolor{yamabuki}	{rgb}	{	0.9098	,	0.8000	,	0.2039	}
\definecolor{tamago}	{rgb}	{	0.8275	,	0.7412	,	0.3373	}
\definecolor{hajizome}	{rgb}	{	0.7412	,	0.6039	,	0.2431	}
\definecolor{yamabukicha}{rgb}	{	0.7059	,	0.5765	,	0.2275	}
\definecolor{kuwazome}	{rgb}	{	0.6471	,	0.5255	,	0.2118	}
\definecolor{namakabe}	{rgb}	{	0.5569	,	0.4549	,	0.1843	}
\definecolor{kuchinashi}{rgb}	{	0.8078	,	0.7333	,	0.3843	}
\definecolor{tomorokoshi}{rgb}	{	0.7804	,	0.7020	,	0.2941	}
\definecolor{shirotsurubami}{rgb}	{	0.8745	,	0.8235	,	0.5922	}
\definecolor{kitsurubami}{rgb}	{	0.6863	,	0.6039	,	0.2118	}
\definecolor{toou}	{rgb}	{	0.8353	,	0.7725	,	0.4235	}
\definecolor{hanaba}	{rgb}	{	0.8549	,	0.8000	,	0.4941	}
\definecolor{torinoko}	{rgb}	{	0.8431	,	0.8118	,	0.6902	}
\definecolor{ukon}	{rgb}	{	0.8039	,	0.7216	,	0.3059	}
\definecolor{kikuchiba}	{rgb}	{	0.7765	,	0.6667	,	0.2902	}
\definecolor{rikyushiracha}{rgb}{	0.6627	,	0.6471	,	0.4784	}
\definecolor{rikyucha}	{rgb}	{	0.5176	,	0.5176	,	0.2588	}
\definecolor{aku}	{rgb}	{	0.5294	,	0.5059	,	0.4039	}
\definecolor{higosusutake}{rgb}	{	0.4863	,	0.4510	,	0.3059	}
\definecolor{rokocha}	{rgb}	{	0.4157	,	0.4353	,	0.3686	}
\definecolor{mirucha}	{rgb}	{	0.4471	,	0.4471	,	0.3725	}
\definecolor{natane}	{rgb}	{	0.7216	,	0.6863	,	0.3882	}
\definecolor{kimirucha}	{rgb}	{	0.5373	,	0.5059	,	0.2549	}
\definecolor{uguisucha}	{rgb}	{	0.4118	,	0.3882	,	0.1961	}
\definecolor{nanohana}	{rgb}	{	0.9882	,	0.9882	,	0.3804	}
\definecolor{kariyasu}	{rgb}	{	0.8039	,	0.7686	,	0.3882	}
\definecolor{kihada}	{rgb}	{	0.9608	,	0.9137	,	0.2863	}
\definecolor{zoge}	{rgb}	{	0.8863	,	0.8235	,	0.7216	}
\definecolor{wara}	{rgb}	{	0.7882	,	0.7490	,	0.4706	}
\definecolor{macha}	{rgb}	{	0.6235	,	0.6392	,	0.4863	}
\definecolor{yamabato}	{rgb}	{	0.5137	,	0.5176	,	0.4000	}
\definecolor{mushikuri}	{rgb}	{	0.8196	,	0.8000	,	0.6118	}
\definecolor{aokuchiba}	{rgb}	{	0.6275	,	0.6392	,	0.3451	}
\definecolor{hiwacha}	{rgb}	{	0.6784	,	0.6863	,	0.4196	}
\definecolor{ominaeshi}	{rgb}	{	0.8745	,	0.8863	,	0.4039	}
\definecolor{wasabi}	{rgb}	{	0.5922	,	0.6824	,	0.5765	}
\definecolor{uguisu}	{rgb}	{	0.3333	,	0.4118	,	0.0627	}
\definecolor{hiwa}	{rgb}	{	0.7098	,	0.7216	,	0.2392	}
\definecolor{aoshirotsurubami}{rgb}	{	0.5961	,	0.6471	,	0.4471	}
\definecolor{yanagicha}	{rgb}	{	0.5373	,	0.5725	,	0.3529	}
\definecolor{rikancha}	{rgb}	{	0.3333	,	0.4196	,	0.2863	}
\definecolor{aikobicha}	{rgb}	{	0.2706	,	0.3412	,	0.2353	}
\definecolor{koke}	{rgb}	{	0.4941	,	0.5686	,	0.3922	}
\definecolor{miru}	{rgb}	{	0.1529	,	0.3216	,	0.1843	}
\definecolor{sensai}	{rgb}	{	0.1373	,	0.2863	,	0.1608	}
\definecolor{baiko}	{rgb}	{	0.5529	,	0.6157	,	0.3922	}
\definecolor{iwai}	{rgb}	{	0.3059	,	0.4118	,	0.2784	}
\definecolor{hiwamoegi}	{rgb}	{	0.4431	,	0.6824	,	0.2431	}
\definecolor{yanagisusutake}{rgb}	{	0.2039	,	0.3333	,	0.1255	}
\definecolor{urayanagi}	{rgb}	{	0.5843	,	0.6824	,	0.2431	}
\definecolor{usumoegi}	{rgb}	{	0.4824	,	0.6706	,	0.2275	}
\definecolor{yanagizome}{rgb}	{	0.4353	,	0.6000	,	0.2078	}
\definecolor{moegi}	{rgb}	{	0.3020	,	0.5961	,	0.1882	}
\definecolor{aoni}	{rgb}	{	0.1216	,	0.4196	,	0.2431	}
\definecolor{matsuba}	{rgb}	{	0.1098	,	0.3686	,	0.2118	}
\definecolor{usuao}	{rgb}	{	0.5529	,	0.7059	,	0.6118	}
\definecolor{wakatake}	{rgb}	{	0.3843	,	0.6824	,	0.5059	}
\definecolor{yanaginezumi}{rgb}	{	0.5020	,	0.6078	,	0.5490	}
\definecolor{oitake}	{rgb}	{	0.4157	,	0.5412	,	0.4392	}
\definecolor{sensaimidori}{rgb}	{	0.2549	,	0.3882	,	0.2627	}
\definecolor{midori}	{rgb}	{	0.0000	,	0.4824	,	0.0000	}
\definecolor{byakuroku}	{rgb}	{	0.6078	,	0.7333	,	0.6196	}
\definecolor{sabiseiji}	{rgb}	{	0.5333	,	0.6588	,	0.5804	}
\definecolor{rokusho}	{rgb}	{	0.3373	,	0.6039	,	0.4039	}
\definecolor{tokusa}	{rgb}	{	0.2549	,	0.4706	,	0.3922	}
\definecolor{onandocha}	{rgb}	{	0.2039	,	0.3725	,	0.3098	}
\definecolor{aotake}	{rgb}	{	0.1412	,	0.5176	,	0.4353	}
\definecolor{rikyunezumi}{rgb}	{	0.4157	,	0.5686	,	0.5490	}
\definecolor{birodo}	{rgb}	{	0.1059	,	0.4275	,	0.3608	}
\definecolor{mishiao}	{rgb}	{	0.1098	,	0.4588	,	0.3882	}
\definecolor{aimirucha}	{rgb}	{	0.2510	,	0.4000	,	0.3608	}
\definecolor{tonotya}	{rgb}	{	0.3059	,	0.4941	,	0.4392	}
\definecolor{mizuasagi}	{rgb}	{	0.1569	,	0.6745	,	0.6745	}
\definecolor{seji}	{rgb}	{	0.4745	,	0.6588	,	0.5922	}
\definecolor{seheki}	{rgb}	{	0.0588	,	0.5569	,	0.4824	}
\definecolor{sabitetsu}	{rgb}	{	0.0392	,	0.3294	,	0.2784	}
\definecolor{tetsu}	{rgb}	{	0.0627	,	0.3961	,	0.3961	}
\definecolor{omeshicha}	{rgb}	{	0.0667	,	0.4667	,	0.4667	}
\definecolor{korainando}{rgb}	{	0.0588	,	0.3922	,	0.0392	}
\definecolor{minatonezumi}{rgb}	{	0.4549	,	0.6000	,	0.6118	}
\definecolor{aonibi}	{rgb}	{	0.1804	,	0.3725	,	0.3882	}
\definecolor{tetsuonando}{rgb}	{	0.1961	,	0.4118	,	0.4275	}
\definecolor{mizu}	{rgb}	{	0.5451	,	0.7412	,	0.7608	}
\definecolor{sabiasagi}	{rgb}	{	0.4510	,	0.6000	,	0.6000	}
\definecolor{kamenozoki}{rgb}	{	0.7098	,	0.9020	,	0.8902	}
\definecolor{asagi}	{rgb}	{	0.3020	,	0.6784	,	0.7098	}
\definecolor{shinbashi}	{rgb}	{	0.0000	,	0.6000	,	0.6353	}
\definecolor{sabionando}{rgb}	{	0.2392	,	0.3961	,	0.3843	}
\definecolor{ainezumi}	{rgb}	{	0.2235	,	0.4000	,	0.3843	}
\definecolor{ai}	{rgb}	{	0.2039	,	0.3765	,	0.4314	}
\definecolor{onando}	{rgb}	{	0.1843	,	0.3686	,	0.4000	}
\definecolor{hanaasagi}	{rgb}	{	0.2000	,	0.6196	,	0.7098	}
\definecolor{chigusa}	{rgb}	{	0.1922	,	0.5725	,	0.6745	}
\definecolor{masuhana}	{rgb}	{	0.1569	,	0.4667	,	0.5569	}
\definecolor{hanada}	{rgb}	{	0.0039	,	0.4275	,	0.5373	}
\definecolor{noshimehana}{rgb}	{	0.1412	,	0.4235	,	0.5098	}
\definecolor{omeshionando}{rgb}	{	0.1176	,	0.3529	,	0.4235	}
\definecolor{sora}	{rgb}	{	0.1451	,	0.7216	,	0.8039	}
\definecolor{konpeki}	{rgb}	{	0.0902	,	0.5098	,	0.7333	}
\definecolor{kurotsurubami}{rgb}{	0.0627	,	0.3137	,	0.3451	}
\definecolor{gunjo}	{rgb}	{	0.5098	,	0.7882	,	0.9098	}
\definecolor{kon}	{rgb}	{	0.0000	,	0.2000	,	0.4000	}
\definecolor{kachi}	{rgb}	{	0.0000	,	0.1765	,	0.3490	}
\definecolor{ruri}	{rgb}	{	0.0078	,	0.3922	,	0.6510	}
\definecolor{konjo}	{rgb}	{	0.0039	,	0.3137	,	0.5216	}
\definecolor{rurikon}	{rgb}	{	0.0000	,	0.3020	,	0.5020	}
\definecolor{benimidori}{rgb}	{	0.3373	,	0.4588	,	0.6784	}
\definecolor{konkikyo}	{rgb}	{	0.0039	,	0.0667	,	0.4275	}
\definecolor{hujinezumi}{rgb}	{	0.3216	,	0.3686	,	0.6118	}
\definecolor{benikakehana}{rgb}	{	0.2275	,	0.2471	,	0.5882	}
\definecolor{hujiiro}	{rgb}	{	0.4706	,	0.4863	,	0.7059	}
\definecolor{hutaai}	{rgb}	{	0.3137	,	0.3333	,	0.5608	}
\definecolor{hujimurasaki}{rgb}	{	0.4157	,	0.3333	,	0.5686	}
\definecolor{kikyo}	{rgb}	{	0.3529	,	0.3216	,	0.5725	}
\definecolor{shion}	{rgb}	{	0.5137	,	0.4196	,	0.6784	}
\definecolor{messhi}	{rgb}	{	0.2196	,	0.1765	,	0.3098	}
\definecolor{shikon}	{rgb}	{	0.2510	,	0.1961	,	0.3529	}
\definecolor{kokimurasaki}{rgb}	{	0.3098	,	0.2000	,	0.3765	}
\definecolor{usu}	{rgb}	{	0.7294	,	0.6353	,	0.7843	}
\definecolor{hashita}	{rgb}	{	0.6000	,	0.4549	,	0.6706	}
\definecolor{rindo}	{rgb}	{	0.5922	,	0.5137	,	0.7529	}
\definecolor{sumire}	{rgb}	{	0.3882	,	0.2157	,	0.5922	}
\definecolor{nasukon}	{rgb}	{	0.3216	,	0.2235	,	0.4353	}
\definecolor{murasaki}	{rgb}	{	0.3137	,	0.1765	,	0.4824	}
\definecolor{kurobeni}	{rgb}	{	0.2353	,	0.1294	,	0.3059	}
\definecolor{ayame}	{rgb}	{	0.4275	,	0.1569	,	0.5098	}
\definecolor{benihuji}	{rgb}	{	0.6824	,	0.5333	,	0.6667	}
\definecolor{edomurasaki}{rgb}	{	0.3961	,	0.1529	,	0.4392	}
\definecolor{kodaimurasaki}{rgb}{	0.4353	,	0.2863	,	0.4627	}
\definecolor{shikon}{rgb}       {	0.4118	,	0.2549	,	0.3765	}
\definecolor{hatobanezumi}{rgb}	{	0.4235	,	0.3804	,	0.4392	}
\definecolor{budonezumi}{rgb}	{	0.3412	,	0.2196	,	0.3529	}
\definecolor{ebizome}	{rgb}	{	0.4000	,	0.2235	,	0.3882	}
\definecolor{hujisusutake}{rgb}	{	0.2549	,	0.0824	,	0.2471	}
\definecolor{usuebi}	{rgb}	{	0.6549	,	0.4392	,	0.6039	}
\definecolor{botan}	{rgb}	{	0.6706	,	0.0392	,	0.4353	}
\definecolor{umemurasaki}{rgb}	{	0.6667	,	0.3922	,	0.5176	}
\definecolor{nisemurasaki}{rgb}	{	0.3686	,	0.0000	,	0.3216	}
\definecolor{murasakitobi}{rgb}	{	0.3922	,	0.3255	,	0.3529	}
\definecolor{ususuo}{rgb}	{	0.7569	,	0.4000	,	0.5412	}
\definecolor{suo}{rgb}	        {	0.6941	,	0.0627	,	0.4157	}
\definecolor{kuwanomi}{rgb}	{	0.4196	,	0.0549	,	0.2745	}
\definecolor{nibi}{rgb}	        {	0.4549	,	0.4235	,	0.3961	}
\definecolor{benikeshi}{rgb}	{	0.4118	,	0.3804	,	0.3843	}
\definecolor{shironeri}{rgb}	{	0.9843	,	0.9843	,	0.9843	}
\definecolor{shironezumi}{rgb}	{	0.6471	,	0.6588	,	0.6627	}
\definecolor{ginnezumi}{rgb}	{	0.5255	,	0.5490	,	0.5922	}
\definecolor{sunezumi}{rgb}	{	0.4275	,	0.4392	,	0.4392	}
\definecolor{dobunezumi}{rgb}	{	0.2863	,	0.2941	,	0.2941	}
\definecolor{aisumicha}{rgb}	{	0.2078	,	0.2196	,	0.2510	}
\definecolor{binrojizome}{rgb}	{	0.2118	,	0.0824	,	0.0706	}
\definecolor{sumizome}{rgb}	{	0.2706	,	0.2706	,	0.2706	}


\setgothicfont{migmix-2p-bold.ttf}
%\setgothicfont{YasashisaBold.ttf}
%\setminchofont{migmix-2p-bold.ttf} % 本文
\mathversion{bold}

\setbeamerfont{title}{size=\HUGE{28}{34},family={\yasagoth}}
\setbeamerfont{frametitle}{size=\HUGE{20}{28},series={\yasagoth}}
\setbeamerfont{frametext}{size=\HUGE{20}{28},series={\yasagoth}}
\setbeamertemplate{frametitle}[default][left]
\usefonttheme{professionalfonts}

\setbeamercolor{background}{bg=white}
\setbeamercolor{author}{fg=black}
\setbeamercolor{date}{fg=black}
\setbeamercolor{title}{fg=white, bg=kachi}
\setbeamercolor{frametitle}{fg=white}
\setbeamercolor{normal text}{fg=black}
\setbeamerfont{normal text}{family=\rmfamily, series=\bfseries}
\setbeamercolor{structure}{fg=black}

\makeatletter
\define@key{beamerframe}{t}[true]{% top
  \beamer@frametopskip=.2cm plus .5\paperheight\relax%
  \beamer@framebottomskip=0pt plus 1fill\relax%
  \beamer@frametopskipautobreak=\beamer@frametopskip\relax%
  \beamer@framebottomskipautobreak=\beamer@framebottomskip\relax%
  \def\beamer@initfirstlineunskip{}%
}
\def\header#1{\vskip.5\baselineskip{\large\sffamily #1}}
\tikzset{
  notice/.style  = { fill=shozyohi, white, 
                     rectangle callout, 
                     rounded corners,
                     callout absolute pointer={#1} }
}
\makeatother

\setlength{\leftmargini}{12pt}
\setlength{\leftmarginii}{12pt}

\edef\0{\string\0}
\DeclareTextCommand{\CarriageReturn}{JY2}{\015}

\title{2018年でもEPSを\TeX{}で使う}
\author{\sffamily 鹿野 桂一郎\\
\bfseries ラムダノート株式会社\\
\small\bfseries \email{k16.shikano@lambdanote.com} \\ 
\twitter{golden\_lucky} 
}
\date{\sffamily\footnotesize 2018年11月10日\\ 於\, TeXConf 2018}

\begin{document}
\fontseries{ub}\selectfont

\frame{\titlepage}

\setbeamertemplate{background canvas}[vertical shading][bottom=white,top=kachi!15]
\setbeamercolor{frametitle}{bg=kachi, fg=white}
\setbeamercolor{structure}{fg=kachi}

\begin{frame}[plain]
  \begin{center}
    \HUGE{28}{34}\color{kachi}\yasagoth
    \begin{tikzpicture}
    \node[text width=100mm,align=center]{\TeX{}では、画像を\textbf{EPS}で埋め込みます};
    \end{tikzpicture}
  \end{center}
\end{frame}

\begin{frame}[plain]
  \begin{center}
    \HUGE{28}{34}\color{kachi}\yasagoth
    \begin{tikzpicture}
    \node[cross out,draw=red,ultra thick,text width=100mm,align=center]{\TeX{}では、画像を\textbf{EPS}で埋め込みます};
    \end{tikzpicture}
  \end{center}
\end{frame}

\begin{frame}[t]{\inhibitglue そもそもEPSとはなにか}
  \sffamily
  \begin{itemize}
    \item Encapsulated {\color{red}PostScript}
    \item PostScriptは、印刷のためのデバイスに依存しない、プログラミング言語
    \item それだと実際の印刷に使いにくいので、{\color{red}DSC}という規約がある
  \end{itemize}
\end{frame}

\begin{frame}[t,fragile=singleslide]{\inhibitglue EPSはDSC準拠のデータ交換形式}
  \sffamily
  \begin{itemize}
    \item Document Structuring Convention
    \item アプリケーションに依存する情報とか、そのファイルがどういう構造になっているかとか、そういう情報を埋め込むときの決まりごと
    \item 他のPostScriptファイルから取り込めるような単ページの絵としてのファイル形式を、DSCに従って決めたものが、{\color{red}EPS}
  \end{itemize}
  \fontsize{7pt}{7pt}\selectfont
  \begin{tcolorbox}
  \begin{Verbatim}[commandchars=\\\{\},mathescape]
%!PS-Adobe-3.1 EPSF-3.0
 \rotatebox{90}{$\cdots$}
%%BoundingBox: 0 0 300 100
  \end{Verbatim}
  \end{tcolorbox}
\end{frame}

\begin{frame}[t,fragile=singleslide]{\inhibitglue \TeX{}はEPSを選んだ}
  \sffamily
  \begin{itemize}
    \item おそらく、ほかの選択肢が事実上なかった
      \begin{itemize}
        \item Plain \TeX{}の\texttt{epsf.sty}マクロはクヌースの手がかかっている
      \end{itemize}
    \item ただし、\TeX{}そのものはPostScriptを「完全には」処理できない
      \begin{itemize}
        \item DVIにおける外部画像の扱いは、「\texttt{special}を使って後段のデバイスに丸投げ」が基本
        \item pdf\TeX{}では、EPSは非サポート
     \end{itemize}
  \end{itemize}
\end{frame}

\begin{frame}[t,fragile=singleslide]{\inhibitglue Ghostscript}
  \sffamily
  \begin{enumerate}
    \item \TeX{}エンジンが画像処理のたびに呼び出す
    \item DVIウェアが画像処理のたびに呼び出す
    \item だったら最初から画像をすべてPDFに変換して、それをpdf\TeX{}や\texttt{dvipdfmx}が直接PDFに埋め込めるようにすべき、というのが最近の潮流
  \end{enumerate}
\end{frame}

\begin{frame}[t,fragile=singleslide]{\inhibitglue Ghostscript}
  \sffamily
  \begin{enumerate}
    \item \TeX{}エンジンが画像処理のたびに呼び出す
    \item DVIウェアが画像処理のたびに呼び出す
    \item だったら最初から画像をすべてPDFに変換して、それをpdf\TeX{}や\texttt{dvipdfmx}が直接PDFに埋め込めるようにすべき、というのが最近の潮流
  \end{enumerate}
  
  \HUGE{21}{28}\color{black}\yasagoth
  \begin{center}
  「\TeX{}はGhostscriptから離れては生きていけないのよ」
  \end{center}
\end{frame}


\setbeamertemplate{background canvas}[vertical shading][bottom=white,top=yamabuki!15]
\setbeamercolor{frametitle}{bg=yamabuki, fg=black}
\setbeamercolor{structure}{fg=yamabuki}

\begin{frame}[plain]
  \begin{center}
    \HUGE{28}{34}\color{black}\yasagoth
    EPSは、いまやほんとうにいらない子なのだろうか?
  \end{center}
\end{frame}

\begin{frame}[t,fragile=singleslide]{\inhibitglue それでもEPSを使いたいこともある}
  \sffamily
  \begin{itemize}
    \item EPSファイルをもらったけどAdobe Illustratorで開いたら微妙な状況になった
    \item 画像をテキスト形式でバージョン管理したい
  \end{itemize}
\end{frame}

\begin{frame}[plain]
  \begin{center}
    \HUGE{28}{34}\color{black}\yasagoth
    \TeX{}(広義)だけでPostScriptを処理できればいいのに……
  \end{center}
\end{frame}

\begin{frame}[t,fragile=singleslide]{\inhibitglue \hologo{METAPOST}由来のEPS}
  \sffamily
  \begin{itemize}
    \item \hologo{METAPOST}のEPS出力は、複雑なPostScriptコードを含まない(\textit{purified EPS})
    \item \underline{\TeX{}のエコシステムだけで処理できる}、{\color{red}良いEPS}
    \begin{itemize}
      \item pdf\TeX{}とLua\TeX{}は、EPSなら自分で処理する!
      \item dvipdfmxも、\hologo{METAPOST}を処理できる!
      \item いずれも\texttt{.mps}という拡張子にする必要あり
      \item dvipdfmxには、\texttt{.mps}を直接PDFにする\texttt{-M}オプションもある
    \end{itemize}
  \end{itemize}
\end{frame}

\begin{frame}[t,fragile=singleslide]{\inhibitglue EPSをpurifyする}
  \sffamily
  \begin{itemize}
    \item その名も\texttt{purifyeps}
    \item \texttt{pstoedit}で\hologo{METAPOST}のソースに変換し、その結果を\texttt{mpost}にかけるだけのPerlスクリプト
    \begin{itemize}
      \item もとのEPSのPostScriptは、前段の\texttt{pstoedit}が解釈してくれる
      \item 後段の\texttt{mpost}で使うフォントマップを\texttt{pstoedit}に指定してくれる
      \item \hologo{METAPOST}が知っているフォント名を適当に当てはめてあげる必要がある
    \end{itemize}
  \end{itemize}
  \begin{center}
  \resizebox{100mm}{!}{% Created by Eps2pgf 0.7.0 (build on 2008-08-24) on Thu Nov 08 14:40:42 JST 2018
\begin{pgfpicture}
\pgfpathmoveto{\pgfqpoint{0cm}{0cm}}
\pgfpathlineto{\pgfqpoint{14.725cm}{0cm}}
\pgfpathlineto{\pgfqpoint{14.725cm}{4.096cm}}
\pgfpathlineto{\pgfqpoint{0cm}{4.096cm}}
\pgfpathclose
\pgfusepath{clip}
\definecolor{eps2pgf_color}{rgb}{0,0,0}\pgfsetstrokecolor{eps2pgf_color}\pgfsetfillcolor{eps2pgf_color}
\pgftext[x=4.585cm,y=3.117cm,rotate=0]{\fontsize{10.04}{14.45}\selectfont{\textsf{purifyeps}}}
\pgftext[x=4.585cm,y=2.041cm,rotate=0]{\fontsize{10.04}{14.45}\selectfont{\textsf{pstoedit + mpost}}}
\pgftext[x=5.185cm,y=1.226cm,rotate=0]{\fontsize{10.04}{14.45}\selectfont{\textsf{+ \hologo{METAPOST}}}}
%\pgftext[x=10.91cm,y=2.559cm,rotate=0]{\fontsize{10.04}{14.45}\selectfont{\textsf{Eps2pgf}}}
\pgftext[x=1.222cm,y=3.075cm,rotate=0]{\fontsize{10.04}{14.45}\selectfont{\textsf{EPS}}}
\pgftext[x=1.222cm,y=2.525cm,rotate=0]{\fontsize{10.04}{14.45}\selectfont{\textsf{(PostScript)}}}
%\pgftext[x=13.35cm,y=3.075cm,rotate=0]{\fontsize{10.04}{14.45}\selectfont{\textsf{\TeX}}}
%\pgftext[x=13.35cm,y=2.532cm,rotate=0]{\fontsize{10.04}{14.45}\selectfont{\textsf{(PGF)}}}
\pgftext[x=8.011cm,y=3.033cm,rotate=0]{\fontsize{10.04}{14.45}\selectfont{\textsf{purified EPS}}}
\pgftext[x=8.011cm,y=2.524cm,rotate=0]{\fontsize{10.04}{14.45}\selectfont{\textsf{(mps)}}}
\pgftext[x=4.529cm,y=3.9cm,rotate=0]{\fontsize{10.04}{14.45}\selectfont{\textsf{mpost.fmp}}}
\pgfsetdash{}{0cm}
\pgfsetlinewidth{0.352mm}
\pgfsetbeveljoin
\pgfpathmoveto{\pgfqpoint{0.018cm}{3.701cm}}
\pgfpathlineto{\pgfqpoint{0.018cm}{1.611cm}}
\pgfpathlineto{\pgfqpoint{1.949cm}{1.611cm}}
\pgfpathlineto{\pgfqpoint{2.367cm}{2.045cm}}
\pgfpathlineto{\pgfqpoint{2.367cm}{3.701cm}}
\pgfpathclose
\pgfusepath{stroke}
\definecolor{eps2pgf_color}{rgb}{0.6,0.6,0.6}\pgfsetstrokecolor{eps2pgf_color}\pgfsetfillcolor{eps2pgf_color}
\pgfpathmoveto{\pgfqpoint{1.949cm}{2.045cm}}
\pgfpathlineto{\pgfqpoint{1.949cm}{1.611cm}}
\pgfpathlineto{\pgfqpoint{2.367cm}{2.045cm}}
\pgfpathclose
\pgfusepath{fill}
\pgfsetdash{}{0cm}
\definecolor{eps2pgf_color}{rgb}{0,0,0}\pgfsetstrokecolor{eps2pgf_color}\pgfsetfillcolor{eps2pgf_color}
\pgfpathmoveto{\pgfqpoint{1.949cm}{2.045cm}}
\pgfpathlineto{\pgfqpoint{1.949cm}{1.611cm}}
\pgfpathlineto{\pgfqpoint{2.367cm}{2.045cm}}
\pgfpathclose
\pgfusepath{stroke}
%\pgfsetdash{}{0cm}
%\pgfpathmoveto{\pgfqpoint{12.217cm}{3.701cm}}
%\pgfpathlineto{\pgfqpoint{12.217cm}{1.611cm}}
%\pgfpathlineto{\pgfqpoint{14.149cm}{1.611cm}}
%\pgfpathlineto{\pgfqpoint{14.566cm}{2.045cm}}
%\pgfpathlineto{\pgfqpoint{14.566cm}{3.701cm}}
%\pgfpathclose
%\pgfusepath{stroke}
%\definecolor{eps2pgf_color}{rgb}{0.6,0.6,0.6}\pgfsetstrokecolor{eps2pgf_color}\pgfsetfillcolor{eps2pgf_color}
%\pgfpathmoveto{\pgfqpoint{14.149cm}{2.045cm}}
%\pgfpathlineto{\pgfqpoint{14.149cm}{1.611cm}}
%\pgfpathlineto{\pgfqpoint{14.566cm}{2.045cm}}
%\pgfpathclose
%\pgfusepath{fill}
%\pgfsetdash{}{0cm}
%\definecolor{eps2pgf_color}{rgb}{0,0,0}\pgfsetstrokecolor{eps2pgf_color}\pgfsetfillcolor{eps2pgf_color}
%\pgfpathmoveto{\pgfqpoint{14.149cm}{2.045cm}}
%\pgfpathlineto{\pgfqpoint{14.149cm}{1.611cm}}
%\pgfpathlineto{\pgfqpoint{14.566cm}{2.045cm}}
%\pgfpathclose
\pgfusepath{stroke}
\pgfsetdash{}{0cm}
\pgfpathmoveto{\pgfqpoint{6.697cm}{3.701cm}}
\pgfpathlineto{\pgfqpoint{6.697cm}{1.611cm}}
\pgfpathlineto{\pgfqpoint{8.91cm}{1.611cm}}
\pgfpathlineto{\pgfqpoint{9.328cm}{2.045cm}}
\pgfpathlineto{\pgfqpoint{9.328cm}{3.701cm}}
\pgfpathclose
\pgfusepath{stroke}
\definecolor{eps2pgf_color}{rgb}{0.6,0.6,0.6}\pgfsetstrokecolor{eps2pgf_color}\pgfsetfillcolor{eps2pgf_color}
\pgfpathmoveto{\pgfqpoint{8.91cm}{2.045cm}}
\pgfpathlineto{\pgfqpoint{8.91cm}{1.611cm}}
\pgfpathlineto{\pgfqpoint{9.328cm}{2.045cm}}
\pgfpathclose
\pgfusepath{fill}
\pgfsetdash{}{0cm}
\definecolor{eps2pgf_color}{rgb}{0,0,0}\pgfsetstrokecolor{eps2pgf_color}\pgfsetfillcolor{eps2pgf_color}
\pgfpathmoveto{\pgfqpoint{8.91cm}{2.045cm}}
\pgfpathlineto{\pgfqpoint{8.91cm}{1.611cm}}
\pgfpathlineto{\pgfqpoint{9.328cm}{2.045cm}}
\pgfpathclose
\pgfusepath{stroke}
\pgfsetdash{}{0cm}
\pgfsetmiterjoin
\pgfpathmoveto{\pgfqpoint{6.168cm}{1.611cm}}
\pgfpathlineto{\pgfqpoint{2.941cm}{1.611cm}}
\pgfpathcurveto{\pgfqpoint{2.828cm}{1.611cm}}{\pgfqpoint{2.737cm}{1.702cm}}{\pgfqpoint{2.737cm}{1.815cm}}
\pgfpathlineto{\pgfqpoint{2.737cm}{3.468cm}}
\pgfpathcurveto{\pgfqpoint{2.737cm}{3.58cm}}{\pgfqpoint{2.828cm}{3.671cm}}{\pgfqpoint{2.941cm}{3.671cm}}
\pgfpathlineto{\pgfqpoint{6.168cm}{3.671cm}}
\pgfpathcurveto{\pgfqpoint{6.28cm}{3.671cm}}{\pgfqpoint{6.371cm}{3.58cm}}{\pgfqpoint{6.371cm}{3.468cm}}
\pgfpathlineto{\pgfqpoint{6.371cm}{1.815cm}}
\pgfpathcurveto{\pgfqpoint{6.371cm}{1.702cm}}{\pgfqpoint{6.28cm}{1.611cm}}{\pgfqpoint{6.168cm}{1.611cm}}
\pgfpathclose
\pgfusepath{stroke}
\pgfsetdash{}{0cm}
%\pgfpathmoveto{\pgfqpoint{11.636cm}{2.053cm}}
%\pgfpathlineto{\pgfqpoint{9.944cm}{2.053cm}}
%\pgfpathcurveto{\pgfqpoint{9.831cm}{2.053cm}}{\pgfqpoint{9.74cm}{2.144cm}}{\pgfqpoint{9.74cm}{2.256cm}}
%\pgfpathlineto{\pgfqpoint{9.74cm}{2.907cm}}
%\pgfpathcurveto{\pgfqpoint{9.74cm}{3.02cm}}{\pgfqpoint{9.831cm}{3.111cm}}{\pgfqpoint{9.944cm}{3.111cm}}
%\pgfpathlineto{\pgfqpoint{11.636cm}{3.111cm}}
%\pgfpathcurveto{\pgfqpoint{11.748cm}{3.111cm}}{\pgfqpoint{11.84cm}{3.02cm}}{\pgfqpoint{11.84cm}{2.907cm}}
%\pgfpathlineto{\pgfqpoint{11.84cm}{2.256cm}}
%\pgfpathcurveto{\pgfqpoint{11.84cm}{2.144cm}}{\pgfqpoint{11.748cm}{2.053cm}}{\pgfqpoint{11.636cm}{2.053cm}}
%\pgfpathclose
\pgfusepath{stroke}
\definecolor{eps2pgf_color}{rgb}{0.01508,0.02539,0.02301}\pgfsetstrokecolor{eps2pgf_color}\pgfsetfillcolor{eps2pgf_color}
\pgfpathmoveto{\pgfqpoint{3.804cm}{0.541cm}}
\pgfpathcurveto{\pgfqpoint{3.804cm}{0.541cm}}{\pgfqpoint{3.758cm}{0.365cm}}{\pgfqpoint{3.379cm}{0.365cm}}
\pgfpathcurveto{\pgfqpoint{3cm}{0.365cm}}{\pgfqpoint{2.995cm}{0.51cm}}{\pgfqpoint{2.995cm}{0.522cm}}
\pgfpathcurveto{\pgfqpoint{3cm}{0.645cm}}{\pgfqpoint{3.199cm}{0.598cm}}{\pgfqpoint{3.199cm}{0.598cm}}
\pgfpathcurveto{\pgfqpoint{3.199cm}{0.598cm}}{\pgfqpoint{3.121cm}{0.595cm}}{\pgfqpoint{3.121cm}{0.525cm}}
\pgfpathcurveto{\pgfqpoint{3.121cm}{0.467cm}}{\pgfqpoint{3.212cm}{0.422cm}}{\pgfqpoint{3.396cm}{0.417cm}}
\pgfpathcurveto{\pgfqpoint{3.77cm}{0.406cm}}{\pgfqpoint{3.804cm}{0.541cm}}{\pgfqpoint{3.804cm}{0.541cm}}
\pgfpathclose
\pgfusepath{fill}
\pgfpathmoveto{\pgfqpoint{3.513cm}{1.376cm}}
\pgfpathcurveto{\pgfqpoint{3.513cm}{1.376cm}}{\pgfqpoint{3.649cm}{1.362cm}}{\pgfqpoint{3.629cm}{1.215cm}}
\pgfpathcurveto{\pgfqpoint{3.629cm}{1.215cm}}{\pgfqpoint{3.715cm}{1.16cm}}{\pgfqpoint{3.709cm}{1.082cm}}
\pgfpathcurveto{\pgfqpoint{3.703cm}{1.005cm}}{\pgfqpoint{3.645cm}{0.913cm}}{\pgfqpoint{3.46cm}{0.883cm}}
\pgfpathcurveto{\pgfqpoint{3.46cm}{0.883cm}}{\pgfqpoint{3.567cm}{0.912cm}}{\pgfqpoint{3.571cm}{0.972cm}}
\pgfpathcurveto{\pgfqpoint{3.575cm}{1.036cm}}{\pgfqpoint{3.465cm}{1.063cm}}{\pgfqpoint{3.465cm}{1.123cm}}
\pgfpathcurveto{\pgfqpoint{3.465cm}{1.183cm}}{\pgfqpoint{3.602cm}{1.191cm}}{\pgfqpoint{3.591cm}{1.292cm}}
\pgfpathcurveto{\pgfqpoint{3.583cm}{1.359cm}}{\pgfqpoint{3.513cm}{1.376cm}}{\pgfqpoint{3.513cm}{1.376cm}}
\pgfpathclose
\pgfusepath{fill}
\definecolor{eps2pgf_color}{rgb}{0.22266,0.69531,0.8164}\pgfsetstrokecolor{eps2pgf_color}\pgfsetfillcolor{eps2pgf_color}
\pgfpathmoveto{\pgfqpoint{3.405cm}{1.07cm}}
\pgfpathcurveto{\pgfqpoint{3.414cm}{1.075cm}}{\pgfqpoint{3.398cm}{1.086cm}}{\pgfqpoint{3.399cm}{1.107cm}}
\pgfpathcurveto{\pgfqpoint{3.4cm}{1.13cm}}{\pgfqpoint{3.409cm}{1.144cm}}{\pgfqpoint{3.396cm}{1.144cm}}
\pgfpathcurveto{\pgfqpoint{3.385cm}{1.143cm}}{\pgfqpoint{3.369cm}{1.124cm}}{\pgfqpoint{3.372cm}{1.103cm}}
\pgfpathcurveto{\pgfqpoint{3.375cm}{1.083cm}}{\pgfqpoint{3.394cm}{1.065cm}}{\pgfqpoint{3.405cm}{1.07cm}}
\pgfpathclose
\pgfusepath{fill}
\pgfpathmoveto{\pgfqpoint{3.391cm}{1.008cm}}
\pgfpathcurveto{\pgfqpoint{3.38cm}{1.001cm}}{\pgfqpoint{3.408cm}{0.993cm}}{\pgfqpoint{3.414cm}{0.972cm}}
\pgfpathcurveto{\pgfqpoint{3.422cm}{0.948cm}}{\pgfqpoint{3.413cm}{0.931cm}}{\pgfqpoint{3.433cm}{0.935cm}}
\pgfpathcurveto{\pgfqpoint{3.448cm}{0.938cm}}{\pgfqpoint{3.463cm}{0.963cm}}{\pgfqpoint{3.452cm}{0.984cm}}
\pgfpathcurveto{\pgfqpoint{3.44cm}{1.004cm}}{\pgfqpoint{3.404cm}{1.017cm}}{\pgfqpoint{3.391cm}{1.008cm}}
\pgfpathclose
\pgfusepath{fill}
\pgfpathmoveto{\pgfqpoint{3.345cm}{1.053cm}}
\pgfpathcurveto{\pgfqpoint{3.353cm}{1.059cm}}{\pgfqpoint{3.335cm}{1.067cm}}{\pgfqpoint{3.332cm}{1.087cm}}
\pgfpathcurveto{\pgfqpoint{3.328cm}{1.11cm}}{\pgfqpoint{3.334cm}{1.126cm}}{\pgfqpoint{3.321cm}{1.123cm}}
\pgfpathcurveto{\pgfqpoint{3.311cm}{1.12cm}}{\pgfqpoint{3.3cm}{1.097cm}}{\pgfqpoint{3.306cm}{1.078cm}}
\pgfpathcurveto{\pgfqpoint{3.313cm}{1.059cm}}{\pgfqpoint{3.336cm}{1.045cm}}{\pgfqpoint{3.345cm}{1.053cm}}
\pgfpathclose
\pgfusepath{fill}
\definecolor{eps2pgf_color}{rgb}{0.01508,0.02539,0.02301}\pgfsetstrokecolor{eps2pgf_color}\pgfsetfillcolor{eps2pgf_color}
\pgfpathmoveto{\pgfqpoint{3.778cm}{0.66cm}}
\pgfpathcurveto{\pgfqpoint{3.723cm}{0.754cm}}{\pgfqpoint{3.559cm}{0.78cm}}{\pgfqpoint{3.441cm}{0.784cm}}
\pgfpathcurveto{\pgfqpoint{3.338cm}{0.787cm}}{\pgfqpoint{3.207cm}{0.794cm}}{\pgfqpoint{3.207cm}{0.831cm}}
\pgfpathcurveto{\pgfqpoint{3.207cm}{0.881cm}}{\pgfqpoint{3.353cm}{0.852cm}}{\pgfqpoint{3.351cm}{0.914cm}}
\pgfpathcurveto{\pgfqpoint{3.35cm}{0.969cm}}{\pgfqpoint{3.213cm}{0.971cm}}{\pgfqpoint{3.232cm}{1.108cm}}
\pgfpathcurveto{\pgfqpoint{3.241cm}{1.172cm}}{\pgfqpoint{3.32cm}{1.206cm}}{\pgfqpoint{3.388cm}{1.222cm}}
\pgfpathcurveto{\pgfqpoint{3.457cm}{1.237cm}}{\pgfqpoint{3.545cm}{1.267cm}}{\pgfqpoint{3.546cm}{1.304cm}}
\pgfpathcurveto{\pgfqpoint{3.548cm}{1.342cm}}{\pgfqpoint{3.508cm}{1.355cm}}{\pgfqpoint{3.508cm}{1.355cm}}
\pgfpathcurveto{\pgfqpoint{3.508cm}{1.355cm}}{\pgfqpoint{3.531cm}{1.341cm}}{\pgfqpoint{3.522cm}{1.313cm}}
\pgfpathcurveto{\pgfqpoint{3.513cm}{1.285cm}}{\pgfqpoint{3.48cm}{1.283cm}}{\pgfqpoint{3.385cm}{1.281cm}}
\pgfpathcurveto{\pgfqpoint{3.305cm}{1.28cm}}{\pgfqpoint{3.188cm}{1.27cm}}{\pgfqpoint{3.111cm}{1.189cm}}
\pgfpathcurveto{\pgfqpoint{3.031cm}{1.104cm}}{\pgfqpoint{3.044cm}{0.964cm}}{\pgfqpoint{3.222cm}{0.908cm}}
\pgfpathcurveto{\pgfqpoint{3.222cm}{0.908cm}}{\pgfqpoint{3.029cm}{0.875cm}}{\pgfqpoint{3.033cm}{0.768cm}}
\pgfpathcurveto{\pgfqpoint{3.037cm}{0.661cm}}{\pgfqpoint{3.161cm}{0.635cm}}{\pgfqpoint{3.354cm}{0.631cm}}
\pgfpathcurveto{\pgfqpoint{3.55cm}{0.627cm}}{\pgfqpoint{3.624cm}{0.608cm}}{\pgfqpoint{3.624cm}{0.562cm}}
\pgfpathcurveto{\pgfqpoint{3.624cm}{0.524cm}}{\pgfqpoint{3.565cm}{0.503cm}}{\pgfqpoint{3.475cm}{0.5cm}}
\pgfpathcurveto{\pgfqpoint{3.38cm}{0.497cm}}{\pgfqpoint{3.313cm}{0.51cm}}{\pgfqpoint{3.313cm}{0.537cm}}
\pgfpathcurveto{\pgfqpoint{3.312cm}{0.572cm}}{\pgfqpoint{3.403cm}{0.555cm}}{\pgfqpoint{3.403cm}{0.555cm}}
\pgfpathcurveto{\pgfqpoint{3.403cm}{0.555cm}}{\pgfqpoint{3.369cm}{0.57cm}}{\pgfqpoint{3.325cm}{0.568cm}}
\pgfpathcurveto{\pgfqpoint{3.293cm}{0.566cm}}{\pgfqpoint{3.271cm}{0.55cm}}{\pgfqpoint{3.271cm}{0.53cm}}
\pgfpathcurveto{\pgfqpoint{3.27cm}{0.5cm}}{\pgfqpoint{3.297cm}{0.475cm}}{\pgfqpoint{3.37cm}{0.462cm}}
\pgfpathcurveto{\pgfqpoint{3.483cm}{0.444cm}}{\pgfqpoint{3.629cm}{0.445cm}}{\pgfqpoint{3.724cm}{0.503cm}}
\pgfpathcurveto{\pgfqpoint{3.795cm}{0.546cm}}{\pgfqpoint{3.803cm}{0.616cm}}{\pgfqpoint{3.778cm}{0.66cm}}
\pgfpathclose
\pgfusepath{fill}
\definecolor{eps2pgf_color}{rgb}{0.22266,0.69531,0.8164}\pgfsetstrokecolor{eps2pgf_color}\pgfsetfillcolor{eps2pgf_color}
\pgfpathmoveto{\pgfqpoint{3.009cm}{0.162cm}}
\pgfpathcurveto{\pgfqpoint{3.009cm}{0.144cm}}{\pgfqpoint{3.012cm}{0.139cm}}{\pgfqpoint{3.028cm}{0.139cm}}
\pgfpathlineto{\pgfqpoint{3.028cm}{0.133cm}}
\pgfpathlineto{\pgfqpoint{2.965cm}{0.133cm}}
\pgfpathlineto{\pgfqpoint{2.965cm}{0.139cm}}
\pgfpathcurveto{\pgfqpoint{2.982cm}{0.139cm}}{\pgfqpoint{2.985cm}{0.143cm}}{\pgfqpoint{2.985cm}{0.159cm}}
\pgfpathlineto{\pgfqpoint{2.985cm}{0.22cm}}
\pgfpathcurveto{\pgfqpoint{2.985cm}{0.236cm}}{\pgfqpoint{2.976cm}{0.246cm}}{\pgfqpoint{2.961cm}{0.246cm}}
\pgfpathcurveto{\pgfqpoint{2.948cm}{0.246cm}}{\pgfqpoint{2.938cm}{0.24cm}}{\pgfqpoint{2.926cm}{0.223cm}}
\pgfpathlineto{\pgfqpoint{2.926cm}{0.156cm}}
\pgfpathcurveto{\pgfqpoint{2.926cm}{0.143cm}}{\pgfqpoint{2.932cm}{0.139cm}}{\pgfqpoint{2.946cm}{0.139cm}}
\pgfpathlineto{\pgfqpoint{2.946cm}{0.133cm}}
\pgfpathlineto{\pgfqpoint{2.882cm}{0.133cm}}
\pgfpathlineto{\pgfqpoint{2.882cm}{0.139cm}}
\pgfpathcurveto{\pgfqpoint{2.9cm}{0.139cm}}{\pgfqpoint{2.903cm}{0.143cm}}{\pgfqpoint{2.903cm}{0.162cm}}
\pgfpathlineto{\pgfqpoint{2.903cm}{0.323cm}}
\pgfpathcurveto{\pgfqpoint{2.903cm}{0.339cm}}{\pgfqpoint{2.899cm}{0.342cm}}{\pgfqpoint{2.882cm}{0.342cm}}
\pgfpathlineto{\pgfqpoint{2.882cm}{0.348cm}}
\pgfpathlineto{\pgfqpoint{2.92cm}{0.352cm}}
\pgfpathlineto{\pgfqpoint{2.926cm}{0.352cm}}
\pgfpathlineto{\pgfqpoint{2.926cm}{0.235cm}}
\pgfpathcurveto{\pgfqpoint{2.942cm}{0.253cm}}{\pgfqpoint{2.953cm}{0.26cm}}{\pgfqpoint{2.968cm}{0.26cm}}
\pgfpathcurveto{\pgfqpoint{2.989cm}{0.26cm}}{\pgfqpoint{3.009cm}{0.246cm}}{\pgfqpoint{3.009cm}{0.217cm}}
\pgfpathlineto{\pgfqpoint{3.009cm}{0.162cm}}
\pgfpathclose
\pgfusepath{fill}
\pgfpathmoveto{\pgfqpoint{3.164cm}{0.194cm}}
\pgfpathcurveto{\pgfqpoint{3.164cm}{0.232cm}}{\pgfqpoint{3.136cm}{0.26cm}}{\pgfqpoint{3.098cm}{0.26cm}}
\pgfpathlineto{\pgfqpoint{3.098cm}{0.253cm}}
\pgfpathcurveto{\pgfqpoint{3.126cm}{0.253cm}}{\pgfqpoint{3.139cm}{0.225cm}}{\pgfqpoint{3.139cm}{0.194cm}}
\pgfpathlineto{\pgfqpoint{3.139cm}{0.194cm}}
\pgfpathcurveto{\pgfqpoint{3.139cm}{0.164cm}}{\pgfqpoint{3.127cm}{0.136cm}}{\pgfqpoint{3.098cm}{0.136cm}}
\pgfpathcurveto{\pgfqpoint{3.07cm}{0.136cm}}{\pgfqpoint{3.057cm}{0.163cm}}{\pgfqpoint{3.057cm}{0.194cm}}
\pgfpathcurveto{\pgfqpoint{3.057cm}{0.23cm}}{\pgfqpoint{3.073cm}{0.253cm}}{\pgfqpoint{3.098cm}{0.253cm}}
\pgfpathlineto{\pgfqpoint{3.098cm}{0.26cm}}
\pgfpathcurveto{\pgfqpoint{3.06cm}{0.26cm}}{\pgfqpoint{3.032cm}{0.232cm}}{\pgfqpoint{3.032cm}{0.194cm}}
\pgfpathcurveto{\pgfqpoint{3.032cm}{0.157cm}}{\pgfqpoint{3.06cm}{0.129cm}}{\pgfqpoint{3.098cm}{0.129cm}}
\pgfpathcurveto{\pgfqpoint{3.136cm}{0.129cm}}{\pgfqpoint{3.164cm}{0.157cm}}{\pgfqpoint{3.164cm}{0.194cm}}
\pgfpathclose
\pgfusepath{fill}
\pgfpathmoveto{\pgfqpoint{3.247cm}{0.222cm}}
\pgfpathcurveto{\pgfqpoint{3.244cm}{0.243cm}}{\pgfqpoint{3.231cm}{0.253cm}}{\pgfqpoint{3.217cm}{0.253cm}}
\pgfpathcurveto{\pgfqpoint{3.205cm}{0.253cm}}{\pgfqpoint{3.197cm}{0.246cm}}{\pgfqpoint{3.197cm}{0.236cm}}
\pgfpathcurveto{\pgfqpoint{3.197cm}{0.205cm}}{\pgfqpoint{3.26cm}{0.213cm}}{\pgfqpoint{3.26cm}{0.167cm}}
\pgfpathcurveto{\pgfqpoint{3.26cm}{0.147cm}}{\pgfqpoint{3.243cm}{0.13cm}}{\pgfqpoint{3.222cm}{0.13cm}}
\pgfpathcurveto{\pgfqpoint{3.209cm}{0.13cm}}{\pgfqpoint{3.197cm}{0.139cm}}{\pgfqpoint{3.192cm}{0.139cm}}
\pgfpathcurveto{\pgfqpoint{3.187cm}{0.139cm}}{\pgfqpoint{3.184cm}{0.135cm}}{\pgfqpoint{3.184cm}{0.13cm}}
\pgfpathlineto{\pgfqpoint{3.178cm}{0.13cm}}
\pgfpathlineto{\pgfqpoint{3.178cm}{0.172cm}}
\pgfpathlineto{\pgfqpoint{3.184cm}{0.172cm}}
\pgfpathcurveto{\pgfqpoint{3.189cm}{0.147cm}}{\pgfqpoint{3.208cm}{0.136cm}}{\pgfqpoint{3.222cm}{0.136cm}}
\pgfpathcurveto{\pgfqpoint{3.234cm}{0.136cm}}{\pgfqpoint{3.243cm}{0.146cm}}{\pgfqpoint{3.243cm}{0.158cm}}
\pgfpathcurveto{\pgfqpoint{3.243cm}{0.192cm}}{\pgfqpoint{3.18cm}{0.183cm}}{\pgfqpoint{3.18cm}{0.224cm}}
\pgfpathcurveto{\pgfqpoint{3.18cm}{0.243cm}}{\pgfqpoint{3.196cm}{0.26cm}}{\pgfqpoint{3.216cm}{0.26cm}}
\pgfpathcurveto{\pgfqpoint{3.229cm}{0.26cm}}{\pgfqpoint{3.233cm}{0.254cm}}{\pgfqpoint{3.241cm}{0.254cm}}
\pgfpathcurveto{\pgfqpoint{3.244cm}{0.254cm}}{\pgfqpoint{3.246cm}{0.256cm}}{\pgfqpoint{3.248cm}{0.26cm}}
\pgfpathlineto{\pgfqpoint{3.254cm}{0.26cm}}
\pgfpathlineto{\pgfqpoint{3.254cm}{0.222cm}}
\pgfpathlineto{\pgfqpoint{3.247cm}{0.222cm}}
\pgfpathclose
\pgfusepath{fill}
\definecolor{eps2pgf_color}{rgb}{0.01508,0.02539,0.02301}\pgfsetstrokecolor{eps2pgf_color}\pgfsetfillcolor{eps2pgf_color}
\pgfpathmoveto{\pgfqpoint{3.452cm}{0.222cm}}
\pgfpathcurveto{\pgfqpoint{3.45cm}{0.242cm}}{\pgfqpoint{3.437cm}{0.253cm}}{\pgfqpoint{3.422cm}{0.253cm}}
\pgfpathcurveto{\pgfqpoint{3.41cm}{0.253cm}}{\pgfqpoint{3.402cm}{0.246cm}}{\pgfqpoint{3.402cm}{0.236cm}}
\pgfpathcurveto{\pgfqpoint{3.402cm}{0.205cm}}{\pgfqpoint{3.466cm}{0.213cm}}{\pgfqpoint{3.466cm}{0.167cm}}
\pgfpathcurveto{\pgfqpoint{3.466cm}{0.147cm}}{\pgfqpoint{3.448cm}{0.129cm}}{\pgfqpoint{3.427cm}{0.129cm}}
\pgfpathcurveto{\pgfqpoint{3.414cm}{0.129cm}}{\pgfqpoint{3.403cm}{0.138cm}}{\pgfqpoint{3.397cm}{0.138cm}}
\pgfpathcurveto{\pgfqpoint{3.392cm}{0.138cm}}{\pgfqpoint{3.389cm}{0.134cm}}{\pgfqpoint{3.389cm}{0.13cm}}
\pgfpathlineto{\pgfqpoint{3.383cm}{0.13cm}}
\pgfpathlineto{\pgfqpoint{3.383cm}{0.172cm}}
\pgfpathlineto{\pgfqpoint{3.389cm}{0.172cm}}
\pgfpathcurveto{\pgfqpoint{3.394cm}{0.146cm}}{\pgfqpoint{3.413cm}{0.136cm}}{\pgfqpoint{3.428cm}{0.136cm}}
\pgfpathcurveto{\pgfqpoint{3.44cm}{0.136cm}}{\pgfqpoint{3.449cm}{0.145cm}}{\pgfqpoint{3.449cm}{0.158cm}}
\pgfpathcurveto{\pgfqpoint{3.449cm}{0.192cm}}{\pgfqpoint{3.385cm}{0.182cm}}{\pgfqpoint{3.385cm}{0.224cm}}
\pgfpathcurveto{\pgfqpoint{3.385cm}{0.243cm}}{\pgfqpoint{3.401cm}{0.26cm}}{\pgfqpoint{3.422cm}{0.26cm}}
\pgfpathcurveto{\pgfqpoint{3.434cm}{0.26cm}}{\pgfqpoint{3.438cm}{0.254cm}}{\pgfqpoint{3.446cm}{0.254cm}}
\pgfpathcurveto{\pgfqpoint{3.449cm}{0.254cm}}{\pgfqpoint{3.451cm}{0.255cm}}{\pgfqpoint{3.453cm}{0.26cm}}
\pgfpathlineto{\pgfqpoint{3.459cm}{0.26cm}}
\pgfpathlineto{\pgfqpoint{3.459cm}{0.222cm}}
\pgfpathlineto{\pgfqpoint{3.452cm}{0.222cm}}
\pgfpathclose
\pgfusepath{fill}
\pgfpathmoveto{\pgfqpoint{3.592cm}{0.215cm}}
\pgfpathlineto{\pgfqpoint{3.586cm}{0.215cm}}
\pgfpathcurveto{\pgfqpoint{3.581cm}{0.238cm}}{\pgfqpoint{3.565cm}{0.252cm}}{\pgfqpoint{3.545cm}{0.252cm}}
\pgfpathcurveto{\pgfqpoint{3.521cm}{0.252cm}}{\pgfqpoint{3.504cm}{0.231cm}}{\pgfqpoint{3.504cm}{0.201cm}}
\pgfpathcurveto{\pgfqpoint{3.504cm}{0.167cm}}{\pgfqpoint{3.525cm}{0.139cm}}{\pgfqpoint{3.55cm}{0.139cm}}
\pgfpathcurveto{\pgfqpoint{3.565cm}{0.139cm}}{\pgfqpoint{3.582cm}{0.147cm}}{\pgfqpoint{3.595cm}{0.165cm}}
\pgfpathlineto{\pgfqpoint{3.595cm}{0.156cm}}
\pgfpathcurveto{\pgfqpoint{3.582cm}{0.138cm}}{\pgfqpoint{3.564cm}{0.128cm}}{\pgfqpoint{3.542cm}{0.128cm}}
\pgfpathcurveto{\pgfqpoint{3.506cm}{0.128cm}}{\pgfqpoint{3.481cm}{0.154cm}}{\pgfqpoint{3.481cm}{0.191cm}}
\pgfpathcurveto{\pgfqpoint{3.481cm}{0.229cm}}{\pgfqpoint{3.509cm}{0.259cm}}{\pgfqpoint{3.543cm}{0.259cm}}
\pgfpathcurveto{\pgfqpoint{3.562cm}{0.259cm}}{\pgfqpoint{3.571cm}{0.252cm}}{\pgfqpoint{3.578cm}{0.252cm}}
\pgfpathcurveto{\pgfqpoint{3.581cm}{0.252cm}}{\pgfqpoint{3.584cm}{0.254cm}}{\pgfqpoint{3.586cm}{0.259cm}}
\pgfpathlineto{\pgfqpoint{3.592cm}{0.259cm}}
\pgfpathlineto{\pgfqpoint{3.592cm}{0.215cm}}
\pgfpathclose
\pgfusepath{fill}
\pgfpathmoveto{\pgfqpoint{3.658cm}{0.156cm}}
\pgfpathcurveto{\pgfqpoint{3.658cm}{0.143cm}}{\pgfqpoint{3.662cm}{0.139cm}}{\pgfqpoint{3.678cm}{0.139cm}}
\pgfpathlineto{\pgfqpoint{3.678cm}{0.133cm}}
\pgfpathlineto{\pgfqpoint{3.614cm}{0.133cm}}
\pgfpathlineto{\pgfqpoint{3.614cm}{0.139cm}}
\pgfpathcurveto{\pgfqpoint{3.633cm}{0.139cm}}{\pgfqpoint{3.635cm}{0.144cm}}{\pgfqpoint{3.635cm}{0.163cm}}
\pgfpathlineto{\pgfqpoint{3.635cm}{0.23cm}}
\pgfpathcurveto{\pgfqpoint{3.635cm}{0.248cm}}{\pgfqpoint{3.63cm}{0.25cm}}{\pgfqpoint{3.61cm}{0.25cm}}
\pgfpathlineto{\pgfqpoint{3.61cm}{0.256cm}}
\pgfpathlineto{\pgfqpoint{3.652cm}{0.26cm}}
\pgfpathlineto{\pgfqpoint{3.658cm}{0.26cm}}
\pgfpathlineto{\pgfqpoint{3.658cm}{0.233cm}}
\pgfpathcurveto{\pgfqpoint{3.681cm}{0.253cm}}{\pgfqpoint{3.691cm}{0.26cm}}{\pgfqpoint{3.699cm}{0.26cm}}
\pgfpathcurveto{\pgfqpoint{3.706cm}{0.26cm}}{\pgfqpoint{3.715cm}{0.256cm}}{\pgfqpoint{3.722cm}{0.25cm}}
\pgfpathlineto{\pgfqpoint{3.712cm}{0.228cm}}
\pgfpathcurveto{\pgfqpoint{3.703cm}{0.234cm}}{\pgfqpoint{3.692cm}{0.239cm}}{\pgfqpoint{3.684cm}{0.239cm}}
\pgfpathcurveto{\pgfqpoint{3.676cm}{0.239cm}}{\pgfqpoint{3.669cm}{0.235cm}}{\pgfqpoint{3.658cm}{0.224cm}}
\pgfpathlineto{\pgfqpoint{3.658cm}{0.156cm}}
\pgfpathclose
\pgfusepath{fill}
\pgfpathmoveto{\pgfqpoint{3.786cm}{0.308cm}}
\pgfpathcurveto{\pgfqpoint{3.786cm}{0.3cm}}{\pgfqpoint{3.779cm}{0.293cm}}{\pgfqpoint{3.771cm}{0.293cm}}
\pgfpathcurveto{\pgfqpoint{3.763cm}{0.293cm}}{\pgfqpoint{3.756cm}{0.3cm}}{\pgfqpoint{3.756cm}{0.308cm}}
\pgfpathcurveto{\pgfqpoint{3.756cm}{0.317cm}}{\pgfqpoint{3.763cm}{0.323cm}}{\pgfqpoint{3.771cm}{0.323cm}}
\pgfpathcurveto{\pgfqpoint{3.779cm}{0.323cm}}{\pgfqpoint{3.786cm}{0.317cm}}{\pgfqpoint{3.786cm}{0.308cm}}
\pgfpathclose
\pgfusepath{fill}
\pgfpathmoveto{\pgfqpoint{3.783cm}{0.156cm}}
\pgfpathcurveto{\pgfqpoint{3.783cm}{0.144cm}}{\pgfqpoint{3.786cm}{0.139cm}}{\pgfqpoint{3.803cm}{0.139cm}}
\pgfpathlineto{\pgfqpoint{3.803cm}{0.133cm}}
\pgfpathlineto{\pgfqpoint{3.736cm}{0.133cm}}
\pgfpathlineto{\pgfqpoint{3.736cm}{0.139cm}}
\pgfpathcurveto{\pgfqpoint{3.755cm}{0.139cm}}{\pgfqpoint{3.759cm}{0.141cm}}{\pgfqpoint{3.759cm}{0.156cm}}
\pgfpathlineto{\pgfqpoint{3.759cm}{0.232cm}}
\pgfpathcurveto{\pgfqpoint{3.759cm}{0.249cm}}{\pgfqpoint{3.754cm}{0.25cm}}{\pgfqpoint{3.736cm}{0.25cm}}
\pgfpathlineto{\pgfqpoint{3.736cm}{0.256cm}}
\pgfpathlineto{\pgfqpoint{3.777cm}{0.26cm}}
\pgfpathlineto{\pgfqpoint{3.783cm}{0.26cm}}
\pgfpathlineto{\pgfqpoint{3.783cm}{0.156cm}}
\pgfpathclose
\pgfusepath{fill}
\pgfpathmoveto{\pgfqpoint{3.825cm}{0.059cm}}
\pgfpathcurveto{\pgfqpoint{3.825cm}{0.046cm}}{\pgfqpoint{3.82cm}{0.042cm}}{\pgfqpoint{3.804cm}{0.042cm}}
\pgfpathlineto{\pgfqpoint{3.804cm}{0.036cm}}
\pgfpathlineto{\pgfqpoint{3.868cm}{0.036cm}}
\pgfpathlineto{\pgfqpoint{3.868cm}{0.042cm}}
\pgfpathcurveto{\pgfqpoint{3.851cm}{0.042cm}}{\pgfqpoint{3.848cm}{0.048cm}}{\pgfqpoint{3.848cm}{0.063cm}}
\pgfpathlineto{\pgfqpoint{3.848cm}{0.144cm}}
\pgfpathcurveto{\pgfqpoint{3.862cm}{0.133cm}}{\pgfqpoint{3.871cm}{0.129cm}}{\pgfqpoint{3.884cm}{0.129cm}}
\pgfpathlineto{\pgfqpoint{3.881cm}{0.137cm}}
\pgfpathcurveto{\pgfqpoint{3.87cm}{0.137cm}}{\pgfqpoint{3.857cm}{0.143cm}}{\pgfqpoint{3.848cm}{0.153cm}}
\pgfpathlineto{\pgfqpoint{3.848cm}{0.23cm}}
\pgfpathlineto{\pgfqpoint{3.848cm}{0.23cm}}
\pgfpathcurveto{\pgfqpoint{3.859cm}{0.243cm}}{\pgfqpoint{3.868cm}{0.248cm}}{\pgfqpoint{3.881cm}{0.248cm}}
\pgfpathcurveto{\pgfqpoint{3.903cm}{0.248cm}}{\pgfqpoint{3.919cm}{0.224cm}}{\pgfqpoint{3.919cm}{0.191cm}}
\pgfpathcurveto{\pgfqpoint{3.919cm}{0.161cm}}{\pgfqpoint{3.903cm}{0.137cm}}{\pgfqpoint{3.881cm}{0.137cm}}
\pgfpathlineto{\pgfqpoint{3.884cm}{0.129cm}}
\pgfpathcurveto{\pgfqpoint{3.919cm}{0.129cm}}{\pgfqpoint{3.944cm}{0.157cm}}{\pgfqpoint{3.944cm}{0.195cm}}
\pgfpathcurveto{\pgfqpoint{3.944cm}{0.232cm}}{\pgfqpoint{3.921cm}{0.26cm}}{\pgfqpoint{3.89cm}{0.26cm}}
\pgfpathcurveto{\pgfqpoint{3.874cm}{0.26cm}}{\pgfqpoint{3.863cm}{0.254cm}}{\pgfqpoint{3.848cm}{0.239cm}}
\pgfpathlineto{\pgfqpoint{3.848cm}{0.26cm}}
\pgfpathlineto{\pgfqpoint{3.842cm}{0.26cm}}
\pgfpathlineto{\pgfqpoint{3.804cm}{0.256cm}}
\pgfpathlineto{\pgfqpoint{3.804cm}{0.25cm}}
\pgfpathcurveto{\pgfqpoint{3.822cm}{0.25cm}}{\pgfqpoint{3.825cm}{0.248cm}}{\pgfqpoint{3.825cm}{0.231cm}}
\pgfpathlineto{\pgfqpoint{3.825cm}{0.059cm}}
\pgfpathclose
\pgfusepath{fill}
\pgfpathmoveto{\pgfqpoint{3.995cm}{0.259cm}}
\pgfpathlineto{\pgfqpoint{4.037cm}{0.259cm}}
\pgfpathlineto{\pgfqpoint{4.037cm}{0.249cm}}
\pgfpathlineto{\pgfqpoint{3.995cm}{0.249cm}}
\pgfpathlineto{\pgfqpoint{3.995cm}{0.17cm}}
\pgfpathcurveto{\pgfqpoint{3.995cm}{0.153cm}}{\pgfqpoint{4.003cm}{0.144cm}}{\pgfqpoint{4.016cm}{0.144cm}}
\pgfpathcurveto{\pgfqpoint{4.025cm}{0.144cm}}{\pgfqpoint{4.032cm}{0.148cm}}{\pgfqpoint{4.039cm}{0.156cm}}
\pgfpathlineto{\pgfqpoint{4.043cm}{0.152cm}}
\pgfpathcurveto{\pgfqpoint{4.032cm}{0.138cm}}{\pgfqpoint{4.02cm}{0.131cm}}{\pgfqpoint{4.006cm}{0.131cm}}
\pgfpathcurveto{\pgfqpoint{3.986cm}{0.131cm}}{\pgfqpoint{3.971cm}{0.146cm}}{\pgfqpoint{3.971cm}{0.167cm}}
\pgfpathlineto{\pgfqpoint{3.971cm}{0.249cm}}
\pgfpathlineto{\pgfqpoint{3.949cm}{0.249cm}}
\pgfpathlineto{\pgfqpoint{3.949cm}{0.255cm}}
\pgfpathcurveto{\pgfqpoint{3.965cm}{0.264cm}}{\pgfqpoint{3.979cm}{0.279cm}}{\pgfqpoint{3.989cm}{0.3cm}}
\pgfpathlineto{\pgfqpoint{3.995cm}{0.3cm}}
\pgfpathlineto{\pgfqpoint{3.995cm}{0.259cm}}
\pgfpathclose
\pgfusepath{fill}
\definecolor{eps2pgf_color}{rgb}{0.22266,0.69531,0.8164}\pgfsetstrokecolor{eps2pgf_color}\pgfsetfillcolor{eps2pgf_color}
\pgfpathmoveto{\pgfqpoint{3.317cm}{0.257cm}}
\pgfpathlineto{\pgfqpoint{3.359cm}{0.257cm}}
\pgfpathlineto{\pgfqpoint{3.359cm}{0.248cm}}
\pgfpathlineto{\pgfqpoint{3.317cm}{0.248cm}}
\pgfpathlineto{\pgfqpoint{3.317cm}{0.168cm}}
\pgfpathcurveto{\pgfqpoint{3.317cm}{0.151cm}}{\pgfqpoint{3.326cm}{0.142cm}}{\pgfqpoint{3.339cm}{0.142cm}}
\pgfpathcurveto{\pgfqpoint{3.348cm}{0.142cm}}{\pgfqpoint{3.354cm}{0.146cm}}{\pgfqpoint{3.362cm}{0.154cm}}
\pgfpathlineto{\pgfqpoint{3.365cm}{0.15cm}}
\pgfpathcurveto{\pgfqpoint{3.355cm}{0.136cm}}{\pgfqpoint{3.342cm}{0.129cm}}{\pgfqpoint{3.329cm}{0.129cm}}
\pgfpathcurveto{\pgfqpoint{3.308cm}{0.129cm}}{\pgfqpoint{3.294cm}{0.144cm}}{\pgfqpoint{3.294cm}{0.165cm}}
\pgfpathlineto{\pgfqpoint{3.294cm}{0.248cm}}
\pgfpathlineto{\pgfqpoint{3.271cm}{0.248cm}}
\pgfpathlineto{\pgfqpoint{3.271cm}{0.254cm}}
\pgfpathcurveto{\pgfqpoint{3.287cm}{0.262cm}}{\pgfqpoint{3.301cm}{0.277cm}}{\pgfqpoint{3.311cm}{0.298cm}}
\pgfpathlineto{\pgfqpoint{3.317cm}{0.298cm}}
\pgfpathlineto{\pgfqpoint{3.317cm}{0.257cm}}
\pgfpathclose
\pgfusepath{fill}
\pgfpathmoveto{\pgfqpoint{2.785cm}{0.087cm}}
\pgfpathcurveto{\pgfqpoint{2.786cm}{0.087cm}}{\pgfqpoint{2.786cm}{0.087cm}}{\pgfqpoint{2.787cm}{0.087cm}}
\pgfpathlineto{\pgfqpoint{2.787cm}{0.086cm}}
\pgfpathlineto{\pgfqpoint{2.799cm}{0.082cm}}
\pgfpathcurveto{\pgfqpoint{2.864cm}{0.071cm}}{\pgfqpoint{2.879cm}{0.063cm}}{\pgfqpoint{2.879cm}{0.045cm}}
\pgfpathcurveto{\pgfqpoint{2.879cm}{0.027cm}}{\pgfqpoint{2.857cm}{0.015cm}}{\pgfqpoint{2.828cm}{0.015cm}}
\pgfpathcurveto{\pgfqpoint{2.789cm}{0.015cm}}{\pgfqpoint{2.76cm}{0.03cm}}{\pgfqpoint{2.76cm}{0.049cm}}
\pgfpathcurveto{\pgfqpoint{2.76cm}{0.063cm}}{\pgfqpoint{2.774cm}{0.075cm}}{\pgfqpoint{2.799cm}{0.082cm}}
\pgfpathlineto{\pgfqpoint{2.799cm}{0.082cm}}
\pgfpathlineto{\pgfqpoint{2.787cm}{0.086cm}}
\pgfpathcurveto{\pgfqpoint{2.752cm}{0.071cm}}{\pgfqpoint{2.743cm}{0.057cm}}{\pgfqpoint{2.743cm}{0.042cm}}
\pgfpathcurveto{\pgfqpoint{2.743cm}{0.017cm}}{\pgfqpoint{2.77cm}{0cm}}{\pgfqpoint{2.809cm}{0cm}}
\pgfpathcurveto{\pgfqpoint{2.856cm}{0cm}}{\pgfqpoint{2.893cm}{0.025cm}}{\pgfqpoint{2.893cm}{0.058cm}}
\pgfpathcurveto{\pgfqpoint{2.893cm}{0.067cm}}{\pgfqpoint{2.89cm}{0.074cm}}{\pgfqpoint{2.882cm}{0.082cm}}
\pgfpathcurveto{\pgfqpoint{2.863cm}{0.102cm}}{\pgfqpoint{2.791cm}{0.102cm}}{\pgfqpoint{2.781cm}{0.115cm}}
\pgfpathcurveto{\pgfqpoint{2.779cm}{0.117cm}}{\pgfqpoint{2.778cm}{0.118cm}}{\pgfqpoint{2.778cm}{0.122cm}}
\pgfpathcurveto{\pgfqpoint{2.778cm}{0.131cm}}{\pgfqpoint{2.801cm}{0.147cm}}{\pgfqpoint{2.822cm}{0.148cm}}
\pgfpathcurveto{\pgfqpoint{2.85cm}{0.15cm}}{\pgfqpoint{2.872cm}{0.176cm}}{\pgfqpoint{2.873cm}{0.203cm}}
\pgfpathcurveto{\pgfqpoint{2.873cm}{0.214cm}}{\pgfqpoint{2.872cm}{0.225cm}}{\pgfqpoint{2.863cm}{0.239cm}}
\pgfpathlineto{\pgfqpoint{2.879cm}{0.274cm}}
\pgfpathlineto{\pgfqpoint{2.874cm}{0.277cm}}
\pgfpathcurveto{\pgfqpoint{2.874cm}{0.277cm}}{\pgfqpoint{2.858cm}{0.261cm}}{\pgfqpoint{2.85cm}{0.26cm}}
\pgfpathcurveto{\pgfqpoint{2.842cm}{0.26cm}}{\pgfqpoint{2.838cm}{0.26cm}}{\pgfqpoint{2.832cm}{0.26cm}}
\pgfpathcurveto{\pgfqpoint{2.825cm}{0.26cm}}{\pgfqpoint{2.819cm}{0.261cm}}{\pgfqpoint{2.814cm}{0.261cm}}
\pgfpathlineto{\pgfqpoint{2.815cm}{0.253cm}}
\pgfpathcurveto{\pgfqpoint{2.835cm}{0.253cm}}{\pgfqpoint{2.848cm}{0.234cm}}{\pgfqpoint{2.848cm}{0.205cm}}
\pgfpathcurveto{\pgfqpoint{2.848cm}{0.173cm}}{\pgfqpoint{2.834cm}{0.155cm}}{\pgfqpoint{2.814cm}{0.155cm}}
\pgfpathcurveto{\pgfqpoint{2.794cm}{0.155cm}}{\pgfqpoint{2.781cm}{0.175cm}}{\pgfqpoint{2.781cm}{0.205cm}}
\pgfpathlineto{\pgfqpoint{2.781cm}{0.205cm}}
\pgfpathcurveto{\pgfqpoint{2.781cm}{0.234cm}}{\pgfqpoint{2.794cm}{0.253cm}}{\pgfqpoint{2.815cm}{0.253cm}}
\pgfpathlineto{\pgfqpoint{2.814cm}{0.261cm}}
\pgfpathcurveto{\pgfqpoint{2.781cm}{0.261cm}}{\pgfqpoint{2.756cm}{0.236cm}}{\pgfqpoint{2.756cm}{0.203cm}}
\pgfpathcurveto{\pgfqpoint{2.756cm}{0.177cm}}{\pgfqpoint{2.772cm}{0.158cm}}{\pgfqpoint{2.799cm}{0.15cm}}
\pgfpathlineto{\pgfqpoint{2.799cm}{0.149cm}}
\pgfpathcurveto{\pgfqpoint{2.772cm}{0.14cm}}{\pgfqpoint{2.757cm}{0.128cm}}{\pgfqpoint{2.757cm}{0.112cm}}
\pgfpathcurveto{\pgfqpoint{2.757cm}{0.101cm}}{\pgfqpoint{2.768cm}{0.092cm}}{\pgfqpoint{2.785cm}{0.087cm}}
\pgfpathclose
\pgfusepath{fill}
\definecolor{eps2pgf_color}{rgb}{0.01508,0.02539,0.02301}\pgfsetstrokecolor{eps2pgf_color}\pgfsetfillcolor{eps2pgf_color}
\pgfpathmoveto{\pgfqpoint{4.079cm}{0.272cm}}
\pgfpathlineto{\pgfqpoint{4.084cm}{0.272cm}}
\pgfpathlineto{\pgfqpoint{4.092cm}{0.26cm}}
\pgfpathlineto{\pgfqpoint{4.097cm}{0.26cm}}
\pgfpathlineto{\pgfqpoint{4.088cm}{0.273cm}}
\pgfpathcurveto{\pgfqpoint{4.093cm}{0.273cm}}{\pgfqpoint{4.096cm}{0.275cm}}{\pgfqpoint{4.096cm}{0.281cm}}
\pgfpathcurveto{\pgfqpoint{4.096cm}{0.287cm}}{\pgfqpoint{4.093cm}{0.289cm}}{\pgfqpoint{4.086cm}{0.289cm}}
\pgfpathlineto{\pgfqpoint{4.085cm}{0.285cm}}
\pgfpathcurveto{\pgfqpoint{4.088cm}{0.285cm}}{\pgfqpoint{4.091cm}{0.285cm}}{\pgfqpoint{4.091cm}{0.281cm}}
\pgfpathcurveto{\pgfqpoint{4.091cm}{0.277cm}}{\pgfqpoint{4.088cm}{0.276cm}}{\pgfqpoint{4.084cm}{0.276cm}}
\pgfpathlineto{\pgfqpoint{4.079cm}{0.276cm}}
\pgfpathlineto{\pgfqpoint{4.079cm}{0.285cm}}
\pgfpathlineto{\pgfqpoint{4.085cm}{0.285cm}}
\pgfpathlineto{\pgfqpoint{4.086cm}{0.289cm}}
\pgfpathlineto{\pgfqpoint{4.075cm}{0.289cm}}
\pgfpathlineto{\pgfqpoint{4.075cm}{0.26cm}}
\pgfpathlineto{\pgfqpoint{4.079cm}{0.26cm}}
\pgfpathlineto{\pgfqpoint{4.079cm}{0.272cm}}
\pgfpathclose
\pgfusepath{fill}
\pgfpathmoveto{\pgfqpoint{4.084cm}{0.249cm}}
\pgfpathlineto{\pgfqpoint{4.084cm}{0.253cm}}
\pgfpathcurveto{\pgfqpoint{4.073cm}{0.253cm}}{\pgfqpoint{4.064cm}{0.262cm}}{\pgfqpoint{4.064cm}{0.275cm}}
\pgfpathlineto{\pgfqpoint{4.064cm}{0.275cm}}
\pgfpathcurveto{\pgfqpoint{4.064cm}{0.287cm}}{\pgfqpoint{4.073cm}{0.296cm}}{\pgfqpoint{4.084cm}{0.296cm}}
\pgfpathcurveto{\pgfqpoint{4.096cm}{0.296cm}}{\pgfqpoint{4.105cm}{0.287cm}}{\pgfqpoint{4.105cm}{0.275cm}}
\pgfpathcurveto{\pgfqpoint{4.105cm}{0.262cm}}{\pgfqpoint{4.096cm}{0.253cm}}{\pgfqpoint{4.084cm}{0.253cm}}
\pgfpathlineto{\pgfqpoint{4.084cm}{0.249cm}}
\pgfpathcurveto{\pgfqpoint{4.098cm}{0.249cm}}{\pgfqpoint{4.11cm}{0.26cm}}{\pgfqpoint{4.11cm}{0.275cm}}
\pgfpathcurveto{\pgfqpoint{4.11cm}{0.289cm}}{\pgfqpoint{4.098cm}{0.3cm}}{\pgfqpoint{4.084cm}{0.3cm}}
\pgfpathcurveto{\pgfqpoint{4.071cm}{0.3cm}}{\pgfqpoint{4.059cm}{0.289cm}}{\pgfqpoint{4.059cm}{0.275cm}}
\pgfpathcurveto{\pgfqpoint{4.059cm}{0.26cm}}{\pgfqpoint{4.071cm}{0.249cm}}{\pgfqpoint{4.084cm}{0.249cm}}
\pgfpathclose
\pgfusepath{fill}
\pgfsetdash{}{0cm}
\pgfsetroundcap
\definecolor{eps2pgf_color}{rgb}{0.01405,0.02406,0.02196}\pgfsetstrokecolor{eps2pgf_color}\pgfsetfillcolor{eps2pgf_color}
\pgfpathmoveto{\pgfqpoint{2.737cm}{2.55cm}}
\pgfpathlineto{\pgfqpoint{2.737cm}{2.55cm}}
\pgfusepath{stroke}
\pgfsetdash{{0.106cm}{0.106cm}}{0cm}
\pgfpathmoveto{\pgfqpoint{2.844cm}{2.55cm}}
\pgfpathlineto{\pgfqpoint{6.318cm}{2.55cm}}
\pgfusepath{stroke}
\pgfsetdash{}{0cm}
\pgfpathmoveto{\pgfqpoint{6.371cm}{2.55cm}}
\pgfpathlineto{\pgfqpoint{6.371cm}{2.55cm}}
\pgfusepath{stroke}
\pgfsetdash{}{0cm}
\pgfsetlinewidth{0.705mm}
\pgfsetbuttcap
\pgfpathmoveto{\pgfqpoint{2.367cm}{2.55cm}}
\pgfpathlineto{\pgfqpoint{3.01cm}{2.55cm}}
\pgfusepath{stroke}
\pgfpathmoveto{\pgfqpoint{2.87cm}{2.431cm}}
\pgfpathlineto{\pgfqpoint{2.989cm}{2.55cm}}
\pgfpathlineto{\pgfqpoint{2.87cm}{2.67cm}}
\pgfpathlineto{\pgfqpoint{2.971cm}{2.67cm}}
\pgfpathlineto{\pgfqpoint{3.091cm}{2.55cm}}
\pgfpathlineto{\pgfqpoint{2.971cm}{2.431cm}}
\pgfpathclose
\pgfusepath{fill}
\pgfsetdash{}{0cm}
\pgfsetlinewidth{0.352mm}
\pgfpathmoveto{\pgfqpoint{4.554cm}{3.704cm}}
\pgfpathlineto{\pgfqpoint{4.554cm}{3.432cm}}
\pgfusepath{stroke}
\pgfpathmoveto{\pgfqpoint{4.41cm}{3.549cm}}
\pgfpathlineto{\pgfqpoint{4.436cm}{3.573cm}}
\pgfpathlineto{\pgfqpoint{4.554cm}{3.446cm}}
\pgfpathlineto{\pgfqpoint{4.673cm}{3.573cm}}
\pgfpathlineto{\pgfqpoint{4.698cm}{3.549cm}}
\pgfpathlineto{\pgfqpoint{4.554cm}{3.394cm}}
\pgfpathclose
\pgfusepath{fill}
\pgfsetdash{}{0cm}
\pgfsetlinewidth{0.705mm}
\pgfpathmoveto{\pgfqpoint{6.371cm}{2.582cm}}
\pgfpathlineto{\pgfqpoint{6.939cm}{2.582cm}}
\pgfusepath{stroke}
\pgfpathmoveto{\pgfqpoint{6.799cm}{2.462cm}}
\pgfpathlineto{\pgfqpoint{6.919cm}{2.582cm}}
\pgfpathlineto{\pgfqpoint{6.799cm}{2.701cm}}
\pgfpathlineto{\pgfqpoint{6.9cm}{2.701cm}}
\pgfpathlineto{\pgfqpoint{7.02cm}{2.582cm}}
\pgfpathlineto{\pgfqpoint{6.9cm}{2.462cm}}
\pgfpathlineto{\pgfqpoint{6.976cm}{2.462cm}}
\pgfpathclose
\pgfusepath{fill}
\pgfsetdash{}{0cm}
%\pgfpathmoveto{\pgfqpoint{9.328cm}{2.582cm}}
%\pgfpathlineto{\pgfqpoint{9.971cm}{2.582cm}}
%\pgfusepath{stroke}
%\pgfpathmoveto{\pgfqpoint{9.831cm}{2.462cm}}
%\pgfpathlineto{\pgfqpoint{9.95cm}{2.582cm}}
%\pgfpathlineto{\pgfqpoint{9.831cm}{2.701cm}}
%\pgfpathlineto{\pgfqpoint{9.932cm}{2.701cm}}
%\pgfpathlineto{\pgfqpoint{10.052cm}{2.582cm}}
%\pgfpathlineto{\pgfqpoint{9.932cm}{2.462cm}}
%\pgfpathclose
%\pgfusepath{fill}
%\pgfsetdash{}{0cm}
%\pgfpathmoveto{\pgfqpoint{11.84cm}{2.582cm}}
%\pgfpathlineto{\pgfqpoint{12.483cm}{2.582cm}}
%\pgfusepath{stroke}
%\pgfpathmoveto{\pgfqpoint{12.343cm}{2.462cm}}
%\pgfpathlineto{\pgfqpoint{12.462cm}{2.582cm}}
%\pgfpathlineto{\pgfqpoint{12.343cm}{2.701cm}}
%\pgfpathlineto{\pgfqpoint{12.444cm}{2.701cm}}
%\pgfpathlineto{\pgfqpoint{12.564cm}{2.582cm}}
%\pgfpathlineto{\pgfqpoint{12.444cm}{2.462cm}}
%\pgfpathclose
%\pgfusepath{fill}
\end{pgfpicture}
}
  \end{center}
\end{frame}

\begin{frame}[t,fragile=singleslide]{\inhibitglue \texttt{pstoedit}がすごい}
  \sffamily
  \begin{itemize}
    \item PDFやEPSを、さまざまな画像ファイルに変換してしまう、グラフィックス界のpandoc
    \begin{itemize}
      \item 実はGhostscriptのラッパー
      \item Ghostscript本体では非推奨になった\texttt{DELAYBIND}がデフォルトで有効という罠がある
      \item 結果として、イラレなどで生成されたEPSの多くは、素の\texttt{pstoedit}(したがって\texttt{purifyeps})で変換しようとすると意味不明なPostScriptエラーになる
      \item \texttt{DELAYBIND}を無効にするには、\texttt{purifyeps}のソースで\texttt{pstoedit}を呼んでいる部分で、\texttt{-nb}オプションを指定しなければならない
    \end{itemize}
  \end{itemize}

\end{frame}

\begin{frame}[t,fragile=singleslide]{\inhibitglue \texttt{pstoedit}がすごい}
  \sffamily
  \begin{itemize}
    \item PDFやEPSを、さまざまな画像ファイルに変換してしまう、グラフィックス界のpandoc
    \begin{itemize}
      \item 実はGhostscriptのラッパー
      \item Ghostscript本体では非推奨になった\texttt{DELAYBIND}がデフォルトで有効という罠がある
      \item 結果として、イラレなどで生成されたEPSの多くは、素の\texttt{pstoedit}(したがって\texttt{purifyeps})で変換しようとすると意味不明なPSエラーになる
      \item \texttt{DELAYBIND}を無効にするには、\texttt{purifyeps}のソースで\texttt{pstoedit}を呼んでいる部分で、\texttt{-nb}オプションを指定しなければならない
    \end{itemize}
  \end{itemize}

  \HUGE{21}{28}\color{black}\yasagoth
  \begin{center}
  「\TeX{}はGhostscriptから(ry
  \end{center}
\end{frame}

\begin{frame}[t,fragile=singleslide]{\inhibitglue ここまできたら……}
  \sffamily
  \begin{itemize}
    \item purified EPSをネイティブの\TeX{}ソースに変換できないか?
    \item TikZのインタフェースではPostScriptのシンタックスと違いすぎる
    \item それなら{\color{red}PGF}
    \begin{itemize}
      \item いまではドキュメントでもTikZと同じインタフェースのように扱われているが、PGF独自のインタフェースはかなりPostScriptっぽい
    \end{itemize}
  \end{itemize}
\end{frame}

\begin{frame}[t,fragile=singleslide]{\inhibitglue \texttt{Eps2pgf}}
  \sffamily
  \begin{itemize}
    \item すでにあった
    \item PostScript処理系ではなく、\hologo{METAPOST}の出力したEPSからPGFへのコンバーター
    \begin{itemize}
      \item Java製でメンテもされてなさそう
      \item Sourceforgeにポストされているがソースがない
    \end{itemize}
    \item 不安材料はあるけど、とにかく動く
  \end{itemize}
  \fontsize{7pt}{7pt}\selectfont
  \begin{tcolorbox}
  \begin{Verbatim}[commandchars=\\\{\}]
$ java -jar eps2pgf.jar image.eps -o image.tex
  \end{Verbatim}
  \end{tcolorbox}
  \begin{center}
  \resizebox{100mm}{!}{% Created by Eps2pgf 0.7.0 (build on 2008-08-24) on Thu Nov 08 14:40:42 JST 2018
\begin{pgfpicture}
\pgfpathmoveto{\pgfqpoint{0cm}{0cm}}
\pgfpathlineto{\pgfqpoint{14.725cm}{0cm}}
\pgfpathlineto{\pgfqpoint{14.725cm}{4.096cm}}
\pgfpathlineto{\pgfqpoint{0cm}{4.096cm}}
\pgfpathclose
\pgfusepath{clip}
\definecolor{eps2pgf_color}{rgb}{0,0,0}\pgfsetstrokecolor{eps2pgf_color}\pgfsetfillcolor{eps2pgf_color}
\pgftext[x=4.585cm,y=3.117cm,rotate=0]{\fontsize{10.04}{14.45}\selectfont{\textsf{purifyeps}}}
\pgftext[x=4.585cm,y=2.041cm,rotate=0]{\fontsize{10.04}{14.45}\selectfont{\textsf{pstoedit + mpost}}}
\pgftext[x=5.185cm,y=1.226cm,rotate=0]{\fontsize{10.04}{14.45}\selectfont{\textsf{+ \hologo{METAPOST}}}}
\pgftext[x=10.91cm,y=2.559cm,rotate=0]{\fontsize{10.04}{14.45}\selectfont{\textsf{Eps2pgf}}}
\pgftext[x=1.222cm,y=3.075cm,rotate=0]{\fontsize{10.04}{14.45}\selectfont{\textsf{EPS}}}
\pgftext[x=1.222cm,y=2.525cm,rotate=0]{\fontsize{10.04}{14.45}\selectfont{\textsf{(PostScript)}}}
\pgftext[x=13.35cm,y=3.075cm,rotate=0]{\fontsize{10.04}{14.45}\selectfont{\textsf{\TeX}}}
\pgftext[x=13.35cm,y=2.532cm,rotate=0]{\fontsize{10.04}{14.45}\selectfont{\textsf{(PGF)}}}
\pgftext[x=8.011cm,y=3.033cm,rotate=0]{\fontsize{10.04}{14.45}\selectfont{\textsf{purified EPS}}}
\pgftext[x=8.011cm,y=2.524cm,rotate=0]{\fontsize{10.04}{14.45}\selectfont{\textsf{(mps)}}}
\pgftext[x=4.529cm,y=3.9cm,rotate=0]{\fontsize{10.04}{14.45}\selectfont{\textsf{mpost.fmp}}}
\pgfsetdash{}{0cm}
\pgfsetlinewidth{0.352mm}
\pgfsetbeveljoin
\pgfpathmoveto{\pgfqpoint{0.018cm}{3.701cm}}
\pgfpathlineto{\pgfqpoint{0.018cm}{1.611cm}}
\pgfpathlineto{\pgfqpoint{1.949cm}{1.611cm}}
\pgfpathlineto{\pgfqpoint{2.367cm}{2.045cm}}
\pgfpathlineto{\pgfqpoint{2.367cm}{3.701cm}}
\pgfpathclose
\pgfusepath{stroke}
\definecolor{eps2pgf_color}{rgb}{0.6,0.6,0.6}\pgfsetstrokecolor{eps2pgf_color}\pgfsetfillcolor{eps2pgf_color}
\pgfpathmoveto{\pgfqpoint{1.949cm}{2.045cm}}
\pgfpathlineto{\pgfqpoint{1.949cm}{1.611cm}}
\pgfpathlineto{\pgfqpoint{2.367cm}{2.045cm}}
\pgfpathclose
\pgfusepath{fill}
\pgfsetdash{}{0cm}
\definecolor{eps2pgf_color}{rgb}{0,0,0}\pgfsetstrokecolor{eps2pgf_color}\pgfsetfillcolor{eps2pgf_color}
\pgfpathmoveto{\pgfqpoint{1.949cm}{2.045cm}}
\pgfpathlineto{\pgfqpoint{1.949cm}{1.611cm}}
\pgfpathlineto{\pgfqpoint{2.367cm}{2.045cm}}
\pgfpathclose
\pgfusepath{stroke}
\pgfsetdash{}{0cm}
\pgfpathmoveto{\pgfqpoint{12.217cm}{3.701cm}}
\pgfpathlineto{\pgfqpoint{12.217cm}{1.611cm}}
\pgfpathlineto{\pgfqpoint{14.149cm}{1.611cm}}
\pgfpathlineto{\pgfqpoint{14.566cm}{2.045cm}}
\pgfpathlineto{\pgfqpoint{14.566cm}{3.701cm}}
\pgfpathclose
\pgfusepath{stroke}
\definecolor{eps2pgf_color}{rgb}{0.6,0.6,0.6}\pgfsetstrokecolor{eps2pgf_color}\pgfsetfillcolor{eps2pgf_color}
\pgfpathmoveto{\pgfqpoint{14.149cm}{2.045cm}}
\pgfpathlineto{\pgfqpoint{14.149cm}{1.611cm}}
\pgfpathlineto{\pgfqpoint{14.566cm}{2.045cm}}
\pgfpathclose
\pgfusepath{fill}
\pgfsetdash{}{0cm}
\definecolor{eps2pgf_color}{rgb}{0,0,0}\pgfsetstrokecolor{eps2pgf_color}\pgfsetfillcolor{eps2pgf_color}
\pgfpathmoveto{\pgfqpoint{14.149cm}{2.045cm}}
\pgfpathlineto{\pgfqpoint{14.149cm}{1.611cm}}
\pgfpathlineto{\pgfqpoint{14.566cm}{2.045cm}}
\pgfpathclose
\pgfusepath{stroke}
\pgfsetdash{}{0cm}
\pgfpathmoveto{\pgfqpoint{6.697cm}{3.701cm}}
\pgfpathlineto{\pgfqpoint{6.697cm}{1.611cm}}
\pgfpathlineto{\pgfqpoint{8.91cm}{1.611cm}}
\pgfpathlineto{\pgfqpoint{9.328cm}{2.045cm}}
\pgfpathlineto{\pgfqpoint{9.328cm}{3.701cm}}
\pgfpathclose
\pgfusepath{stroke}
\definecolor{eps2pgf_color}{rgb}{0.6,0.6,0.6}\pgfsetstrokecolor{eps2pgf_color}\pgfsetfillcolor{eps2pgf_color}
\pgfpathmoveto{\pgfqpoint{8.91cm}{2.045cm}}
\pgfpathlineto{\pgfqpoint{8.91cm}{1.611cm}}
\pgfpathlineto{\pgfqpoint{9.328cm}{2.045cm}}
\pgfpathclose
\pgfusepath{fill}
\pgfsetdash{}{0cm}
\definecolor{eps2pgf_color}{rgb}{0,0,0}\pgfsetstrokecolor{eps2pgf_color}\pgfsetfillcolor{eps2pgf_color}
\pgfpathmoveto{\pgfqpoint{8.91cm}{2.045cm}}
\pgfpathlineto{\pgfqpoint{8.91cm}{1.611cm}}
\pgfpathlineto{\pgfqpoint{9.328cm}{2.045cm}}
\pgfpathclose
\pgfusepath{stroke}
\pgfsetdash{}{0cm}
\pgfsetmiterjoin
\pgfpathmoveto{\pgfqpoint{6.168cm}{1.611cm}}
\pgfpathlineto{\pgfqpoint{2.941cm}{1.611cm}}
\pgfpathcurveto{\pgfqpoint{2.828cm}{1.611cm}}{\pgfqpoint{2.737cm}{1.702cm}}{\pgfqpoint{2.737cm}{1.815cm}}
\pgfpathlineto{\pgfqpoint{2.737cm}{3.468cm}}
\pgfpathcurveto{\pgfqpoint{2.737cm}{3.58cm}}{\pgfqpoint{2.828cm}{3.671cm}}{\pgfqpoint{2.941cm}{3.671cm}}
\pgfpathlineto{\pgfqpoint{6.168cm}{3.671cm}}
\pgfpathcurveto{\pgfqpoint{6.28cm}{3.671cm}}{\pgfqpoint{6.371cm}{3.58cm}}{\pgfqpoint{6.371cm}{3.468cm}}
\pgfpathlineto{\pgfqpoint{6.371cm}{1.815cm}}
\pgfpathcurveto{\pgfqpoint{6.371cm}{1.702cm}}{\pgfqpoint{6.28cm}{1.611cm}}{\pgfqpoint{6.168cm}{1.611cm}}
\pgfpathclose
\pgfusepath{stroke}
\pgfsetdash{}{0cm}
\pgfpathmoveto{\pgfqpoint{11.636cm}{2.053cm}}
\pgfpathlineto{\pgfqpoint{9.944cm}{2.053cm}}
\pgfpathcurveto{\pgfqpoint{9.831cm}{2.053cm}}{\pgfqpoint{9.74cm}{2.144cm}}{\pgfqpoint{9.74cm}{2.256cm}}
\pgfpathlineto{\pgfqpoint{9.74cm}{2.907cm}}
\pgfpathcurveto{\pgfqpoint{9.74cm}{3.02cm}}{\pgfqpoint{9.831cm}{3.111cm}}{\pgfqpoint{9.944cm}{3.111cm}}
\pgfpathlineto{\pgfqpoint{11.636cm}{3.111cm}}
\pgfpathcurveto{\pgfqpoint{11.748cm}{3.111cm}}{\pgfqpoint{11.84cm}{3.02cm}}{\pgfqpoint{11.84cm}{2.907cm}}
\pgfpathlineto{\pgfqpoint{11.84cm}{2.256cm}}
\pgfpathcurveto{\pgfqpoint{11.84cm}{2.144cm}}{\pgfqpoint{11.748cm}{2.053cm}}{\pgfqpoint{11.636cm}{2.053cm}}
\pgfpathclose
\pgfusepath{stroke}
\definecolor{eps2pgf_color}{rgb}{0.01508,0.02539,0.02301}\pgfsetstrokecolor{eps2pgf_color}\pgfsetfillcolor{eps2pgf_color}
\pgfpathmoveto{\pgfqpoint{3.804cm}{0.541cm}}
\pgfpathcurveto{\pgfqpoint{3.804cm}{0.541cm}}{\pgfqpoint{3.758cm}{0.365cm}}{\pgfqpoint{3.379cm}{0.365cm}}
\pgfpathcurveto{\pgfqpoint{3cm}{0.365cm}}{\pgfqpoint{2.995cm}{0.51cm}}{\pgfqpoint{2.995cm}{0.522cm}}
\pgfpathcurveto{\pgfqpoint{3cm}{0.645cm}}{\pgfqpoint{3.199cm}{0.598cm}}{\pgfqpoint{3.199cm}{0.598cm}}
\pgfpathcurveto{\pgfqpoint{3.199cm}{0.598cm}}{\pgfqpoint{3.121cm}{0.595cm}}{\pgfqpoint{3.121cm}{0.525cm}}
\pgfpathcurveto{\pgfqpoint{3.121cm}{0.467cm}}{\pgfqpoint{3.212cm}{0.422cm}}{\pgfqpoint{3.396cm}{0.417cm}}
\pgfpathcurveto{\pgfqpoint{3.77cm}{0.406cm}}{\pgfqpoint{3.804cm}{0.541cm}}{\pgfqpoint{3.804cm}{0.541cm}}
\pgfpathclose
\pgfusepath{fill}
\pgfpathmoveto{\pgfqpoint{3.513cm}{1.376cm}}
\pgfpathcurveto{\pgfqpoint{3.513cm}{1.376cm}}{\pgfqpoint{3.649cm}{1.362cm}}{\pgfqpoint{3.629cm}{1.215cm}}
\pgfpathcurveto{\pgfqpoint{3.629cm}{1.215cm}}{\pgfqpoint{3.715cm}{1.16cm}}{\pgfqpoint{3.709cm}{1.082cm}}
\pgfpathcurveto{\pgfqpoint{3.703cm}{1.005cm}}{\pgfqpoint{3.645cm}{0.913cm}}{\pgfqpoint{3.46cm}{0.883cm}}
\pgfpathcurveto{\pgfqpoint{3.46cm}{0.883cm}}{\pgfqpoint{3.567cm}{0.912cm}}{\pgfqpoint{3.571cm}{0.972cm}}
\pgfpathcurveto{\pgfqpoint{3.575cm}{1.036cm}}{\pgfqpoint{3.465cm}{1.063cm}}{\pgfqpoint{3.465cm}{1.123cm}}
\pgfpathcurveto{\pgfqpoint{3.465cm}{1.183cm}}{\pgfqpoint{3.602cm}{1.191cm}}{\pgfqpoint{3.591cm}{1.292cm}}
\pgfpathcurveto{\pgfqpoint{3.583cm}{1.359cm}}{\pgfqpoint{3.513cm}{1.376cm}}{\pgfqpoint{3.513cm}{1.376cm}}
\pgfpathclose
\pgfusepath{fill}
\definecolor{eps2pgf_color}{rgb}{0.22266,0.69531,0.8164}\pgfsetstrokecolor{eps2pgf_color}\pgfsetfillcolor{eps2pgf_color}
\pgfpathmoveto{\pgfqpoint{3.405cm}{1.07cm}}
\pgfpathcurveto{\pgfqpoint{3.414cm}{1.075cm}}{\pgfqpoint{3.398cm}{1.086cm}}{\pgfqpoint{3.399cm}{1.107cm}}
\pgfpathcurveto{\pgfqpoint{3.4cm}{1.13cm}}{\pgfqpoint{3.409cm}{1.144cm}}{\pgfqpoint{3.396cm}{1.144cm}}
\pgfpathcurveto{\pgfqpoint{3.385cm}{1.143cm}}{\pgfqpoint{3.369cm}{1.124cm}}{\pgfqpoint{3.372cm}{1.103cm}}
\pgfpathcurveto{\pgfqpoint{3.375cm}{1.083cm}}{\pgfqpoint{3.394cm}{1.065cm}}{\pgfqpoint{3.405cm}{1.07cm}}
\pgfpathclose
\pgfusepath{fill}
\pgfpathmoveto{\pgfqpoint{3.391cm}{1.008cm}}
\pgfpathcurveto{\pgfqpoint{3.38cm}{1.001cm}}{\pgfqpoint{3.408cm}{0.993cm}}{\pgfqpoint{3.414cm}{0.972cm}}
\pgfpathcurveto{\pgfqpoint{3.422cm}{0.948cm}}{\pgfqpoint{3.413cm}{0.931cm}}{\pgfqpoint{3.433cm}{0.935cm}}
\pgfpathcurveto{\pgfqpoint{3.448cm}{0.938cm}}{\pgfqpoint{3.463cm}{0.963cm}}{\pgfqpoint{3.452cm}{0.984cm}}
\pgfpathcurveto{\pgfqpoint{3.44cm}{1.004cm}}{\pgfqpoint{3.404cm}{1.017cm}}{\pgfqpoint{3.391cm}{1.008cm}}
\pgfpathclose
\pgfusepath{fill}
\pgfpathmoveto{\pgfqpoint{3.345cm}{1.053cm}}
\pgfpathcurveto{\pgfqpoint{3.353cm}{1.059cm}}{\pgfqpoint{3.335cm}{1.067cm}}{\pgfqpoint{3.332cm}{1.087cm}}
\pgfpathcurveto{\pgfqpoint{3.328cm}{1.11cm}}{\pgfqpoint{3.334cm}{1.126cm}}{\pgfqpoint{3.321cm}{1.123cm}}
\pgfpathcurveto{\pgfqpoint{3.311cm}{1.12cm}}{\pgfqpoint{3.3cm}{1.097cm}}{\pgfqpoint{3.306cm}{1.078cm}}
\pgfpathcurveto{\pgfqpoint{3.313cm}{1.059cm}}{\pgfqpoint{3.336cm}{1.045cm}}{\pgfqpoint{3.345cm}{1.053cm}}
\pgfpathclose
\pgfusepath{fill}
\definecolor{eps2pgf_color}{rgb}{0.01508,0.02539,0.02301}\pgfsetstrokecolor{eps2pgf_color}\pgfsetfillcolor{eps2pgf_color}
\pgfpathmoveto{\pgfqpoint{3.778cm}{0.66cm}}
\pgfpathcurveto{\pgfqpoint{3.723cm}{0.754cm}}{\pgfqpoint{3.559cm}{0.78cm}}{\pgfqpoint{3.441cm}{0.784cm}}
\pgfpathcurveto{\pgfqpoint{3.338cm}{0.787cm}}{\pgfqpoint{3.207cm}{0.794cm}}{\pgfqpoint{3.207cm}{0.831cm}}
\pgfpathcurveto{\pgfqpoint{3.207cm}{0.881cm}}{\pgfqpoint{3.353cm}{0.852cm}}{\pgfqpoint{3.351cm}{0.914cm}}
\pgfpathcurveto{\pgfqpoint{3.35cm}{0.969cm}}{\pgfqpoint{3.213cm}{0.971cm}}{\pgfqpoint{3.232cm}{1.108cm}}
\pgfpathcurveto{\pgfqpoint{3.241cm}{1.172cm}}{\pgfqpoint{3.32cm}{1.206cm}}{\pgfqpoint{3.388cm}{1.222cm}}
\pgfpathcurveto{\pgfqpoint{3.457cm}{1.237cm}}{\pgfqpoint{3.545cm}{1.267cm}}{\pgfqpoint{3.546cm}{1.304cm}}
\pgfpathcurveto{\pgfqpoint{3.548cm}{1.342cm}}{\pgfqpoint{3.508cm}{1.355cm}}{\pgfqpoint{3.508cm}{1.355cm}}
\pgfpathcurveto{\pgfqpoint{3.508cm}{1.355cm}}{\pgfqpoint{3.531cm}{1.341cm}}{\pgfqpoint{3.522cm}{1.313cm}}
\pgfpathcurveto{\pgfqpoint{3.513cm}{1.285cm}}{\pgfqpoint{3.48cm}{1.283cm}}{\pgfqpoint{3.385cm}{1.281cm}}
\pgfpathcurveto{\pgfqpoint{3.305cm}{1.28cm}}{\pgfqpoint{3.188cm}{1.27cm}}{\pgfqpoint{3.111cm}{1.189cm}}
\pgfpathcurveto{\pgfqpoint{3.031cm}{1.104cm}}{\pgfqpoint{3.044cm}{0.964cm}}{\pgfqpoint{3.222cm}{0.908cm}}
\pgfpathcurveto{\pgfqpoint{3.222cm}{0.908cm}}{\pgfqpoint{3.029cm}{0.875cm}}{\pgfqpoint{3.033cm}{0.768cm}}
\pgfpathcurveto{\pgfqpoint{3.037cm}{0.661cm}}{\pgfqpoint{3.161cm}{0.635cm}}{\pgfqpoint{3.354cm}{0.631cm}}
\pgfpathcurveto{\pgfqpoint{3.55cm}{0.627cm}}{\pgfqpoint{3.624cm}{0.608cm}}{\pgfqpoint{3.624cm}{0.562cm}}
\pgfpathcurveto{\pgfqpoint{3.624cm}{0.524cm}}{\pgfqpoint{3.565cm}{0.503cm}}{\pgfqpoint{3.475cm}{0.5cm}}
\pgfpathcurveto{\pgfqpoint{3.38cm}{0.497cm}}{\pgfqpoint{3.313cm}{0.51cm}}{\pgfqpoint{3.313cm}{0.537cm}}
\pgfpathcurveto{\pgfqpoint{3.312cm}{0.572cm}}{\pgfqpoint{3.403cm}{0.555cm}}{\pgfqpoint{3.403cm}{0.555cm}}
\pgfpathcurveto{\pgfqpoint{3.403cm}{0.555cm}}{\pgfqpoint{3.369cm}{0.57cm}}{\pgfqpoint{3.325cm}{0.568cm}}
\pgfpathcurveto{\pgfqpoint{3.293cm}{0.566cm}}{\pgfqpoint{3.271cm}{0.55cm}}{\pgfqpoint{3.271cm}{0.53cm}}
\pgfpathcurveto{\pgfqpoint{3.27cm}{0.5cm}}{\pgfqpoint{3.297cm}{0.475cm}}{\pgfqpoint{3.37cm}{0.462cm}}
\pgfpathcurveto{\pgfqpoint{3.483cm}{0.444cm}}{\pgfqpoint{3.629cm}{0.445cm}}{\pgfqpoint{3.724cm}{0.503cm}}
\pgfpathcurveto{\pgfqpoint{3.795cm}{0.546cm}}{\pgfqpoint{3.803cm}{0.616cm}}{\pgfqpoint{3.778cm}{0.66cm}}
\pgfpathclose
\pgfusepath{fill}
\definecolor{eps2pgf_color}{rgb}{0.22266,0.69531,0.8164}\pgfsetstrokecolor{eps2pgf_color}\pgfsetfillcolor{eps2pgf_color}
\pgfpathmoveto{\pgfqpoint{3.009cm}{0.162cm}}
\pgfpathcurveto{\pgfqpoint{3.009cm}{0.144cm}}{\pgfqpoint{3.012cm}{0.139cm}}{\pgfqpoint{3.028cm}{0.139cm}}
\pgfpathlineto{\pgfqpoint{3.028cm}{0.133cm}}
\pgfpathlineto{\pgfqpoint{2.965cm}{0.133cm}}
\pgfpathlineto{\pgfqpoint{2.965cm}{0.139cm}}
\pgfpathcurveto{\pgfqpoint{2.982cm}{0.139cm}}{\pgfqpoint{2.985cm}{0.143cm}}{\pgfqpoint{2.985cm}{0.159cm}}
\pgfpathlineto{\pgfqpoint{2.985cm}{0.22cm}}
\pgfpathcurveto{\pgfqpoint{2.985cm}{0.236cm}}{\pgfqpoint{2.976cm}{0.246cm}}{\pgfqpoint{2.961cm}{0.246cm}}
\pgfpathcurveto{\pgfqpoint{2.948cm}{0.246cm}}{\pgfqpoint{2.938cm}{0.24cm}}{\pgfqpoint{2.926cm}{0.223cm}}
\pgfpathlineto{\pgfqpoint{2.926cm}{0.156cm}}
\pgfpathcurveto{\pgfqpoint{2.926cm}{0.143cm}}{\pgfqpoint{2.932cm}{0.139cm}}{\pgfqpoint{2.946cm}{0.139cm}}
\pgfpathlineto{\pgfqpoint{2.946cm}{0.133cm}}
\pgfpathlineto{\pgfqpoint{2.882cm}{0.133cm}}
\pgfpathlineto{\pgfqpoint{2.882cm}{0.139cm}}
\pgfpathcurveto{\pgfqpoint{2.9cm}{0.139cm}}{\pgfqpoint{2.903cm}{0.143cm}}{\pgfqpoint{2.903cm}{0.162cm}}
\pgfpathlineto{\pgfqpoint{2.903cm}{0.323cm}}
\pgfpathcurveto{\pgfqpoint{2.903cm}{0.339cm}}{\pgfqpoint{2.899cm}{0.342cm}}{\pgfqpoint{2.882cm}{0.342cm}}
\pgfpathlineto{\pgfqpoint{2.882cm}{0.348cm}}
\pgfpathlineto{\pgfqpoint{2.92cm}{0.352cm}}
\pgfpathlineto{\pgfqpoint{2.926cm}{0.352cm}}
\pgfpathlineto{\pgfqpoint{2.926cm}{0.235cm}}
\pgfpathcurveto{\pgfqpoint{2.942cm}{0.253cm}}{\pgfqpoint{2.953cm}{0.26cm}}{\pgfqpoint{2.968cm}{0.26cm}}
\pgfpathcurveto{\pgfqpoint{2.989cm}{0.26cm}}{\pgfqpoint{3.009cm}{0.246cm}}{\pgfqpoint{3.009cm}{0.217cm}}
\pgfpathlineto{\pgfqpoint{3.009cm}{0.162cm}}
\pgfpathclose
\pgfusepath{fill}
\pgfpathmoveto{\pgfqpoint{3.164cm}{0.194cm}}
\pgfpathcurveto{\pgfqpoint{3.164cm}{0.232cm}}{\pgfqpoint{3.136cm}{0.26cm}}{\pgfqpoint{3.098cm}{0.26cm}}
\pgfpathlineto{\pgfqpoint{3.098cm}{0.253cm}}
\pgfpathcurveto{\pgfqpoint{3.126cm}{0.253cm}}{\pgfqpoint{3.139cm}{0.225cm}}{\pgfqpoint{3.139cm}{0.194cm}}
\pgfpathlineto{\pgfqpoint{3.139cm}{0.194cm}}
\pgfpathcurveto{\pgfqpoint{3.139cm}{0.164cm}}{\pgfqpoint{3.127cm}{0.136cm}}{\pgfqpoint{3.098cm}{0.136cm}}
\pgfpathcurveto{\pgfqpoint{3.07cm}{0.136cm}}{\pgfqpoint{3.057cm}{0.163cm}}{\pgfqpoint{3.057cm}{0.194cm}}
\pgfpathcurveto{\pgfqpoint{3.057cm}{0.23cm}}{\pgfqpoint{3.073cm}{0.253cm}}{\pgfqpoint{3.098cm}{0.253cm}}
\pgfpathlineto{\pgfqpoint{3.098cm}{0.26cm}}
\pgfpathcurveto{\pgfqpoint{3.06cm}{0.26cm}}{\pgfqpoint{3.032cm}{0.232cm}}{\pgfqpoint{3.032cm}{0.194cm}}
\pgfpathcurveto{\pgfqpoint{3.032cm}{0.157cm}}{\pgfqpoint{3.06cm}{0.129cm}}{\pgfqpoint{3.098cm}{0.129cm}}
\pgfpathcurveto{\pgfqpoint{3.136cm}{0.129cm}}{\pgfqpoint{3.164cm}{0.157cm}}{\pgfqpoint{3.164cm}{0.194cm}}
\pgfpathclose
\pgfusepath{fill}
\pgfpathmoveto{\pgfqpoint{3.247cm}{0.222cm}}
\pgfpathcurveto{\pgfqpoint{3.244cm}{0.243cm}}{\pgfqpoint{3.231cm}{0.253cm}}{\pgfqpoint{3.217cm}{0.253cm}}
\pgfpathcurveto{\pgfqpoint{3.205cm}{0.253cm}}{\pgfqpoint{3.197cm}{0.246cm}}{\pgfqpoint{3.197cm}{0.236cm}}
\pgfpathcurveto{\pgfqpoint{3.197cm}{0.205cm}}{\pgfqpoint{3.26cm}{0.213cm}}{\pgfqpoint{3.26cm}{0.167cm}}
\pgfpathcurveto{\pgfqpoint{3.26cm}{0.147cm}}{\pgfqpoint{3.243cm}{0.13cm}}{\pgfqpoint{3.222cm}{0.13cm}}
\pgfpathcurveto{\pgfqpoint{3.209cm}{0.13cm}}{\pgfqpoint{3.197cm}{0.139cm}}{\pgfqpoint{3.192cm}{0.139cm}}
\pgfpathcurveto{\pgfqpoint{3.187cm}{0.139cm}}{\pgfqpoint{3.184cm}{0.135cm}}{\pgfqpoint{3.184cm}{0.13cm}}
\pgfpathlineto{\pgfqpoint{3.178cm}{0.13cm}}
\pgfpathlineto{\pgfqpoint{3.178cm}{0.172cm}}
\pgfpathlineto{\pgfqpoint{3.184cm}{0.172cm}}
\pgfpathcurveto{\pgfqpoint{3.189cm}{0.147cm}}{\pgfqpoint{3.208cm}{0.136cm}}{\pgfqpoint{3.222cm}{0.136cm}}
\pgfpathcurveto{\pgfqpoint{3.234cm}{0.136cm}}{\pgfqpoint{3.243cm}{0.146cm}}{\pgfqpoint{3.243cm}{0.158cm}}
\pgfpathcurveto{\pgfqpoint{3.243cm}{0.192cm}}{\pgfqpoint{3.18cm}{0.183cm}}{\pgfqpoint{3.18cm}{0.224cm}}
\pgfpathcurveto{\pgfqpoint{3.18cm}{0.243cm}}{\pgfqpoint{3.196cm}{0.26cm}}{\pgfqpoint{3.216cm}{0.26cm}}
\pgfpathcurveto{\pgfqpoint{3.229cm}{0.26cm}}{\pgfqpoint{3.233cm}{0.254cm}}{\pgfqpoint{3.241cm}{0.254cm}}
\pgfpathcurveto{\pgfqpoint{3.244cm}{0.254cm}}{\pgfqpoint{3.246cm}{0.256cm}}{\pgfqpoint{3.248cm}{0.26cm}}
\pgfpathlineto{\pgfqpoint{3.254cm}{0.26cm}}
\pgfpathlineto{\pgfqpoint{3.254cm}{0.222cm}}
\pgfpathlineto{\pgfqpoint{3.247cm}{0.222cm}}
\pgfpathclose
\pgfusepath{fill}
\definecolor{eps2pgf_color}{rgb}{0.01508,0.02539,0.02301}\pgfsetstrokecolor{eps2pgf_color}\pgfsetfillcolor{eps2pgf_color}
\pgfpathmoveto{\pgfqpoint{3.452cm}{0.222cm}}
\pgfpathcurveto{\pgfqpoint{3.45cm}{0.242cm}}{\pgfqpoint{3.437cm}{0.253cm}}{\pgfqpoint{3.422cm}{0.253cm}}
\pgfpathcurveto{\pgfqpoint{3.41cm}{0.253cm}}{\pgfqpoint{3.402cm}{0.246cm}}{\pgfqpoint{3.402cm}{0.236cm}}
\pgfpathcurveto{\pgfqpoint{3.402cm}{0.205cm}}{\pgfqpoint{3.466cm}{0.213cm}}{\pgfqpoint{3.466cm}{0.167cm}}
\pgfpathcurveto{\pgfqpoint{3.466cm}{0.147cm}}{\pgfqpoint{3.448cm}{0.129cm}}{\pgfqpoint{3.427cm}{0.129cm}}
\pgfpathcurveto{\pgfqpoint{3.414cm}{0.129cm}}{\pgfqpoint{3.403cm}{0.138cm}}{\pgfqpoint{3.397cm}{0.138cm}}
\pgfpathcurveto{\pgfqpoint{3.392cm}{0.138cm}}{\pgfqpoint{3.389cm}{0.134cm}}{\pgfqpoint{3.389cm}{0.13cm}}
\pgfpathlineto{\pgfqpoint{3.383cm}{0.13cm}}
\pgfpathlineto{\pgfqpoint{3.383cm}{0.172cm}}
\pgfpathlineto{\pgfqpoint{3.389cm}{0.172cm}}
\pgfpathcurveto{\pgfqpoint{3.394cm}{0.146cm}}{\pgfqpoint{3.413cm}{0.136cm}}{\pgfqpoint{3.428cm}{0.136cm}}
\pgfpathcurveto{\pgfqpoint{3.44cm}{0.136cm}}{\pgfqpoint{3.449cm}{0.145cm}}{\pgfqpoint{3.449cm}{0.158cm}}
\pgfpathcurveto{\pgfqpoint{3.449cm}{0.192cm}}{\pgfqpoint{3.385cm}{0.182cm}}{\pgfqpoint{3.385cm}{0.224cm}}
\pgfpathcurveto{\pgfqpoint{3.385cm}{0.243cm}}{\pgfqpoint{3.401cm}{0.26cm}}{\pgfqpoint{3.422cm}{0.26cm}}
\pgfpathcurveto{\pgfqpoint{3.434cm}{0.26cm}}{\pgfqpoint{3.438cm}{0.254cm}}{\pgfqpoint{3.446cm}{0.254cm}}
\pgfpathcurveto{\pgfqpoint{3.449cm}{0.254cm}}{\pgfqpoint{3.451cm}{0.255cm}}{\pgfqpoint{3.453cm}{0.26cm}}
\pgfpathlineto{\pgfqpoint{3.459cm}{0.26cm}}
\pgfpathlineto{\pgfqpoint{3.459cm}{0.222cm}}
\pgfpathlineto{\pgfqpoint{3.452cm}{0.222cm}}
\pgfpathclose
\pgfusepath{fill}
\pgfpathmoveto{\pgfqpoint{3.592cm}{0.215cm}}
\pgfpathlineto{\pgfqpoint{3.586cm}{0.215cm}}
\pgfpathcurveto{\pgfqpoint{3.581cm}{0.238cm}}{\pgfqpoint{3.565cm}{0.252cm}}{\pgfqpoint{3.545cm}{0.252cm}}
\pgfpathcurveto{\pgfqpoint{3.521cm}{0.252cm}}{\pgfqpoint{3.504cm}{0.231cm}}{\pgfqpoint{3.504cm}{0.201cm}}
\pgfpathcurveto{\pgfqpoint{3.504cm}{0.167cm}}{\pgfqpoint{3.525cm}{0.139cm}}{\pgfqpoint{3.55cm}{0.139cm}}
\pgfpathcurveto{\pgfqpoint{3.565cm}{0.139cm}}{\pgfqpoint{3.582cm}{0.147cm}}{\pgfqpoint{3.595cm}{0.165cm}}
\pgfpathlineto{\pgfqpoint{3.595cm}{0.156cm}}
\pgfpathcurveto{\pgfqpoint{3.582cm}{0.138cm}}{\pgfqpoint{3.564cm}{0.128cm}}{\pgfqpoint{3.542cm}{0.128cm}}
\pgfpathcurveto{\pgfqpoint{3.506cm}{0.128cm}}{\pgfqpoint{3.481cm}{0.154cm}}{\pgfqpoint{3.481cm}{0.191cm}}
\pgfpathcurveto{\pgfqpoint{3.481cm}{0.229cm}}{\pgfqpoint{3.509cm}{0.259cm}}{\pgfqpoint{3.543cm}{0.259cm}}
\pgfpathcurveto{\pgfqpoint{3.562cm}{0.259cm}}{\pgfqpoint{3.571cm}{0.252cm}}{\pgfqpoint{3.578cm}{0.252cm}}
\pgfpathcurveto{\pgfqpoint{3.581cm}{0.252cm}}{\pgfqpoint{3.584cm}{0.254cm}}{\pgfqpoint{3.586cm}{0.259cm}}
\pgfpathlineto{\pgfqpoint{3.592cm}{0.259cm}}
\pgfpathlineto{\pgfqpoint{3.592cm}{0.215cm}}
\pgfpathclose
\pgfusepath{fill}
\pgfpathmoveto{\pgfqpoint{3.658cm}{0.156cm}}
\pgfpathcurveto{\pgfqpoint{3.658cm}{0.143cm}}{\pgfqpoint{3.662cm}{0.139cm}}{\pgfqpoint{3.678cm}{0.139cm}}
\pgfpathlineto{\pgfqpoint{3.678cm}{0.133cm}}
\pgfpathlineto{\pgfqpoint{3.614cm}{0.133cm}}
\pgfpathlineto{\pgfqpoint{3.614cm}{0.139cm}}
\pgfpathcurveto{\pgfqpoint{3.633cm}{0.139cm}}{\pgfqpoint{3.635cm}{0.144cm}}{\pgfqpoint{3.635cm}{0.163cm}}
\pgfpathlineto{\pgfqpoint{3.635cm}{0.23cm}}
\pgfpathcurveto{\pgfqpoint{3.635cm}{0.248cm}}{\pgfqpoint{3.63cm}{0.25cm}}{\pgfqpoint{3.61cm}{0.25cm}}
\pgfpathlineto{\pgfqpoint{3.61cm}{0.256cm}}
\pgfpathlineto{\pgfqpoint{3.652cm}{0.26cm}}
\pgfpathlineto{\pgfqpoint{3.658cm}{0.26cm}}
\pgfpathlineto{\pgfqpoint{3.658cm}{0.233cm}}
\pgfpathcurveto{\pgfqpoint{3.681cm}{0.253cm}}{\pgfqpoint{3.691cm}{0.26cm}}{\pgfqpoint{3.699cm}{0.26cm}}
\pgfpathcurveto{\pgfqpoint{3.706cm}{0.26cm}}{\pgfqpoint{3.715cm}{0.256cm}}{\pgfqpoint{3.722cm}{0.25cm}}
\pgfpathlineto{\pgfqpoint{3.712cm}{0.228cm}}
\pgfpathcurveto{\pgfqpoint{3.703cm}{0.234cm}}{\pgfqpoint{3.692cm}{0.239cm}}{\pgfqpoint{3.684cm}{0.239cm}}
\pgfpathcurveto{\pgfqpoint{3.676cm}{0.239cm}}{\pgfqpoint{3.669cm}{0.235cm}}{\pgfqpoint{3.658cm}{0.224cm}}
\pgfpathlineto{\pgfqpoint{3.658cm}{0.156cm}}
\pgfpathclose
\pgfusepath{fill}
\pgfpathmoveto{\pgfqpoint{3.786cm}{0.308cm}}
\pgfpathcurveto{\pgfqpoint{3.786cm}{0.3cm}}{\pgfqpoint{3.779cm}{0.293cm}}{\pgfqpoint{3.771cm}{0.293cm}}
\pgfpathcurveto{\pgfqpoint{3.763cm}{0.293cm}}{\pgfqpoint{3.756cm}{0.3cm}}{\pgfqpoint{3.756cm}{0.308cm}}
\pgfpathcurveto{\pgfqpoint{3.756cm}{0.317cm}}{\pgfqpoint{3.763cm}{0.323cm}}{\pgfqpoint{3.771cm}{0.323cm}}
\pgfpathcurveto{\pgfqpoint{3.779cm}{0.323cm}}{\pgfqpoint{3.786cm}{0.317cm}}{\pgfqpoint{3.786cm}{0.308cm}}
\pgfpathclose
\pgfusepath{fill}
\pgfpathmoveto{\pgfqpoint{3.783cm}{0.156cm}}
\pgfpathcurveto{\pgfqpoint{3.783cm}{0.144cm}}{\pgfqpoint{3.786cm}{0.139cm}}{\pgfqpoint{3.803cm}{0.139cm}}
\pgfpathlineto{\pgfqpoint{3.803cm}{0.133cm}}
\pgfpathlineto{\pgfqpoint{3.736cm}{0.133cm}}
\pgfpathlineto{\pgfqpoint{3.736cm}{0.139cm}}
\pgfpathcurveto{\pgfqpoint{3.755cm}{0.139cm}}{\pgfqpoint{3.759cm}{0.141cm}}{\pgfqpoint{3.759cm}{0.156cm}}
\pgfpathlineto{\pgfqpoint{3.759cm}{0.232cm}}
\pgfpathcurveto{\pgfqpoint{3.759cm}{0.249cm}}{\pgfqpoint{3.754cm}{0.25cm}}{\pgfqpoint{3.736cm}{0.25cm}}
\pgfpathlineto{\pgfqpoint{3.736cm}{0.256cm}}
\pgfpathlineto{\pgfqpoint{3.777cm}{0.26cm}}
\pgfpathlineto{\pgfqpoint{3.783cm}{0.26cm}}
\pgfpathlineto{\pgfqpoint{3.783cm}{0.156cm}}
\pgfpathclose
\pgfusepath{fill}
\pgfpathmoveto{\pgfqpoint{3.825cm}{0.059cm}}
\pgfpathcurveto{\pgfqpoint{3.825cm}{0.046cm}}{\pgfqpoint{3.82cm}{0.042cm}}{\pgfqpoint{3.804cm}{0.042cm}}
\pgfpathlineto{\pgfqpoint{3.804cm}{0.036cm}}
\pgfpathlineto{\pgfqpoint{3.868cm}{0.036cm}}
\pgfpathlineto{\pgfqpoint{3.868cm}{0.042cm}}
\pgfpathcurveto{\pgfqpoint{3.851cm}{0.042cm}}{\pgfqpoint{3.848cm}{0.048cm}}{\pgfqpoint{3.848cm}{0.063cm}}
\pgfpathlineto{\pgfqpoint{3.848cm}{0.144cm}}
\pgfpathcurveto{\pgfqpoint{3.862cm}{0.133cm}}{\pgfqpoint{3.871cm}{0.129cm}}{\pgfqpoint{3.884cm}{0.129cm}}
\pgfpathlineto{\pgfqpoint{3.881cm}{0.137cm}}
\pgfpathcurveto{\pgfqpoint{3.87cm}{0.137cm}}{\pgfqpoint{3.857cm}{0.143cm}}{\pgfqpoint{3.848cm}{0.153cm}}
\pgfpathlineto{\pgfqpoint{3.848cm}{0.23cm}}
\pgfpathlineto{\pgfqpoint{3.848cm}{0.23cm}}
\pgfpathcurveto{\pgfqpoint{3.859cm}{0.243cm}}{\pgfqpoint{3.868cm}{0.248cm}}{\pgfqpoint{3.881cm}{0.248cm}}
\pgfpathcurveto{\pgfqpoint{3.903cm}{0.248cm}}{\pgfqpoint{3.919cm}{0.224cm}}{\pgfqpoint{3.919cm}{0.191cm}}
\pgfpathcurveto{\pgfqpoint{3.919cm}{0.161cm}}{\pgfqpoint{3.903cm}{0.137cm}}{\pgfqpoint{3.881cm}{0.137cm}}
\pgfpathlineto{\pgfqpoint{3.884cm}{0.129cm}}
\pgfpathcurveto{\pgfqpoint{3.919cm}{0.129cm}}{\pgfqpoint{3.944cm}{0.157cm}}{\pgfqpoint{3.944cm}{0.195cm}}
\pgfpathcurveto{\pgfqpoint{3.944cm}{0.232cm}}{\pgfqpoint{3.921cm}{0.26cm}}{\pgfqpoint{3.89cm}{0.26cm}}
\pgfpathcurveto{\pgfqpoint{3.874cm}{0.26cm}}{\pgfqpoint{3.863cm}{0.254cm}}{\pgfqpoint{3.848cm}{0.239cm}}
\pgfpathlineto{\pgfqpoint{3.848cm}{0.26cm}}
\pgfpathlineto{\pgfqpoint{3.842cm}{0.26cm}}
\pgfpathlineto{\pgfqpoint{3.804cm}{0.256cm}}
\pgfpathlineto{\pgfqpoint{3.804cm}{0.25cm}}
\pgfpathcurveto{\pgfqpoint{3.822cm}{0.25cm}}{\pgfqpoint{3.825cm}{0.248cm}}{\pgfqpoint{3.825cm}{0.231cm}}
\pgfpathlineto{\pgfqpoint{3.825cm}{0.059cm}}
\pgfpathclose
\pgfusepath{fill}
\pgfpathmoveto{\pgfqpoint{3.995cm}{0.259cm}}
\pgfpathlineto{\pgfqpoint{4.037cm}{0.259cm}}
\pgfpathlineto{\pgfqpoint{4.037cm}{0.249cm}}
\pgfpathlineto{\pgfqpoint{3.995cm}{0.249cm}}
\pgfpathlineto{\pgfqpoint{3.995cm}{0.17cm}}
\pgfpathcurveto{\pgfqpoint{3.995cm}{0.153cm}}{\pgfqpoint{4.003cm}{0.144cm}}{\pgfqpoint{4.016cm}{0.144cm}}
\pgfpathcurveto{\pgfqpoint{4.025cm}{0.144cm}}{\pgfqpoint{4.032cm}{0.148cm}}{\pgfqpoint{4.039cm}{0.156cm}}
\pgfpathlineto{\pgfqpoint{4.043cm}{0.152cm}}
\pgfpathcurveto{\pgfqpoint{4.032cm}{0.138cm}}{\pgfqpoint{4.02cm}{0.131cm}}{\pgfqpoint{4.006cm}{0.131cm}}
\pgfpathcurveto{\pgfqpoint{3.986cm}{0.131cm}}{\pgfqpoint{3.971cm}{0.146cm}}{\pgfqpoint{3.971cm}{0.167cm}}
\pgfpathlineto{\pgfqpoint{3.971cm}{0.249cm}}
\pgfpathlineto{\pgfqpoint{3.949cm}{0.249cm}}
\pgfpathlineto{\pgfqpoint{3.949cm}{0.255cm}}
\pgfpathcurveto{\pgfqpoint{3.965cm}{0.264cm}}{\pgfqpoint{3.979cm}{0.279cm}}{\pgfqpoint{3.989cm}{0.3cm}}
\pgfpathlineto{\pgfqpoint{3.995cm}{0.3cm}}
\pgfpathlineto{\pgfqpoint{3.995cm}{0.259cm}}
\pgfpathclose
\pgfusepath{fill}
\definecolor{eps2pgf_color}{rgb}{0.22266,0.69531,0.8164}\pgfsetstrokecolor{eps2pgf_color}\pgfsetfillcolor{eps2pgf_color}
\pgfpathmoveto{\pgfqpoint{3.317cm}{0.257cm}}
\pgfpathlineto{\pgfqpoint{3.359cm}{0.257cm}}
\pgfpathlineto{\pgfqpoint{3.359cm}{0.248cm}}
\pgfpathlineto{\pgfqpoint{3.317cm}{0.248cm}}
\pgfpathlineto{\pgfqpoint{3.317cm}{0.168cm}}
\pgfpathcurveto{\pgfqpoint{3.317cm}{0.151cm}}{\pgfqpoint{3.326cm}{0.142cm}}{\pgfqpoint{3.339cm}{0.142cm}}
\pgfpathcurveto{\pgfqpoint{3.348cm}{0.142cm}}{\pgfqpoint{3.354cm}{0.146cm}}{\pgfqpoint{3.362cm}{0.154cm}}
\pgfpathlineto{\pgfqpoint{3.365cm}{0.15cm}}
\pgfpathcurveto{\pgfqpoint{3.355cm}{0.136cm}}{\pgfqpoint{3.342cm}{0.129cm}}{\pgfqpoint{3.329cm}{0.129cm}}
\pgfpathcurveto{\pgfqpoint{3.308cm}{0.129cm}}{\pgfqpoint{3.294cm}{0.144cm}}{\pgfqpoint{3.294cm}{0.165cm}}
\pgfpathlineto{\pgfqpoint{3.294cm}{0.248cm}}
\pgfpathlineto{\pgfqpoint{3.271cm}{0.248cm}}
\pgfpathlineto{\pgfqpoint{3.271cm}{0.254cm}}
\pgfpathcurveto{\pgfqpoint{3.287cm}{0.262cm}}{\pgfqpoint{3.301cm}{0.277cm}}{\pgfqpoint{3.311cm}{0.298cm}}
\pgfpathlineto{\pgfqpoint{3.317cm}{0.298cm}}
\pgfpathlineto{\pgfqpoint{3.317cm}{0.257cm}}
\pgfpathclose
\pgfusepath{fill}
\pgfpathmoveto{\pgfqpoint{2.785cm}{0.087cm}}
\pgfpathcurveto{\pgfqpoint{2.786cm}{0.087cm}}{\pgfqpoint{2.786cm}{0.087cm}}{\pgfqpoint{2.787cm}{0.087cm}}
\pgfpathlineto{\pgfqpoint{2.787cm}{0.086cm}}
\pgfpathlineto{\pgfqpoint{2.799cm}{0.082cm}}
\pgfpathcurveto{\pgfqpoint{2.864cm}{0.071cm}}{\pgfqpoint{2.879cm}{0.063cm}}{\pgfqpoint{2.879cm}{0.045cm}}
\pgfpathcurveto{\pgfqpoint{2.879cm}{0.027cm}}{\pgfqpoint{2.857cm}{0.015cm}}{\pgfqpoint{2.828cm}{0.015cm}}
\pgfpathcurveto{\pgfqpoint{2.789cm}{0.015cm}}{\pgfqpoint{2.76cm}{0.03cm}}{\pgfqpoint{2.76cm}{0.049cm}}
\pgfpathcurveto{\pgfqpoint{2.76cm}{0.063cm}}{\pgfqpoint{2.774cm}{0.075cm}}{\pgfqpoint{2.799cm}{0.082cm}}
\pgfpathlineto{\pgfqpoint{2.799cm}{0.082cm}}
\pgfpathlineto{\pgfqpoint{2.787cm}{0.086cm}}
\pgfpathcurveto{\pgfqpoint{2.752cm}{0.071cm}}{\pgfqpoint{2.743cm}{0.057cm}}{\pgfqpoint{2.743cm}{0.042cm}}
\pgfpathcurveto{\pgfqpoint{2.743cm}{0.017cm}}{\pgfqpoint{2.77cm}{0cm}}{\pgfqpoint{2.809cm}{0cm}}
\pgfpathcurveto{\pgfqpoint{2.856cm}{0cm}}{\pgfqpoint{2.893cm}{0.025cm}}{\pgfqpoint{2.893cm}{0.058cm}}
\pgfpathcurveto{\pgfqpoint{2.893cm}{0.067cm}}{\pgfqpoint{2.89cm}{0.074cm}}{\pgfqpoint{2.882cm}{0.082cm}}
\pgfpathcurveto{\pgfqpoint{2.863cm}{0.102cm}}{\pgfqpoint{2.791cm}{0.102cm}}{\pgfqpoint{2.781cm}{0.115cm}}
\pgfpathcurveto{\pgfqpoint{2.779cm}{0.117cm}}{\pgfqpoint{2.778cm}{0.118cm}}{\pgfqpoint{2.778cm}{0.122cm}}
\pgfpathcurveto{\pgfqpoint{2.778cm}{0.131cm}}{\pgfqpoint{2.801cm}{0.147cm}}{\pgfqpoint{2.822cm}{0.148cm}}
\pgfpathcurveto{\pgfqpoint{2.85cm}{0.15cm}}{\pgfqpoint{2.872cm}{0.176cm}}{\pgfqpoint{2.873cm}{0.203cm}}
\pgfpathcurveto{\pgfqpoint{2.873cm}{0.214cm}}{\pgfqpoint{2.872cm}{0.225cm}}{\pgfqpoint{2.863cm}{0.239cm}}
\pgfpathlineto{\pgfqpoint{2.879cm}{0.274cm}}
\pgfpathlineto{\pgfqpoint{2.874cm}{0.277cm}}
\pgfpathcurveto{\pgfqpoint{2.874cm}{0.277cm}}{\pgfqpoint{2.858cm}{0.261cm}}{\pgfqpoint{2.85cm}{0.26cm}}
\pgfpathcurveto{\pgfqpoint{2.842cm}{0.26cm}}{\pgfqpoint{2.838cm}{0.26cm}}{\pgfqpoint{2.832cm}{0.26cm}}
\pgfpathcurveto{\pgfqpoint{2.825cm}{0.26cm}}{\pgfqpoint{2.819cm}{0.261cm}}{\pgfqpoint{2.814cm}{0.261cm}}
\pgfpathlineto{\pgfqpoint{2.815cm}{0.253cm}}
\pgfpathcurveto{\pgfqpoint{2.835cm}{0.253cm}}{\pgfqpoint{2.848cm}{0.234cm}}{\pgfqpoint{2.848cm}{0.205cm}}
\pgfpathcurveto{\pgfqpoint{2.848cm}{0.173cm}}{\pgfqpoint{2.834cm}{0.155cm}}{\pgfqpoint{2.814cm}{0.155cm}}
\pgfpathcurveto{\pgfqpoint{2.794cm}{0.155cm}}{\pgfqpoint{2.781cm}{0.175cm}}{\pgfqpoint{2.781cm}{0.205cm}}
\pgfpathlineto{\pgfqpoint{2.781cm}{0.205cm}}
\pgfpathcurveto{\pgfqpoint{2.781cm}{0.234cm}}{\pgfqpoint{2.794cm}{0.253cm}}{\pgfqpoint{2.815cm}{0.253cm}}
\pgfpathlineto{\pgfqpoint{2.814cm}{0.261cm}}
\pgfpathcurveto{\pgfqpoint{2.781cm}{0.261cm}}{\pgfqpoint{2.756cm}{0.236cm}}{\pgfqpoint{2.756cm}{0.203cm}}
\pgfpathcurveto{\pgfqpoint{2.756cm}{0.177cm}}{\pgfqpoint{2.772cm}{0.158cm}}{\pgfqpoint{2.799cm}{0.15cm}}
\pgfpathlineto{\pgfqpoint{2.799cm}{0.149cm}}
\pgfpathcurveto{\pgfqpoint{2.772cm}{0.14cm}}{\pgfqpoint{2.757cm}{0.128cm}}{\pgfqpoint{2.757cm}{0.112cm}}
\pgfpathcurveto{\pgfqpoint{2.757cm}{0.101cm}}{\pgfqpoint{2.768cm}{0.092cm}}{\pgfqpoint{2.785cm}{0.087cm}}
\pgfpathclose
\pgfusepath{fill}
\definecolor{eps2pgf_color}{rgb}{0.01508,0.02539,0.02301}\pgfsetstrokecolor{eps2pgf_color}\pgfsetfillcolor{eps2pgf_color}
\pgfpathmoveto{\pgfqpoint{4.079cm}{0.272cm}}
\pgfpathlineto{\pgfqpoint{4.084cm}{0.272cm}}
\pgfpathlineto{\pgfqpoint{4.092cm}{0.26cm}}
\pgfpathlineto{\pgfqpoint{4.097cm}{0.26cm}}
\pgfpathlineto{\pgfqpoint{4.088cm}{0.273cm}}
\pgfpathcurveto{\pgfqpoint{4.093cm}{0.273cm}}{\pgfqpoint{4.096cm}{0.275cm}}{\pgfqpoint{4.096cm}{0.281cm}}
\pgfpathcurveto{\pgfqpoint{4.096cm}{0.287cm}}{\pgfqpoint{4.093cm}{0.289cm}}{\pgfqpoint{4.086cm}{0.289cm}}
\pgfpathlineto{\pgfqpoint{4.085cm}{0.285cm}}
\pgfpathcurveto{\pgfqpoint{4.088cm}{0.285cm}}{\pgfqpoint{4.091cm}{0.285cm}}{\pgfqpoint{4.091cm}{0.281cm}}
\pgfpathcurveto{\pgfqpoint{4.091cm}{0.277cm}}{\pgfqpoint{4.088cm}{0.276cm}}{\pgfqpoint{4.084cm}{0.276cm}}
\pgfpathlineto{\pgfqpoint{4.079cm}{0.276cm}}
\pgfpathlineto{\pgfqpoint{4.079cm}{0.285cm}}
\pgfpathlineto{\pgfqpoint{4.085cm}{0.285cm}}
\pgfpathlineto{\pgfqpoint{4.086cm}{0.289cm}}
\pgfpathlineto{\pgfqpoint{4.075cm}{0.289cm}}
\pgfpathlineto{\pgfqpoint{4.075cm}{0.26cm}}
\pgfpathlineto{\pgfqpoint{4.079cm}{0.26cm}}
\pgfpathlineto{\pgfqpoint{4.079cm}{0.272cm}}
\pgfpathclose
\pgfusepath{fill}
\pgfpathmoveto{\pgfqpoint{4.084cm}{0.249cm}}
\pgfpathlineto{\pgfqpoint{4.084cm}{0.253cm}}
\pgfpathcurveto{\pgfqpoint{4.073cm}{0.253cm}}{\pgfqpoint{4.064cm}{0.262cm}}{\pgfqpoint{4.064cm}{0.275cm}}
\pgfpathlineto{\pgfqpoint{4.064cm}{0.275cm}}
\pgfpathcurveto{\pgfqpoint{4.064cm}{0.287cm}}{\pgfqpoint{4.073cm}{0.296cm}}{\pgfqpoint{4.084cm}{0.296cm}}
\pgfpathcurveto{\pgfqpoint{4.096cm}{0.296cm}}{\pgfqpoint{4.105cm}{0.287cm}}{\pgfqpoint{4.105cm}{0.275cm}}
\pgfpathcurveto{\pgfqpoint{4.105cm}{0.262cm}}{\pgfqpoint{4.096cm}{0.253cm}}{\pgfqpoint{4.084cm}{0.253cm}}
\pgfpathlineto{\pgfqpoint{4.084cm}{0.249cm}}
\pgfpathcurveto{\pgfqpoint{4.098cm}{0.249cm}}{\pgfqpoint{4.11cm}{0.26cm}}{\pgfqpoint{4.11cm}{0.275cm}}
\pgfpathcurveto{\pgfqpoint{4.11cm}{0.289cm}}{\pgfqpoint{4.098cm}{0.3cm}}{\pgfqpoint{4.084cm}{0.3cm}}
\pgfpathcurveto{\pgfqpoint{4.071cm}{0.3cm}}{\pgfqpoint{4.059cm}{0.289cm}}{\pgfqpoint{4.059cm}{0.275cm}}
\pgfpathcurveto{\pgfqpoint{4.059cm}{0.26cm}}{\pgfqpoint{4.071cm}{0.249cm}}{\pgfqpoint{4.084cm}{0.249cm}}
\pgfpathclose
\pgfusepath{fill}
\pgfsetdash{}{0cm}
\pgfsetroundcap
\definecolor{eps2pgf_color}{rgb}{0.01405,0.02406,0.02196}\pgfsetstrokecolor{eps2pgf_color}\pgfsetfillcolor{eps2pgf_color}
\pgfpathmoveto{\pgfqpoint{2.737cm}{2.55cm}}
\pgfpathlineto{\pgfqpoint{2.737cm}{2.55cm}}
\pgfusepath{stroke}
\pgfsetdash{{0.106cm}{0.106cm}}{0cm}
\pgfpathmoveto{\pgfqpoint{2.844cm}{2.55cm}}
\pgfpathlineto{\pgfqpoint{6.318cm}{2.55cm}}
\pgfusepath{stroke}
\pgfsetdash{}{0cm}
\pgfpathmoveto{\pgfqpoint{6.371cm}{2.55cm}}
\pgfpathlineto{\pgfqpoint{6.371cm}{2.55cm}}
\pgfusepath{stroke}
\pgfsetdash{}{0cm}
\pgfsetlinewidth{0.705mm}
\pgfsetbuttcap
\pgfpathmoveto{\pgfqpoint{2.367cm}{2.55cm}}
\pgfpathlineto{\pgfqpoint{3.01cm}{2.55cm}}
\pgfusepath{stroke}
\pgfpathmoveto{\pgfqpoint{2.87cm}{2.431cm}}
\pgfpathlineto{\pgfqpoint{2.989cm}{2.55cm}}
\pgfpathlineto{\pgfqpoint{2.87cm}{2.67cm}}
\pgfpathlineto{\pgfqpoint{2.971cm}{2.67cm}}
\pgfpathlineto{\pgfqpoint{3.091cm}{2.55cm}}
\pgfpathlineto{\pgfqpoint{2.971cm}{2.431cm}}
\pgfpathclose
\pgfusepath{fill}
\pgfsetdash{}{0cm}
\pgfsetlinewidth{0.352mm}
\pgfpathmoveto{\pgfqpoint{4.554cm}{3.704cm}}
\pgfpathlineto{\pgfqpoint{4.554cm}{3.432cm}}
\pgfusepath{stroke}
\pgfpathmoveto{\pgfqpoint{4.41cm}{3.549cm}}
\pgfpathlineto{\pgfqpoint{4.436cm}{3.573cm}}
\pgfpathlineto{\pgfqpoint{4.554cm}{3.446cm}}
\pgfpathlineto{\pgfqpoint{4.673cm}{3.573cm}}
\pgfpathlineto{\pgfqpoint{4.698cm}{3.549cm}}
\pgfpathlineto{\pgfqpoint{4.554cm}{3.394cm}}
\pgfpathclose
\pgfusepath{fill}
\pgfsetdash{}{0cm}
\pgfsetlinewidth{0.705mm}
\pgfpathmoveto{\pgfqpoint{6.371cm}{2.582cm}}
\pgfpathlineto{\pgfqpoint{6.939cm}{2.582cm}}
\pgfusepath{stroke}
\pgfpathmoveto{\pgfqpoint{6.799cm}{2.462cm}}
\pgfpathlineto{\pgfqpoint{6.919cm}{2.582cm}}
\pgfpathlineto{\pgfqpoint{6.799cm}{2.701cm}}
\pgfpathlineto{\pgfqpoint{6.9cm}{2.701cm}}
\pgfpathlineto{\pgfqpoint{7.02cm}{2.582cm}}
\pgfpathlineto{\pgfqpoint{6.9cm}{2.462cm}}
\pgfpathlineto{\pgfqpoint{6.976cm}{2.462cm}}
\pgfpathclose
\pgfusepath{fill}
\pgfsetdash{}{0cm}
\pgfpathmoveto{\pgfqpoint{9.328cm}{2.582cm}}
\pgfpathlineto{\pgfqpoint{9.971cm}{2.582cm}}
\pgfusepath{stroke}
\pgfpathmoveto{\pgfqpoint{9.831cm}{2.462cm}}
\pgfpathlineto{\pgfqpoint{9.95cm}{2.582cm}}
\pgfpathlineto{\pgfqpoint{9.831cm}{2.701cm}}
\pgfpathlineto{\pgfqpoint{9.932cm}{2.701cm}}
\pgfpathlineto{\pgfqpoint{10.052cm}{2.582cm}}
\pgfpathlineto{\pgfqpoint{9.932cm}{2.462cm}}
\pgfpathclose
\pgfusepath{fill}
\pgfsetdash{}{0cm}
\pgfpathmoveto{\pgfqpoint{11.84cm}{2.582cm}}
\pgfpathlineto{\pgfqpoint{12.483cm}{2.582cm}}
\pgfusepath{stroke}
\pgfpathmoveto{\pgfqpoint{12.343cm}{2.462cm}}
\pgfpathlineto{\pgfqpoint{12.462cm}{2.582cm}}
\pgfpathlineto{\pgfqpoint{12.343cm}{2.701cm}}
\pgfpathlineto{\pgfqpoint{12.444cm}{2.701cm}}
\pgfpathlineto{\pgfqpoint{12.564cm}{2.582cm}}
\pgfpathlineto{\pgfqpoint{12.444cm}{2.462cm}}
\pgfpathclose
\pgfusepath{fill}
\end{pgfpicture}
}
  \end{center}

\end{frame}

\begin{frame}[t,fragile=singleslide]{\inhibitglue EPSを\TeX{}として使う}
  \sffamily
  \begin{itemize}
      \item 本文と図のフォントが当然のように一致する。画期的
      \item もちろんテキストなのでバージョン管理も簡単
      \item 座標で位置あわせを数値指定できる
      \item 複数の画像ファイルでパーツを一気に差し替える、みたいなことも可能
      \begin{itemize}\item パーツだけをPGF内で\texttt{\bslash{}includegraphics}すればいい\end{itemize}
      \item \TeX{}にした図中では、日本語もまともな組版で使える
      \begin{itemize}\item 元のEPSを日本語で作ってしまうと\texttt{pstoedit}によってアウトライン化されてしまうけど\end{itemize}
  \end{itemize}
\end{frame}

\begin{frame}[t]{\inhibitglue まとめ}
  \sffamily
  \begin{itemize}
    \item このスライドはすべて\texttt{.tex}だけで作られています
    \item \TeX{}はGhostscriptから離れては生きていけない
    \item \hologo{METAPOST}は福音かも
    \item ラムダノート株式会社は出版を中心として技術文書まわりのお手伝いをいろいろする会社です
    \begin{itemize}
      \item \url{https://lambdanote.com}
    \end{itemize}
  \end{itemize}
  \begin{center}
  \resizebox{.5\textwidth}{!}{% Created by Eps2pgf 0.7.0 (build on 2008-08-24) on Thu Nov 08 15:24:42 JST 2018
\begin{pgfpicture}
\pgfpathmoveto{\pgfqpoint{0.351cm}{1.046cm}}
\pgfpathlineto{\pgfqpoint{11.343cm}{1.046cm}}
\pgfpathlineto{\pgfqpoint{11.343cm}{4.353cm}}
\pgfpathlineto{\pgfqpoint{0.351cm}{4.353cm}}
\pgfpathclose
\pgfusepath{clip}
\definecolor{eps2pgf_color}{rgb}{0,0,0}\pgfsetstrokecolor{eps2pgf_color}\pgfsetfillcolor{eps2pgf_color}
\pgftext[x=3.585cm,y=1.517cm,rotate=0]{\fontsize{28.04}{14.45}\selectfont{\texttt{Lambda Note}}}
\pgfsetdash{}{0cm}
\definecolor{eps2pgf_color}{rgb}{0.6,0.6,0.6}\pgfsetstrokecolor{eps2pgf_color}\pgfsetfillcolor{eps2pgf_color}
\pgfpathmoveto{\pgfqpoint{8.299cm}{1.542cm}}
\pgfpathcurveto{\pgfqpoint{8.3cm}{1.45cm}}{\pgfqpoint{8.317cm}{1.346cm}}{\pgfqpoint{8.269cm}{1.276cm}}
\pgfpathcurveto{\pgfqpoint{8.194cm}{1.166cm}}{\pgfqpoint{8.025cm}{1.112cm}}{\pgfqpoint{7.881cm}{1.109cm}}
\pgfpathcurveto{\pgfqpoint{7.646cm}{1.105cm}}{\pgfqpoint{7.453cm}{1.292cm}}{\pgfqpoint{7.449cm}{1.526cm}}
\pgfpathcurveto{\pgfqpoint{7.444cm}{1.761cm}}{\pgfqpoint{7.631cm}{1.955cm}}{\pgfqpoint{7.866cm}{1.959cm}}
\pgfpathcurveto{\pgfqpoint{8.101cm}{1.963cm}}{\pgfqpoint{8.295cm}{1.777cm}}{\pgfqpoint{8.299cm}{1.542cm}}
\pgfpathclose
\pgfusepath{fill}
\definecolor{eps2pgf_color}{rgb}{0,0,0}\pgfsetstrokecolor{eps2pgf_color}\pgfsetfillcolor{eps2pgf_color}
\pgfpathmoveto{\pgfqpoint{11.301cm}{1.675cm}}
\pgfpathcurveto{\pgfqpoint{11.266cm}{1.69cm}}{\pgfqpoint{11.224cm}{1.674cm}}{\pgfqpoint{11.208cm}{1.639cm}}
\pgfpathlineto{\pgfqpoint{11.203cm}{1.626cm}}
\pgfpathcurveto{\pgfqpoint{11.203cm}{1.626cm}}{\pgfqpoint{11.201cm}{1.623cm}}{\pgfqpoint{11.198cm}{1.615cm}}
\pgfpathcurveto{\pgfqpoint{11.194cm}{1.608cm}}{\pgfqpoint{11.192cm}{1.597cm}}{\pgfqpoint{11.183cm}{1.584cm}}
\pgfpathcurveto{\pgfqpoint{11.169cm}{1.556cm}}{\pgfqpoint{11.137cm}{1.517cm}}{\pgfqpoint{11.09cm}{1.474cm}}
\pgfpathcurveto{\pgfqpoint{11.041cm}{1.432cm}}{\pgfqpoint{10.973cm}{1.388cm}}{\pgfqpoint{10.888cm}{1.352cm}}
\pgfpathcurveto{\pgfqpoint{10.803cm}{1.316cm}}{\pgfqpoint{10.7cm}{1.291cm}}{\pgfqpoint{10.587cm}{1.284cm}}
\pgfpathcurveto{\pgfqpoint{10.573cm}{1.283cm}}{\pgfqpoint{10.561cm}{1.284cm}}{\pgfqpoint{10.547cm}{1.283cm}}
\pgfpathcurveto{\pgfqpoint{10.533cm}{1.283cm}}{\pgfqpoint{10.522cm}{1.282cm}}{\pgfqpoint{10.504cm}{1.284cm}}
\pgfpathcurveto{\pgfqpoint{10.488cm}{1.285cm}}{\pgfqpoint{10.471cm}{1.286cm}}{\pgfqpoint{10.454cm}{1.287cm}}
\pgfpathlineto{\pgfqpoint{10.442cm}{1.288cm}}
\pgfpathcurveto{\pgfqpoint{10.451cm}{1.288cm}}{\pgfqpoint{10.443cm}{1.288cm}}{\pgfqpoint{10.445cm}{1.288cm}}
\pgfpathlineto{\pgfqpoint{10.444cm}{1.288cm}}
\pgfpathlineto{\pgfqpoint{10.441cm}{1.288cm}}
\pgfpathlineto{\pgfqpoint{10.435cm}{1.289cm}}
\pgfpathlineto{\pgfqpoint{10.413cm}{1.293cm}}
\pgfpathcurveto{\pgfqpoint{10.398cm}{1.295cm}}{\pgfqpoint{10.383cm}{1.298cm}}{\pgfqpoint{10.367cm}{1.3cm}}
\pgfpathcurveto{\pgfqpoint{10.352cm}{1.303cm}}{\pgfqpoint{10.337cm}{1.307cm}}{\pgfqpoint{10.322cm}{1.311cm}}
\pgfpathcurveto{\pgfqpoint{10.307cm}{1.315cm}}{\pgfqpoint{10.292cm}{1.317cm}}{\pgfqpoint{10.277cm}{1.322cm}}
\pgfpathcurveto{\pgfqpoint{10.262cm}{1.327cm}}{\pgfqpoint{10.247cm}{1.332cm}}{\pgfqpoint{10.232cm}{1.337cm}}
\pgfpathlineto{\pgfqpoint{10.209cm}{1.345cm}}
\pgfpathlineto{\pgfqpoint{10.187cm}{1.355cm}}
\pgfpathcurveto{\pgfqpoint{10.173cm}{1.361cm}}{\pgfqpoint{10.158cm}{1.368cm}}{\pgfqpoint{10.143cm}{1.374cm}}
\pgfpathcurveto{\pgfqpoint{10.115cm}{1.39cm}}{\pgfqpoint{10.084cm}{1.404cm}}{\pgfqpoint{10.057cm}{1.423cm}}
\pgfpathcurveto{\pgfqpoint{10.001cm}{1.458cm}}{\pgfqpoint{9.948cm}{1.502cm}}{\pgfqpoint{9.899cm}{1.55cm}}
\pgfpathcurveto{\pgfqpoint{9.851cm}{1.599cm}}{\pgfqpoint{9.806cm}{1.655cm}}{\pgfqpoint{9.769cm}{1.714cm}}
\pgfpathcurveto{\pgfqpoint{9.765cm}{1.721cm}}{\pgfqpoint{9.761cm}{1.728cm}}{\pgfqpoint{9.757cm}{1.736cm}}
\pgfpathlineto{\pgfqpoint{9.75cm}{1.747cm}}
\pgfpathlineto{\pgfqpoint{9.746cm}{1.755cm}}
\pgfpathcurveto{\pgfqpoint{9.743cm}{1.761cm}}{\pgfqpoint{9.741cm}{1.765cm}}{\pgfqpoint{9.736cm}{1.776cm}}
\pgfpathlineto{\pgfqpoint{9.731cm}{1.787cm}}
\pgfpathcurveto{\pgfqpoint{9.731cm}{1.787cm}}{\pgfqpoint{9.731cm}{1.787cm}}{\pgfqpoint{9.731cm}{1.787cm}}
\pgfpathlineto{\pgfqpoint{9.731cm}{1.789cm}}
\pgfpathlineto{\pgfqpoint{9.73cm}{1.793cm}}
\pgfpathlineto{\pgfqpoint{9.727cm}{1.8cm}}
\pgfpathcurveto{\pgfqpoint{9.714cm}{1.837cm}}{\pgfqpoint{9.701cm}{1.874cm}}{\pgfqpoint{9.688cm}{1.911cm}}
\pgfpathcurveto{\pgfqpoint{9.637cm}{2.062cm}}{\pgfqpoint{9.597cm}{2.221cm}}{\pgfqpoint{9.564cm}{2.384cm}}
\pgfpathcurveto{\pgfqpoint{9.531cm}{2.55cm}}{\pgfqpoint{9.498cm}{2.716cm}}{\pgfqpoint{9.465cm}{2.879cm}}
\pgfpathcurveto{\pgfqpoint{9.432cm}{3.045cm}}{\pgfqpoint{9.392cm}{3.211cm}}{\pgfqpoint{9.335cm}{3.371cm}}
\pgfpathcurveto{\pgfqpoint{9.322cm}{3.411cm}}{\pgfqpoint{9.303cm}{3.453cm}}{\pgfqpoint{9.287cm}{3.493cm}}
\pgfpathcurveto{\pgfqpoint{9.279cm}{3.513cm}}{\pgfqpoint{9.271cm}{3.529cm}}{\pgfqpoint{9.262cm}{3.547cm}}
\pgfpathlineto{\pgfqpoint{9.25cm}{3.573cm}}
\pgfpathlineto{\pgfqpoint{9.244cm}{3.586cm}}
\pgfpathlineto{\pgfqpoint{9.241cm}{3.593cm}}
\pgfpathlineto{\pgfqpoint{9.237cm}{3.6cm}}
\pgfpathlineto{\pgfqpoint{9.235cm}{3.604cm}}
\pgfpathcurveto{\pgfqpoint{9.19cm}{3.687cm}}{\pgfqpoint{9.136cm}{3.753cm}}{\pgfqpoint{9.079cm}{3.813cm}}
\pgfpathcurveto{\pgfqpoint{9.027cm}{3.867cm}}{\pgfqpoint{8.972cm}{3.914cm}}{\pgfqpoint{8.914cm}{3.954cm}}
\pgfpathlineto{\pgfqpoint{8.97cm}{4.144cm}}
\pgfpathcurveto{\pgfqpoint{8.979cm}{4.176cm}}{\pgfqpoint{8.966cm}{4.211cm}}{\pgfqpoint{8.935cm}{4.227cm}}
\pgfpathcurveto{\pgfqpoint{8.901cm}{4.246cm}}{\pgfqpoint{8.857cm}{4.233cm}}{\pgfqpoint{8.838cm}{4.198cm}}
\pgfpathlineto{\pgfqpoint{8.755cm}{4.044cm}}
\pgfpathcurveto{\pgfqpoint{8.735cm}{4.053cm}}{\pgfqpoint{8.715cm}{4.062cm}}{\pgfqpoint{8.695cm}{4.069cm}}
\pgfpathcurveto{\pgfqpoint{8.66cm}{4.081cm}}{\pgfqpoint{8.627cm}{4.093cm}}{\pgfqpoint{8.592cm}{4.1cm}}
\pgfpathcurveto{\pgfqpoint{8.568cm}{4.107cm}}{\pgfqpoint{8.544cm}{4.111cm}}{\pgfqpoint{8.52cm}{4.114cm}}
\pgfpathlineto{\pgfqpoint{8.491cm}{4.293cm}}
\pgfpathcurveto{\pgfqpoint{8.486cm}{4.325cm}}{\pgfqpoint{8.458cm}{4.351cm}}{\pgfqpoint{8.424cm}{4.353cm}}
\pgfpathcurveto{\pgfqpoint{8.384cm}{4.355cm}}{\pgfqpoint{8.351cm}{4.324cm}}{\pgfqpoint{8.349cm}{4.285cm}}
\pgfpathlineto{\pgfqpoint{8.341cm}{4.118cm}}
\pgfpathcurveto{\pgfqpoint{8.324cm}{4.116cm}}{\pgfqpoint{8.308cm}{4.114cm}}{\pgfqpoint{8.293cm}{4.112cm}}
\pgfpathcurveto{\pgfqpoint{8.266cm}{4.107cm}}{\pgfqpoint{8.243cm}{4.101cm}}{\pgfqpoint{8.219cm}{4.096cm}}
\pgfpathlineto{\pgfqpoint{8.204cm}{4.092cm}}
\pgfpathlineto{\pgfqpoint{8.194cm}{4.089cm}}
\pgfpathcurveto{\pgfqpoint{8.187cm}{4.087cm}}{\pgfqpoint{8.181cm}{4.084cm}}{\pgfqpoint{8.174cm}{4.082cm}}
\pgfpathcurveto{\pgfqpoint{8.161cm}{4.078cm}}{\pgfqpoint{8.149cm}{4.073cm}}{\pgfqpoint{8.136cm}{4.068cm}}
\pgfpathcurveto{\pgfqpoint{8.088cm}{4.047cm}}{\pgfqpoint{8.047cm}{4.022cm}}{\pgfqpoint{8.012cm}{3.997cm}}
\pgfpathcurveto{\pgfqpoint{7.943cm}{3.947cm}}{\pgfqpoint{7.901cm}{3.897cm}}{\pgfqpoint{7.874cm}{3.864cm}}
\pgfpathcurveto{\pgfqpoint{7.86cm}{3.847cm}}{\pgfqpoint{7.849cm}{3.833cm}}{\pgfqpoint{7.843cm}{3.824cm}}
\pgfpathcurveto{\pgfqpoint{7.837cm}{3.816cm}}{\pgfqpoint{7.834cm}{3.811cm}}{\pgfqpoint{7.834cm}{3.811cm}}
\pgfpathcurveto{\pgfqpoint{7.814cm}{3.783cm}}{\pgfqpoint{7.817cm}{3.743cm}}{\pgfqpoint{7.843cm}{3.718cm}}
\pgfpathcurveto{\pgfqpoint{7.872cm}{3.691cm}}{\pgfqpoint{7.917cm}{3.692cm}}{\pgfqpoint{7.944cm}{3.721cm}}
\pgfpathcurveto{\pgfqpoint{7.944cm}{3.721cm}}{\pgfqpoint{7.948cm}{3.725cm}}{\pgfqpoint{7.955cm}{3.732cm}}
\pgfpathcurveto{\pgfqpoint{7.962cm}{3.74cm}}{\pgfqpoint{7.971cm}{3.748cm}}{\pgfqpoint{7.984cm}{3.761cm}}
\pgfpathcurveto{\pgfqpoint{8.011cm}{3.785cm}}{\pgfqpoint{8.052cm}{3.819cm}}{\pgfqpoint{8.107cm}{3.851cm}}
\pgfpathcurveto{\pgfqpoint{8.135cm}{3.866cm}}{\pgfqpoint{8.167cm}{3.88cm}}{\pgfqpoint{8.203cm}{3.891cm}}
\pgfpathcurveto{\pgfqpoint{8.211cm}{3.893cm}}{\pgfqpoint{8.22cm}{3.895cm}}{\pgfqpoint{8.23cm}{3.898cm}}
\pgfpathlineto{\pgfqpoint{8.244cm}{3.901cm}}
\pgfpathlineto{\pgfqpoint{8.246cm}{3.901cm}}
\pgfpathcurveto{\pgfqpoint{8.246cm}{3.901cm}}{\pgfqpoint{8.248cm}{3.902cm}}{\pgfqpoint{8.247cm}{3.902cm}}
\pgfpathlineto{\pgfqpoint{8.249cm}{3.902cm}}
\pgfpathlineto{\pgfqpoint{8.254cm}{3.902cm}}
\pgfpathcurveto{\pgfqpoint{8.277cm}{3.905cm}}{\pgfqpoint{8.304cm}{3.909cm}}{\pgfqpoint{8.327cm}{3.911cm}}
\pgfpathcurveto{\pgfqpoint{8.348cm}{3.91cm}}{\pgfqpoint{8.367cm}{3.911cm}}{\pgfqpoint{8.389cm}{3.91cm}}
\pgfpathcurveto{\pgfqpoint{8.401cm}{3.909cm}}{\pgfqpoint{8.412cm}{3.907cm}}{\pgfqpoint{8.424cm}{3.906cm}}
\pgfpathcurveto{\pgfqpoint{8.436cm}{3.905cm}}{\pgfqpoint{8.448cm}{3.903cm}}{\pgfqpoint{8.46cm}{3.9cm}}
\pgfpathcurveto{\pgfqpoint{8.483cm}{3.895cm}}{\pgfqpoint{8.508cm}{3.891cm}}{\pgfqpoint{8.532cm}{3.881cm}}
\pgfpathcurveto{\pgfqpoint{8.557cm}{3.875cm}}{\pgfqpoint{8.581cm}{3.864cm}}{\pgfqpoint{8.606cm}{3.853cm}}
\pgfpathcurveto{\pgfqpoint{8.655cm}{3.829cm}}{\pgfqpoint{8.705cm}{3.801cm}}{\pgfqpoint{8.751cm}{3.763cm}}
\pgfpathcurveto{\pgfqpoint{8.799cm}{3.727cm}}{\pgfqpoint{8.846cm}{3.685cm}}{\pgfqpoint{8.887cm}{3.637cm}}
\pgfpathcurveto{\pgfqpoint{8.928cm}{3.589cm}}{\pgfqpoint{8.966cm}{3.534cm}}{\pgfqpoint{8.99cm}{3.481cm}}
\pgfpathcurveto{\pgfqpoint{8.997cm}{3.465cm}}{\pgfqpoint{9.004cm}{3.449cm}}{\pgfqpoint{9.011cm}{3.433cm}}
\pgfpathcurveto{\pgfqpoint{9.019cm}{3.416cm}}{\pgfqpoint{9.028cm}{3.397cm}}{\pgfqpoint{9.034cm}{3.38cm}}
\pgfpathcurveto{\pgfqpoint{9.045cm}{3.347cm}}{\pgfqpoint{9.058cm}{3.317cm}}{\pgfqpoint{9.068cm}{3.281cm}}
\pgfpathcurveto{\pgfqpoint{9.113cm}{3.139cm}}{\pgfqpoint{9.145cm}{2.983cm}}{\pgfqpoint{9.177cm}{2.822cm}}
\pgfpathcurveto{\pgfqpoint{9.178cm}{2.816cm}}{\pgfqpoint{9.179cm}{2.81cm}}{\pgfqpoint{9.181cm}{2.804cm}}
\pgfpathcurveto{\pgfqpoint{9.179cm}{2.799cm}}{\pgfqpoint{9.178cm}{2.795cm}}{\pgfqpoint{9.177cm}{2.79cm}}
\pgfpathcurveto{\pgfqpoint{9.172cm}{2.77cm}}{\pgfqpoint{9.164cm}{2.744cm}}{\pgfqpoint{9.156cm}{2.716cm}}
\pgfpathcurveto{\pgfqpoint{9.147cm}{2.687cm}}{\pgfqpoint{9.137cm}{2.656cm}}{\pgfqpoint{9.125cm}{2.621cm}}
\pgfpathcurveto{\pgfqpoint{9.113cm}{2.587cm}}{\pgfqpoint{9.101cm}{2.55cm}}{\pgfqpoint{9.086cm}{2.51cm}}
\pgfpathcurveto{\pgfqpoint{9.057cm}{2.43cm}}{\pgfqpoint{9.022cm}{2.34cm}}{\pgfqpoint{8.983cm}{2.242cm}}
\pgfpathcurveto{\pgfqpoint{8.945cm}{2.144cm}}{\pgfqpoint{8.905cm}{2.037cm}}{\pgfqpoint{8.857cm}{1.926cm}}
\pgfpathcurveto{\pgfqpoint{8.844cm}{1.899cm}}{\pgfqpoint{8.832cm}{1.871cm}}{\pgfqpoint{8.819cm}{1.843cm}}
\pgfpathcurveto{\pgfqpoint{8.805cm}{1.816cm}}{\pgfqpoint{8.791cm}{1.788cm}}{\pgfqpoint{8.776cm}{1.761cm}}
\pgfpathcurveto{\pgfqpoint{8.768cm}{1.748cm}}{\pgfqpoint{8.76cm}{1.735cm}}{\pgfqpoint{8.753cm}{1.721cm}}
\pgfpathcurveto{\pgfqpoint{8.744cm}{1.709cm}}{\pgfqpoint{8.736cm}{1.696cm}}{\pgfqpoint{8.728cm}{1.684cm}}
\pgfpathcurveto{\pgfqpoint{8.711cm}{1.66cm}}{\pgfqpoint{8.688cm}{1.63cm}}{\pgfqpoint{8.669cm}{1.607cm}}
\pgfpathcurveto{\pgfqpoint{8.589cm}{1.505cm}}{\pgfqpoint{8.497cm}{1.408cm}}{\pgfqpoint{8.389cm}{1.333cm}}
\pgfpathcurveto{\pgfqpoint{8.335cm}{1.296cm}}{\pgfqpoint{8.277cm}{1.264cm}}{\pgfqpoint{8.216cm}{1.239cm}}
\pgfpathcurveto{\pgfqpoint{8.185cm}{1.226cm}}{\pgfqpoint{8.154cm}{1.215cm}}{\pgfqpoint{8.121cm}{1.206cm}}
\pgfpathcurveto{\pgfqpoint{8.105cm}{1.2cm}}{\pgfqpoint{8.089cm}{1.196cm}}{\pgfqpoint{8.073cm}{1.192cm}}
\pgfpathlineto{\pgfqpoint{8.06cm}{1.189cm}}
\pgfpathlineto{\pgfqpoint{8.054cm}{1.187cm}}
\pgfpathlineto{\pgfqpoint{8.053cm}{1.187cm}}
\pgfpathlineto{\pgfqpoint{8.05cm}{1.186cm}}
\pgfpathcurveto{\pgfqpoint{8.04cm}{1.184cm}}{\pgfqpoint{8.035cm}{1.184cm}}{\pgfqpoint{8.029cm}{1.183cm}}
\pgfpathcurveto{\pgfqpoint{7.903cm}{1.171cm}}{\pgfqpoint{7.765cm}{1.196cm}}{\pgfqpoint{7.664cm}{1.264cm}}
\pgfpathlineto{\pgfqpoint{7.645cm}{1.277cm}}
\pgfpathcurveto{\pgfqpoint{7.639cm}{1.281cm}}{\pgfqpoint{7.633cm}{1.286cm}}{\pgfqpoint{7.627cm}{1.291cm}}
\pgfpathlineto{\pgfqpoint{7.61cm}{1.306cm}}
\pgfpathlineto{\pgfqpoint{7.594cm}{1.322cm}}
\pgfpathlineto{\pgfqpoint{7.579cm}{1.337cm}}
\pgfpathlineto{\pgfqpoint{7.563cm}{1.357cm}}
\pgfpathcurveto{\pgfqpoint{7.555cm}{1.368cm}}{\pgfqpoint{7.547cm}{1.378cm}}{\pgfqpoint{7.54cm}{1.392cm}}
\pgfpathcurveto{\pgfqpoint{7.509cm}{1.443cm}}{\pgfqpoint{7.49cm}{1.501cm}}{\pgfqpoint{7.484cm}{1.557cm}}
\pgfpathcurveto{\pgfqpoint{7.478cm}{1.614cm}}{\pgfqpoint{7.486cm}{1.668cm}}{\pgfqpoint{7.509cm}{1.714cm}}
\pgfpathlineto{\pgfqpoint{7.518cm}{1.731cm}}
\pgfpathlineto{\pgfqpoint{7.518cm}{1.731cm}}
\pgfpathlineto{\pgfqpoint{7.518cm}{1.731cm}}
\pgfpathlineto{\pgfqpoint{7.522cm}{1.736cm}}
\pgfpathlineto{\pgfqpoint{7.529cm}{1.746cm}}
\pgfpathcurveto{\pgfqpoint{7.533cm}{1.752cm}}{\pgfqpoint{7.538cm}{1.76cm}}{\pgfqpoint{7.542cm}{1.765cm}}
\pgfpathlineto{\pgfqpoint{7.554cm}{1.777cm}}
\pgfpathcurveto{\pgfqpoint{7.56cm}{1.785cm}}{\pgfqpoint{7.571cm}{1.793cm}}{\pgfqpoint{7.581cm}{1.801cm}}
\pgfpathcurveto{\pgfqpoint{7.59cm}{1.81cm}}{\pgfqpoint{7.601cm}{1.816cm}}{\pgfqpoint{7.612cm}{1.823cm}}
\pgfpathcurveto{\pgfqpoint{7.699cm}{1.877cm}}{\pgfqpoint{7.809cm}{1.888cm}}{\pgfqpoint{7.9cm}{1.871cm}}
\pgfpathcurveto{\pgfqpoint{7.946cm}{1.862cm}}{\pgfqpoint{7.989cm}{1.848cm}}{\pgfqpoint{8.025cm}{1.828cm}}
\pgfpathcurveto{\pgfqpoint{8.035cm}{1.824cm}}{\pgfqpoint{8.043cm}{1.817cm}}{\pgfqpoint{8.051cm}{1.812cm}}
\pgfpathlineto{\pgfqpoint{8.058cm}{1.808cm}}
\pgfpathlineto{\pgfqpoint{8.061cm}{1.806cm}}
\pgfpathlineto{\pgfqpoint{8.062cm}{1.806cm}}
\pgfpathlineto{\pgfqpoint{8.075cm}{1.795cm}}
\pgfpathcurveto{\pgfqpoint{8.083cm}{1.788cm}}{\pgfqpoint{8.094cm}{1.781cm}}{\pgfqpoint{8.1cm}{1.775cm}}
\pgfpathcurveto{\pgfqpoint{8.106cm}{1.769cm}}{\pgfqpoint{8.112cm}{1.763cm}}{\pgfqpoint{8.117cm}{1.757cm}}
\pgfpathcurveto{\pgfqpoint{8.166cm}{1.704cm}}{\pgfqpoint{8.194cm}{1.642cm}}{\pgfqpoint{8.208cm}{1.589cm}}
\pgfpathcurveto{\pgfqpoint{8.223cm}{1.535cm}}{\pgfqpoint{8.227cm}{1.491cm}}{\pgfqpoint{8.228cm}{1.461cm}}
\pgfpathcurveto{\pgfqpoint{8.229cm}{1.447cm}}{\pgfqpoint{8.229cm}{1.435cm}}{\pgfqpoint{8.228cm}{1.428cm}}
\pgfpathcurveto{\pgfqpoint{8.228cm}{1.421cm}}{\pgfqpoint{8.228cm}{1.418cm}}{\pgfqpoint{8.228cm}{1.418cm}}
\pgfpathlineto{\pgfqpoint{8.228cm}{1.414cm}}
\pgfpathcurveto{\pgfqpoint{8.228cm}{1.413cm}}{\pgfqpoint{8.228cm}{1.413cm}}{\pgfqpoint{8.228cm}{1.413cm}}
\pgfpathcurveto{\pgfqpoint{8.227cm}{1.374cm}}{\pgfqpoint{8.258cm}{1.341cm}}{\pgfqpoint{8.297cm}{1.34cm}}
\pgfpathcurveto{\pgfqpoint{8.336cm}{1.339cm}}{\pgfqpoint{8.369cm}{1.37cm}}{\pgfqpoint{8.37cm}{1.409cm}}
\pgfpathcurveto{\pgfqpoint{8.37cm}{1.409cm}}{\pgfqpoint{8.37cm}{1.415cm}}{\pgfqpoint{8.37cm}{1.426cm}}
\pgfpathcurveto{\pgfqpoint{8.371cm}{1.436cm}}{\pgfqpoint{8.37cm}{1.45cm}}{\pgfqpoint{8.369cm}{1.469cm}}
\pgfpathcurveto{\pgfqpoint{8.367cm}{1.506cm}}{\pgfqpoint{8.361cm}{1.56cm}}{\pgfqpoint{8.341cm}{1.627cm}}
\pgfpathcurveto{\pgfqpoint{8.322cm}{1.694cm}}{\pgfqpoint{8.286cm}{1.775cm}}{\pgfqpoint{8.216cm}{1.851cm}}
\pgfpathcurveto{\pgfqpoint{8.206cm}{1.86cm}}{\pgfqpoint{8.196cm}{1.87cm}}{\pgfqpoint{8.186cm}{1.88cm}}
\pgfpathcurveto{\pgfqpoint{8.176cm}{1.89cm}}{\pgfqpoint{8.168cm}{1.894cm}}{\pgfqpoint{8.158cm}{1.902cm}}
\pgfpathlineto{\pgfqpoint{8.144cm}{1.913cm}}
\pgfpathlineto{\pgfqpoint{8.14cm}{1.916cm}}
\pgfpathlineto{\pgfqpoint{8.137cm}{1.918cm}}
\pgfpathlineto{\pgfqpoint{8.133cm}{1.921cm}}
\pgfpathlineto{\pgfqpoint{8.124cm}{1.926cm}}
\pgfpathcurveto{\pgfqpoint{8.112cm}{1.933cm}}{\pgfqpoint{8.101cm}{1.941cm}}{\pgfqpoint{8.088cm}{1.947cm}}
\pgfpathcurveto{\pgfqpoint{8.039cm}{1.974cm}}{\pgfqpoint{7.982cm}{1.992cm}}{\pgfqpoint{7.923cm}{2.002cm}}
\pgfpathcurveto{\pgfqpoint{7.805cm}{2.021cm}}{\pgfqpoint{7.667cm}{2.011cm}}{\pgfqpoint{7.543cm}{1.935cm}}
\pgfpathcurveto{\pgfqpoint{7.528cm}{1.925cm}}{\pgfqpoint{7.512cm}{1.916cm}}{\pgfqpoint{7.498cm}{1.904cm}}
\pgfpathcurveto{\pgfqpoint{7.484cm}{1.892cm}}{\pgfqpoint{7.47cm}{1.881cm}}{\pgfqpoint{7.456cm}{1.864cm}}
\pgfpathlineto{\pgfqpoint{7.435cm}{1.841cm}}
\pgfpathcurveto{\pgfqpoint{7.43cm}{1.834cm}}{\pgfqpoint{7.426cm}{1.828cm}}{\pgfqpoint{7.421cm}{1.822cm}}
\pgfpathlineto{\pgfqpoint{7.414cm}{1.812cm}}
\pgfpathlineto{\pgfqpoint{7.411cm}{1.807cm}}
\pgfpathlineto{\pgfqpoint{7.409cm}{1.804cm}}
\pgfpathcurveto{\pgfqpoint{7.409cm}{1.805cm}}{\pgfqpoint{7.404cm}{1.796cm}}{\pgfqpoint{7.406cm}{1.799cm}}
\pgfpathlineto{\pgfqpoint{7.405cm}{1.798cm}}
\pgfpathlineto{\pgfqpoint{7.391cm}{1.772cm}}
\pgfpathcurveto{\pgfqpoint{7.356cm}{1.701cm}}{\pgfqpoint{7.346cm}{1.62cm}}{\pgfqpoint{7.353cm}{1.544cm}}
\pgfpathcurveto{\pgfqpoint{7.362cm}{1.467cm}}{\pgfqpoint{7.387cm}{1.392cm}}{\pgfqpoint{7.427cm}{1.325cm}}
\pgfpathcurveto{\pgfqpoint{7.436cm}{1.308cm}}{\pgfqpoint{7.449cm}{1.29cm}}{\pgfqpoint{7.462cm}{1.274cm}}
\pgfpathlineto{\pgfqpoint{7.478cm}{1.253cm}}
\pgfpathlineto{\pgfqpoint{7.482cm}{1.249cm}}
\pgfpathlineto{\pgfqpoint{7.485cm}{1.246cm}}
\pgfpathlineto{\pgfqpoint{7.49cm}{1.241cm}}
\pgfpathlineto{\pgfqpoint{7.5cm}{1.231cm}}
\pgfpathlineto{\pgfqpoint{7.521cm}{1.21cm}}
\pgfpathlineto{\pgfqpoint{7.544cm}{1.19cm}}
\pgfpathcurveto{\pgfqpoint{7.559cm}{1.177cm}}{\pgfqpoint{7.575cm}{1.166cm}}{\pgfqpoint{7.592cm}{1.155cm}}
\pgfpathcurveto{\pgfqpoint{7.658cm}{1.111cm}}{\pgfqpoint{7.733cm}{1.081cm}}{\pgfqpoint{7.809cm}{1.064cm}}
\pgfpathcurveto{\pgfqpoint{7.886cm}{1.048cm}}{\pgfqpoint{7.963cm}{1.042cm}}{\pgfqpoint{8.044cm}{1.05cm}}
\pgfpathcurveto{\pgfqpoint{8.055cm}{1.051cm}}{\pgfqpoint{8.068cm}{1.053cm}}{\pgfqpoint{8.075cm}{1.054cm}}
\pgfpathlineto{\pgfqpoint{8.079cm}{1.055cm}}
\pgfpathlineto{\pgfqpoint{8.084cm}{1.056cm}}
\pgfpathlineto{\pgfqpoint{8.091cm}{1.058cm}}
\pgfpathlineto{\pgfqpoint{8.105cm}{1.061cm}}
\pgfpathcurveto{\pgfqpoint{8.123cm}{1.065cm}}{\pgfqpoint{8.142cm}{1.07cm}}{\pgfqpoint{8.16cm}{1.076cm}}
\pgfpathcurveto{\pgfqpoint{8.196cm}{1.086cm}}{\pgfqpoint{8.232cm}{1.098cm}}{\pgfqpoint{8.267cm}{1.112cm}}
\pgfpathcurveto{\pgfqpoint{8.338cm}{1.141cm}}{\pgfqpoint{8.405cm}{1.177cm}}{\pgfqpoint{8.467cm}{1.219cm}}
\pgfpathcurveto{\pgfqpoint{8.592cm}{1.305cm}}{\pgfqpoint{8.693cm}{1.411cm}}{\pgfqpoint{8.78cm}{1.519cm}}
\pgfpathcurveto{\pgfqpoint{8.803cm}{1.547cm}}{\pgfqpoint{8.821cm}{1.571cm}}{\pgfqpoint{8.843cm}{1.6cm}}
\pgfpathcurveto{\pgfqpoint{8.853cm}{1.614cm}}{\pgfqpoint{8.864cm}{1.629cm}}{\pgfqpoint{8.874cm}{1.643cm}}
\pgfpathcurveto{\pgfqpoint{8.883cm}{1.658cm}}{\pgfqpoint{8.892cm}{1.672cm}}{\pgfqpoint{8.901cm}{1.686cm}}
\pgfpathcurveto{\pgfqpoint{8.937cm}{1.743cm}}{\pgfqpoint{8.969cm}{1.799cm}}{\pgfqpoint{9.001cm}{1.854cm}}
\pgfpathcurveto{\pgfqpoint{9.063cm}{1.963cm}}{\pgfqpoint{9.117cm}{2.068cm}}{\pgfqpoint{9.161cm}{2.169cm}}
\pgfpathcurveto{\pgfqpoint{9.198cm}{2.256cm}}{\pgfqpoint{9.229cm}{2.338cm}}{\pgfqpoint{9.254cm}{2.413cm}}
\pgfpathcurveto{\pgfqpoint{9.26cm}{2.383cm}}{\pgfqpoint{9.266cm}{2.353cm}}{\pgfqpoint{9.273cm}{2.322cm}}
\pgfpathcurveto{\pgfqpoint{9.31cm}{2.153cm}}{\pgfqpoint{9.354cm}{1.982cm}}{\pgfqpoint{9.412cm}{1.817cm}}
\pgfpathcurveto{\pgfqpoint{9.426cm}{1.775cm}}{\pgfqpoint{9.443cm}{1.735cm}}{\pgfqpoint{9.458cm}{1.694cm}}
\pgfpathlineto{\pgfqpoint{9.466cm}{1.674cm}}
\pgfpathlineto{\pgfqpoint{9.473cm}{1.66cm}}
\pgfpathcurveto{\pgfqpoint{9.476cm}{1.652cm}}{\pgfqpoint{9.484cm}{1.638cm}}{\pgfqpoint{9.49cm}{1.626cm}}
\pgfpathlineto{\pgfqpoint{9.5cm}{1.607cm}}
\pgfpathlineto{\pgfqpoint{9.51cm}{1.591cm}}
\pgfpathcurveto{\pgfqpoint{9.516cm}{1.581cm}}{\pgfqpoint{9.522cm}{1.571cm}}{\pgfqpoint{9.529cm}{1.561cm}}
\pgfpathcurveto{\pgfqpoint{9.581cm}{1.482cm}}{\pgfqpoint{9.641cm}{1.412cm}}{\pgfqpoint{9.707cm}{1.351cm}}
\pgfpathcurveto{\pgfqpoint{9.739cm}{1.32cm}}{\pgfqpoint{9.774cm}{1.292cm}}{\pgfqpoint{9.809cm}{1.265cm}}
\pgfpathcurveto{\pgfqpoint{9.845cm}{1.239cm}}{\pgfqpoint{9.881cm}{1.214cm}}{\pgfqpoint{9.92cm}{1.194cm}}
\pgfpathcurveto{\pgfqpoint{9.957cm}{1.171cm}}{\pgfqpoint{9.997cm}{1.154cm}}{\pgfqpoint{10.035cm}{1.136cm}}
\pgfpathcurveto{\pgfqpoint{10.054cm}{1.128cm}}{\pgfqpoint{10.074cm}{1.121cm}}{\pgfqpoint{10.094cm}{1.113cm}}
\pgfpathlineto{\pgfqpoint{10.123cm}{1.102cm}}
\pgfpathlineto{\pgfqpoint{10.153cm}{1.094cm}}
\pgfpathcurveto{\pgfqpoint{10.173cm}{1.088cm}}{\pgfqpoint{10.192cm}{1.083cm}}{\pgfqpoint{10.212cm}{1.077cm}}
\pgfpathcurveto{\pgfqpoint{10.232cm}{1.072cm}}{\pgfqpoint{10.251cm}{1.069cm}}{\pgfqpoint{10.271cm}{1.066cm}}
\pgfpathcurveto{\pgfqpoint{10.29cm}{1.062cm}}{\pgfqpoint{10.31cm}{1.058cm}}{\pgfqpoint{10.329cm}{1.056cm}}
\pgfpathcurveto{\pgfqpoint{10.349cm}{1.054cm}}{\pgfqpoint{10.368cm}{1.053cm}}{\pgfqpoint{10.387cm}{1.051cm}}
\pgfpathlineto{\pgfqpoint{10.415cm}{1.049cm}}
\pgfpathlineto{\pgfqpoint{10.434cm}{1.047cm}}
\pgfpathlineto{\pgfqpoint{10.447cm}{1.047cm}}
\pgfpathcurveto{\pgfqpoint{10.464cm}{1.047cm}}{\pgfqpoint{10.481cm}{1.047cm}}{\pgfqpoint{10.497cm}{1.048cm}}
\pgfpathcurveto{\pgfqpoint{10.512cm}{1.047cm}}{\pgfqpoint{10.534cm}{1.049cm}}{\pgfqpoint{10.552cm}{1.05cm}}
\pgfpathcurveto{\pgfqpoint{10.571cm}{1.052cm}}{\pgfqpoint{10.591cm}{1.053cm}}{\pgfqpoint{10.609cm}{1.055cm}}
\pgfpathcurveto{\pgfqpoint{10.75cm}{1.074cm}}{\pgfqpoint{10.873cm}{1.118cm}}{\pgfqpoint{10.975cm}{1.171cm}}
\pgfpathcurveto{\pgfqpoint{11.076cm}{1.225cm}}{\pgfqpoint{11.154cm}{1.29cm}}{\pgfqpoint{11.211cm}{1.353cm}}
\pgfpathcurveto{\pgfqpoint{11.268cm}{1.415cm}}{\pgfqpoint{11.303cm}{1.475cm}}{\pgfqpoint{11.319cm}{1.521cm}}
\pgfpathcurveto{\pgfqpoint{11.329cm}{1.542cm}}{\pgfqpoint{11.333cm}{1.562cm}}{\pgfqpoint{11.336cm}{1.574cm}}
\pgfpathcurveto{\pgfqpoint{11.339cm}{1.586cm}}{\pgfqpoint{11.341cm}{1.593cm}}{\pgfqpoint{11.341cm}{1.593cm}}
\pgfpathcurveto{\pgfqpoint{11.35cm}{1.625cm}}{\pgfqpoint{11.333cm}{1.66cm}}{\pgfqpoint{11.301cm}{1.675cm}}
\pgfpathclose
\pgfusepath{fill}
\end{pgfpicture}
}
  \end{center}
\end{frame}

\begin{frame}[t]{\inhibitglue 参考資料}
  \sffamily
  \fontsize{9pt}{9pt}\selectfont

  \begin{itemize}
    \item ``The epsf package'', \url{http://tug.ctan.org/macros/generic/epsf/epsf-doc.pdf} \\
    Plain \TeX{}でEPSを取り込むのに使われるepsfパッケージのマニュアル。クヌースの関与もわかる。
    \item ``The Dvipdfmx User’s Manual'' \url{http://www.tug.org/texlive//devsrc/Master/texmf-dist/doc/dvipdfmx/dvipdfmx.pdf}\\
     dvipdfmxのマニュアル。dvipdfmxにおける画像の扱いの考え方がわかる。
    \item ``The pdfTEX user manual'', \url{http://texdoc.net/texmf-dist/doc/pdftex/manual/pdftex-a.pdf}\\
     pdf\TeX{}のマニュアル。pdf\TeX{}における画像の扱いの考え方がわかる。Heiko Oberdiek氏によるepstopdfパッケージのドキュメントもよい資料(``The epstopdf package'', \url{http://mirrors.ctan.org/macros/latex/contrib/oberdiek/epstopdf.pdf} )。
    \item ``METAPOST, a user's manual'', \url{https://www.tug.org/docs/metapost/mpman.pdf} \\
    \hologo{METAPOST}のマニュアル
  \end{itemize}

\end{frame}

\begin{frame}[t]{\inhibitglue 参考資料(つづき)}
  \sffamily
  \fontsize{9pt}{9pt}\selectfont

  \begin{itemize}
    \item ``Ghostscript and the PostScript Language'', \url{https://www.ghostscript.com/doc/9.20/Language.htm}\\
     PostScriptの\texttt{bind}をGhostscriptでは\texttt{.bind}として再定義していました、ということが書いてある。
    \item ``PS interpreter - remove superexec from systemdict'', \url{http://git.ghostscript.com/?p=ghostpdl.git;a=commitdiff;h=8556b698892e4706aa0b9d996bec82fed645eaa5}\\
    \texttt{DELAYBIND}は、PostScriptの\texttt{bind}コマンドの動作をちょっと変えることで、標準ライブラリのコマンド名を上書きしているようなpsファイルを扱えるようにするためのGhostscript独自の仕掛け。Adobe Distillerが隠し持っている\texttt{internaldict}辞書を操作する\texttt{superexec}というコマンドが\texttt{systemdict}にあったのを取り除いたときに、副作用があるので除去された。そのときのコミット。
    \item ``PostScript Language Reference Manual''\\
    第2版がアドビシステムズジャパン監訳で翻訳されている。幸い、DSCとEPSについては第3版より第2版のほうがわかりやすい。
    \item ``Eps2pgf'' \url{https://sourceforge.net/projects/eps2pgf/}\\
    SourceForgeのEps2pgfの配布サイト。
    \item ``User's Guide to the PGF Package, Version 0.61'' \url{https://www.tuteurs.ens.fr/noncvs/docs/pgf/pgfuserguide.pdf}\\
    2004年ころのPGF(非TikZ)のマニュアル(全25ページ!)。
  \end{itemize}

\end{frame}

%\bibliographystyle{alpha}
%\bibliography{void}

\end{document}
