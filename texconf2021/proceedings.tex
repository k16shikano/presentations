\documentclass[uplatex,12pt,dvipdfmx]{jsarticle}

\usepackage{type1cm}
\usepackage[T1]{fontenc}
\usepackage[deluxe, uplatex]{otf}
\usepackage{textcomp}
\usepackage[scaled]{helvet}
\usepackage{lucidabr}
\usepackage{graphicx}
\usepackage{wrapfig}

\pagestyle{empty}

\begin{document}
\vspace*{-3\baselineskip}

\hbox to\textwidth{\small\hfil 2021年11月13日}\par
\hbox to\textwidth{\small\hfil \TeX{}Conf 2021}

\vspace{\baselineskip}

\begin{center}\large
表とリスト
\end{center}

\begin{center}
\small
鹿野 桂一郎\\[0pt]
\footnotesize
k16.shikano@lambdanote.com(@golden\_lucky)
\end{center}

\vspace{\baselineskip}

平面上に文字を並べて情報を表現する際には、その情報がもつ構造に「何らかの視覚的な効果」を適用することになる。「何らかの視覚的な効果」としては、以下のようなものが(ときには組み合わせて)利用される。

\begin{itemize}
\item インデント
\item 行頭の記号や連番
\item 格子
\end{itemize}

これら視覚的な効果は、一般にはそれぞれに対応する自然な構造を表すのに用いられる。
ただし、この関係は常に一対一ではないことに注意。
たとえば、以下の2種類の視覚的な効果はいずれも同一の構造を表すのに利用できる場合があるだろう。

\begin{itemize}
\item \textsf{単純な表}: 複数の文字列を格子上に配置したときの縦軸と横軸とで決まる単純な関係から情報の意味を表現するもの
\item \textsf{二重入れ子の箇条書き}:同一の行頭記号によって並列関係にある複数の文字列を表現する「箇条書き」の各項目に対し、さらに別の箇条書きを入れ子にするもの
\end{itemize}

ところで、コンピューター上で文書を編集する手段としては「テキストエディタ」が広く使われている。
テキストエディタでは、人間が文字列を行単位で入力していくことが前提とされている。
つまり、平面の任意の位置に文字を入力できるようにはなっておらず、「行の先頭から文字を順番に入力する」という、いわば行指向のインターフェイスになっている。

行指向のインターフェイスは、テキストエディタの欠点ではなく、むしろ利点である。
しかし、冒頭に挙げた視覚効果のうち「格子」、つまり横方向と縦方向の2つの軸で決まる「文字列の平面上での位置」によって意味をもたせた視覚効果を意図する情報の入力には不向きである。
これは格子という視覚効果が持つ性質に由来する傾向だといえる。
したがって、格子の代表である「表」は、その複雑さによらず、そのままテキストエディタで入力および編集をするには困難が伴う。

ここで、上記で例として挙げた「単純な表」と「二重入れ子の箇条書き」によって同一の構造を表せるという性質に着目しよう。
「二重入れ子の箇条書き」は、行頭の記号とインデントの深さという、行指向の入力において不自由がない視覚表現である。そのため、同じ構造を表すうえでは、「単純な表」よりもテキストエディタでの入力や編集に向いているといえる。
そこで、「二重入れ子の箇条書き」を入力方法とし、最終的な視覚効果として「単純な表」を得るというアプローチが考えられる。
このようなアプローチは「\textsf{リストテーブル}」などと呼ばれている。

リストテーブルは、原稿をテキストエディタで編集しつつ出力でリッチな視覚効果を実現することを目指したドキュメントシステムまたはフォーマットにおいて実装されていることがある。
たとえばSphinx(reST)にはそのような構文が実装されている。
なお、reSTの入力に対応している他の実装としてPandocがあり、こちらも現在はリストテーブルに対応しているが、Pandoc構造としては\texttt{Table}しかないので、これはあくまでもreST記法のReaderで\texttt{Table}として読んでいるだけである。

では、\LaTeX においてリストテーブルを利用することは現実的だろうか?
残念ながら\LaTeX の記法ではインデントに意味をもたせることが難しいこともあり、これまでそのようなアプローチは存在しなかったとおもわれる。
\texttt{itemize}環境を入れ子にするのも、\texttt{tabular}環境を書くのも、テキストエディタ上での編集の手間や視覚的なわかりやすさに大差がない、という側面もあるだろう。

とはいえ、reSTにおけるリストテーブルのような機能を\LaTeX で実装できないわけではない。
本発表ではナイーブな実装も紹介する。

\end{document}
