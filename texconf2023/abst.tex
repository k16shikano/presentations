\documentclass[a4paper,11pt]{ltjsarticle}

\usepackage{luacode,luatexbase}
\usepackage{luatexja}
\usepackage{luatexja-otf}
\usepackage[match]{luatexja-fontspec}
\usepackage[no-math,deluxe,expert,haranoaji]{luatexja-preset}

\ltjnewpreset{ln}{%
mc=RoHMinKokPro-Lt.otf,
mc-bx=A-OTF-ReimYfozPro-Bold,
gt-m=KoburinaGoStdN-W3.otf,
gt-b=A-OTF-FutoGoB101Pro-Bold,
gt-bx=A-OTF-FutoGoB101Pro-Bold,
gt-eb=A-OTF-GothicMB101Pro-Heavy,
mg-m=A-OTF-Jun101Pro-Light,
__custom = false, __office = false, __noembed = false}
\ltjapplypreset{ln}

\def\n@baseline{16.25}%

\protected\def\pdfliteral{\pdfextension literal}

\usepackage[skins, breakable]{tcolorbox}

%\usepackage{nodetree}
%\NodetreeRegisterCallback{pre_output_filter}

\makeatletter
\newdimen\my@tcb@ftn@height
\my@tcb@ftn@height\z@
\newcount\yafootnotecount
\newdimen\footnotewidth
\footnotewidth=\textwidth
\newdimen\paryafootnoteskip
\paryafootnoteskip=.1\baselineskip

\def\yafootnote#1{\nobreak%
  \global\advance\yafootnotecount 1
  \global\expandafter\newbox
    \csname yafoot_\the\yafootnotecount\endcsname
  \begingroup
    \attribute100=\expandafter\the\csname yafoot_\the\yafootnotecount\endcsname
    \expandafter\yafootnotemark\expandafter{\the\yafootnotecount}
    \vadjust {\pdfliteral{}}%
  \endgroup
  \global\expandafter\setbox
    \csname yafoot_\the\yafootnotecount\endcsname
    \vtop{\yafootnotetext{#1}}%
}

\footnotesep13pt
\def\@thefnmark{\the\yafootnotecount}
\def\yafootnotetext{%
    \normalfont\footnotesize
    \interlinepenalty\interfootnotelinepenalty
    \hsize\footnotewidth \@parboxrestore
    \protected@edef\@currentlabel{%
       \csname p@footnote\endcsname\@thefnmark
    }%
    \@makefntext{%
      \rule\z@\footnotesep\ignorespaces\leavevmode\inhibitglue}}
\renewcommand\@makefntext[1]{%
  \advance\leftskip 3\zw
  \parindent 1\zw
  \noindent
  \llap{\yafootnotemark{\the\yafootnotecount}\hskip0.3\zw}#1}
\def\yafootnotemark#1{\scriptsize\ensuremath{^{\mbox{\thefootnote #1}}}\nobreak}
\def\thefootnote{\leavevmode\raise.2ex\hbox{\scriptsize$\dagger$}}

\directlua{dofile("./yafoot.lua")}

\directlua{%
    luatexbase.add_to_callback
      ("post_linebreak_filter",push_footnotes_below_lines,"pushftn")}

\directlua{%
    luatexbase.add_to_callback
      ("pre_output_filter",move_footnote_bottom,"moveftn")}

\directlua{%
    luatexbase.add_to_callback
      ("vpack_filter",crush_height_of_vlist,"crushvbox")}

\directlua{%
    luatexbase.add_to_callback
      ("buildpage_filter",page_ftn_height,"truncatepage")}

% from tcolorbox/tcbbreakable.code.tex
\def\tcb@vsplit@upper{%
  \tcbdimto\tcb@split@dim{\tcb@split@dim-\my@tcb@ftn@height}\global\my@tcb@ftn@height\z@
  \setbox\tcb@upperbox=\vsplit\tcb@totalupperbox to\tcb@split@dim%
  \edef\tcb@upper@box@badness{\the\badness}%
  }

\def\my@tcb@output{%
  \ifdim\my@tcb@ftn@height>\z@
    % 
    \ifnum\outputpenalty>\z@
      \my@tcb@ftn@height\ht\footins
      \advance\my@tcb@ftn@height\footskip
    \fi
%    \expandafter\showthe\my@tcb@ftn@height
    \@tempdima\vsize
    \advance\@tempdima-\my@tcb@ftn@height
    \setbox\z@\vsplit\@cclv to \@tempdima%
    \setbox\tw@ =\vbox{\copy\@cclv}\unvbox\tw@%
    \global\my@tcb@ftn@height\z@
    \global\setbox\@cclv=\vbox{\copy\z@}%
  \fi
  \let \par \@@par
  \ifnum \outputpenalty<-\@M
    \@specialoutput
  \else
    \@makecol
    \@opcol
    \@startcolumn
    \@whilesw \if@fcolmade \fi
      {%
       \@opcol\@startcolumn}%
  \fi
  \ifnum \outputpenalty>-\@Miv
  \ifdim \@colroom<1.5\baselineskip
      \ifdim \@colroom<\textheight
        \@latex@warning@no@line {Text page \thepage\space
                               contains only floats}%
        \@emptycol
      \else
        \global \vsize \@colroom
      \fi
    \else
      \global \vsize \@colroom
    \fi
  \else
    \global \vsize \maxdimen
  \fi}

%\output{\my@tcb@output}

\makeatother

\title{\vspace*{-3\baselineskip}\textbackslash{}footnote 再考}
\author{鹿野桂一郎}
\date{2023年9月30日}

%\usepackage[colorlinks=true]{hyperref}
\begin{document}

\maketitle


横書き書籍においては、本文の任意の箇所に対して補足的な情報を付記したい場合、その本文と同一ページの最下部に「脚注」
\yafootnote{ここでは、補足情報をページの左右余白に配置する「傍注」、章末や巻末に配置する「後注」、行間に配置する「行間注」
\yafootnote{段落末に配置する「段落注」も行間注の一種とみなせる。}などと「脚注」を区別していることに注意。}として補足情報を組版することが多い。

\TeX{}系の組版システム\yafootnote{以降では単に「\TeX{}」と表記する。}では脚注を簡単に利用できる。
とくに\LaTeX{}で用意されている標準的な\texttt{\textbackslash{}footnote}コマンドは、さまざまなパラメーターを設定するだけで、脚注に対する組版上のさまざまな要件を簡単にカスタマイズできるようになっている。

一方、\TeX{}において脚注を実現する仕組み自体は、それほど単純ではない。
実際のところ\TeX{}における脚注は、専用の機能として実装されているわけではなく、「インサート」と呼ばれる汎用機能を使うことで利用者自身が好きな形で実装することが想定されている。
したがって、本来であれば利用者が自由に挙動を制御できるようになっているのだが、インサートそのものが単純な機構ではないことから、むしろ自由にならない側面が多い。
典型的には以下のような制限が知られている。

\begin{tcolorbox}[breakable,enhanced,before upper={\parindent1em}]

\begin{itemize}
\item minipageやtcolorboxにおいてはボックス末尾注\yafootnote{「ボックス末尾注」という呼称は一般的ではないが、何を指しているかはおおむね伝わるであろう
  \yafootnote{この脚注がボックス末尾注ではなく「脚注」になれている仕掛けを説明することが、きたるTeXConf~2023における本発表の目的の1つである。}。}になる
\item 上記のような環境でも「脚注」にする手段はあるが\yafootnote{\texttt{\textbackslash{}footnotemark}および\texttt{\textbackslash{}footnotetext}を利用する方法がよく知られている
  \yafootnote{ただし本稿における脚注はそれらを用いて組版されたものではない。}。}、複数ページにまたがる場合に中間ページに対しても「脚注」を実現する手段はない
  \yafootnote{本稿にはページ跨ぎボックスがないので、この種の脚注の具体例は出てこないが、TeXConf~2023における本発表ではこのような脚注の事例も紹介する。}
\item 脚注に対しては脚注を付与できない\yafootnote{本稿ではすでに何か所かで利用している\yafootnote{こんなふうに。}。この仕掛けもまたTeXConf~2023における本発表で説明される。}
\end{itemize}

\end{tcolorbox}

TeXConf~2023における本発表では、Lua\LaTeX{}を用いた上記の制限の回避方法の一例を示す。
脚注にこだわりがない方、もしくは脚注の乱用を心配する方にとっても、Lua\TeX{}で新しく導入された\texttt{attribute}レジスタおよびLua\TeX{}の各種フックを応用したLuaと\LaTeX{}との間での情報のやり取りの例として、主に実装面から興味深い事例になるものと考える。

\end{document}

\begin{tcolorbox}[breakable,enhanced,before upper={\parindent1em}]
  このボックスはページを跨ぐが、その内部の脚注\yafootnote{ページを跨ぐボックス内の脚注の例。}を当該のページに出すことは通常はできない。
  \clearpage
  
  ボックスの中間ページにも脚注を挿入できる\yafootnote{ページを跨ぐボックス内の脚注の例。}。
  
  \vspace*{15\baselineskip}
  (この余白は意図的です)
  \vspace*{15\baselineskip}
  
  \clearpage
  ボックスの最後のページ。
\end{tcolorbox}



\end{document}

