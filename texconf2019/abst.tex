\documentclass[book,paper=a4]{jlreq}
\begin{document}
あ
\end{document}

\directlua{dofile("./yafoot.lua")}

\documentclass[12pt]{ltjsarticle}

\newcount\yafootnotecount
\newdimen\footnotewidth
\footnotewidth=\textwidth
\newdimen\paryafootnoteskip
\paryafootnoteskip=.1\baselineskip

\def\yafootnote#1{%
  \global\advance\yafootnotecount 1
  \global\expandafter\newbox
    \csname yafoot_\the\yafootnotecount\endcsname
  \global\expandafter\setbox
    \csname yafoot_\the\yafootnotecount\endcsname=
    \vtop{\hsize\footnotewidth\par\hbox{\vrule width0pt height4pt}
      \footnotesize\narrowbaselines\noindent *\the\yafootnotecount\, #1}%
  \bgroup
    \attribute100=\expandafter\the
      \csname yafoot_\the\yafootnotecount\endcsname
    \expandafter\yafootnotemark\expandafter{\the\yafootnotecount}
    \vadjust {\pdfliteral{}}%
  \egroup
}

\def\yafootnotemark#1{\inhibitglue\small *\ensuremath{^{\mbox{#1}}}}

\directlua{%
    luatexbase.add_to_callback
      ("post_linebreak_filter",push_footnotes_below_lines,"pushftn")}

\directlua{%
    luatexbase.add_to_callback
      ("pre_output_filter",move_footnote_bottom,"moveftn")}

\directlua{%
    luatexbase.add_to_callback
      ("vpack_filter",crush_height_of_hlist,"crushhbox")}

\directlua{%
    luatexbase.add_to_callback
      ("hpack_filter",crush_height_of_hlist,"crushhbox")}

\title{\TeX の脚注をなんとかしたい}
\author{鹿野桂一郎}
\date{2019年10月12日}

\usepackage[most]{tcolorbox}
\usepackage[colorlinks=true]{hyperref}
\begin{document}
\maketitle

\vskip15\baselineskip

\begin{tcolorbox}[breakable,enhanced,before upper={\parindent1em}]
  脚注は、横組の書籍においては各ページの下部に配置される、本文の内容に対する付加的な情報である。
  通常、本文の途中にアンカーを設置し、その箇所に対応する付加的な内容を同一ページの下部に出力する
  \yafootnote{発表者は、\href{https://texconf16.tumblr.com/}{「TeXユーザの集い2016」}にて、アンカーと同一ページの下部に配置しない脚注の可能性を模索する発表を行った。なお本発表はそのときの発表には直接関係しない。なお本発表はそのときの発表には直接関係しない。なお本発表はそのときの発表には直接関係しない。なお本発表はそのときの発表には直接関係しない。なお本発表はそのときの発表には直接関係しない。なお本発表はそのときの発表には直接関係しない。なお本発表はそのときの発表には直接関係しない。}。

  自動組版で脚注を実現するためには、脚注として組まれるべき内容を配置するため、ページ下部にあらかじめ必要な余白を確保する必要がある。
  その余白の量を決定するには、そもそもページをどこで分割するかの決定に割り込む必要がある。
  
  \LaTeX などの\TeX ベースの自動組版システムでは、\TeX の「インサート」と呼ばれる機能を利用して、この割り込み処理を実現している。
  通常の本文段落にアンカーが出現する脚注に関しては、このインサートを利用したアルゴリズムにより、ほぼ満足がいく形で実現できる。

  しかしよく知られているように、この仕組みは\TeX の表組(\texttt{tabular}環境)中では利用できない。
  また、近年の書籍組版で多様されることが多いtcolorboxなどでも、ページ分割を許容する設定(\texttt{breakable})では、
  「アンカーと同一ページの下部に配置される」という、脚注として期待される挙動を得ることが事実上不可能である。

  本発表では、これらの環境で\TeX のインサートを利用した脚注が期待通りに動作しない理由を示す。
  また、発表者がこれまでに試みてきた手法を紹介する。
\end{tcolorbox}



\end{document}

