\begin{frame}[containsverbatim, t]{\inhibitglue CalibreかPandocでHTML→EPUB}
  \sffamily

  \begin{itemize}
    \item 電子書籍管理アプリケーションのCalibreには、各種フォーマットの変換コマンド\texttt{ebook-convert}が用意されている\\
    \begin{alltt}\small
$ ebook-convert --extra-css=book.css {\bslash}
                book.html book.epub
    \end{alltt}
    \item ただし、CalibreはEPUB2しか生成できないので、いまはPandocを使うのがよさそう\\
    \begin{alltt}\small
$ pandoc -t epub3 -o book.epub {\bslash}
         --epub-stylesheet=book.css {\bslash}
         book.html
    \end{alltt}
    \item 実際にはこんなオプションでは済まない!
  \end{itemize}

\end{frame}

\begin{frame}[containsverbatim, t]{\inhibitglue 自力でやるのも現実的}
  \sffamily

  \begin{itemize}
    \item HTMLをかき集めてメタ情報を用意すればいいので、難しくはない
    \item 実際に自作して使っている\\
    \begin{itemize}
      \item qnda (\url{https://github.com/k16shikano/qnda})
      \item \LaTeX{}が自動生成する情報(\texttt{{\bslash}ref}とか連番)を生成するのが面倒
      \item それでも、CalibreやPandocのオプションを調べるより楽だと思う(個人差があります)
    \end{itemize}
    \item いずれにせよ\texttt{epubcheck}を忘れずに\\
    \begin{alltt}\small
$ java -jar epubcheck.jar book.epub
    \end{alltt}
  \end{itemize}

\end{frame}

