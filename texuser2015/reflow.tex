\begin{frame}[t]{\inhibitglue EPUBといえばリフロー}
  \sffamily

  \begin{itemize}
    \item ユーザの手元で動的に作られる「ページ」
    \item 古き良き「ページ」というインターフェイスの\\ 再発明
  \end{itemize}
  \pause

  \begin{columns}[T]
    \begin{column}{.4551\textwidth}
      \begin{itemize}
        \item ページとその上の文字の物理的な制約が(主にユーザにとって)なくなる\\
        \begin{itemize}
          \item 小さい画面、巨大な画面、丸い画面、どんな読書端末で読んでもいい
        \end{itemize}
      \end{itemize}
      \vfill
    \end{column}
    \pause
    \begin{column}{.5\textwidth}
      \vskip\baselineskip
      \scriptsize\color{kachi}
      \shapepar{\circleshape}{\colorbox{yellow}{月世界上陸}\hfill  月世界の探険に於て、一番難所といわれるのは、無引力空間の通過だった。その空間は、丁度地球の引力と月の引力とが同じ強さのところであって、}
    \end{column}
  \end{columns}
      
      {\tiny \hfill\hbox{海野十三『月世界探険記』より}}

\end{frame} 

\begin{frame}[t]{\inhibitglue EPUBといえばリフロー}
  \sffamily

  \begin{itemize}
    \item ユーザの手元で動的に作られる「ページ」
    \item 古き良き「ページ」というインターフェイスの\\ 再発明
  \end{itemize}

  \begin{columns}[T]
    \begin{column}{.4551\textwidth}
      \begin{itemize}
        \item ページとその上の文字の物理的な制約が(主にユーザにとって)なくなる\\
        \begin{itemize}
          \item 小さい画面、巨大な画面、丸い画面、どんな読書端末で読んでもいい
        \end{itemize}
      \end{itemize}
      \vfill
    \end{column}
    \begin{column}{.5\textwidth}
      \vskip\baselineskip
      \scriptsize\color{kachi}
      \shapepar{\circleshape}{もしそこでまごまごしていたり、エンジンが\ruby{止}{とま}ったりすると、そこから先、月の方へゆくこともできず、さりとて地球の方へ引かえすことも出来ず}
    \end{column}
  \end{columns}
      
      {\tiny \hfill\hbox{海野十三『月世界探険記』より}}

\end{frame}

\begin{frame}[t]{\inhibitglue EPUBといえばリフロー}
  \sffamily

  \begin{itemize}
    \item ユーザの手元で動的に作られる「ページ」
    \item 古き良き「ページ」というインターフェイスの\\ 再発明
  \end{itemize}

  \begin{columns}[T]
    \begin{column}{.4551\textwidth}
      \begin{itemize}
        \item ページとその上の文字の物理的な制約が(主にユーザにとって)なくなる\\
        \begin{itemize}
          \item 小さい画面、巨大な画面、丸い画面、どんな読書端末で読んでもいい
        \end{itemize}
      \end{itemize}
      \vfill
    \end{column}
    \begin{column}{.5\textwidth}
      \vskip\baselineskip
      \scriptsize\color{kachi}
      \shapepar{\circleshape}{宙ぶらりんになってしまって、ただもう餓死を待つより外しかたがないという恐ろしい空間帯だった。\mbox{\hss}\hfill
\ruby{蜂谷艇長}{はちやていちょう}の巧みな指揮が、幸いにエンジン}
    \end{column}
  \end{columns}
      
      {\tiny \hfill\hbox{海野十三『月世界探険記』より}}

\end{frame}


