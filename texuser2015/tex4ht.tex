\begin{frame}[containsverbatim, t]{\inhibitglue \TeX{}4ht}
  \sffamily
  
  \begin{itemize}
    \item プロ向けにはよく使われているらしい
    \item 基本的な仕組みは、\\
    \begin{itemize}
\item[\color{black}1.] HTML構造用のヒントを\texttt{{\bslash}special}で埋め込んだ特別なdviを作るパッケージを読み込む\\
    \begin{alltt}\scriptsize
{\bslash}documentclass\{jsbook\}
...
{\bslash}usepackage[xhtml,mathml,charset=utf-8]\{tex4ht\}
...
{\bslash}begin\{document\}
    \end{alltt}
\item[\color{black}2.] 生成された特殊なdviを、\texttt{tex4ht}というコマンドで処理すると、HTMLができる
\item[\color{black}3.] さらに\texttt{t4ht}というコマンドで処理することでCSSを作る
    \end{itemize}
    \item \texttt{tex4ht}コマンドは、\texttt{platex}で処理されたdvi(に指定されている日本語用のjfm)を読めない!
  \end{itemize}
\end{frame}

\begin{frame}[containsverbatim, t]{\inhibitglue \TeX{}4htを日本語で使う方法、その1}
  \sffamily
  
  \begin{itemize}
    \item p\TeX{}を使わなければいい
    \item TeX4ebookというパッケージの機能を利用してfontspecを使う\\
    \begin{alltt}\scriptsize
{\bslash}documentclass\{book\} % jbook/jsbookはNG
{\bslash}usepackage\{alternative4ht\}
{\bslash}altusepackage\{fontspec\}
{\bslash}altusepackage\{xeCJK\}
{\bslash}altusepackage\{xunicode\}
{\bslash}setCJKmainfont\{IPAMincho\}
...
{\bslash}begin\{document\}
    \end{alltt}
    \item 実行にはLua\LaTeX{}が必要(\texttt{-l}オプション)\\
    \begin{alltt}\small
$ make4ht -l book.tex
    \end{alltt}
  \end{itemize}
\end{frame}

\begin{frame}[containsverbatim, t]{\inhibitglue \TeX{}4htを日本語で使う方法、その2}
  \sffamily
  
  \begin{itemize}
    \item 1つめの方法だとp\TeX{}のプリミティブが封じられる
    \item \texttt{platex}でコンパイルしたdviを\texttt{tex4ht}に読ませる手段はないか?
    \item \texttt{tex4ht}が読めないp\TeX{}由来のdvi命令\texttt{set2 \#N}を、\texttt{set\_char\_\#n}命令に変換できないか?
    \item 実はj\TeX{}は\texttt{set\_char\_\#n}命令だけで日本語の\\ 文字を印字している!
  \end{itemize}
\end{frame}

\begin{frame}[containsverbatim, t]{\inhibitglue p\TeX{}のdviをj\TeX{}のdviに変換}
  \sffamily
  
  \begin{itemize}
    \item 必要なもの:\\
    \begin{itemize}
      \item \texttt{dvi2dvi}
      \item \texttt{dvi2dvi}が使う仮想フォント{\scriptsize (Debianならdvi2ps-fontdata-a2n)}
      \item \texttt{jtex}用の\texttt{(dgj|dmj)*.tfm}{\scriptsize(Debianならjtex-base)}
      \item \texttt{tex4ht}が使う、\texttt{(dgj|dmj)*.tfm}からHTML用の文字への対応表{\scriptsize (このチートを開発した行木孝夫先生がむかし作ったものがW32\TeX{}に同梱されている)}
    \end{itemize}
    \item 実行手順:\\
    \begin{alltt}\scriptsize
$ platex book.dvi
$ dvi2dvi -F a2n -S book.dvi > book-ntt.dvi
$ tex4ht -i~/texmf/tex4ht/ht-fonts/ja/dnp {\bslash} 
         -cunihtf -utf8 book-ntt.dvi
    \end{alltt}
    \item (ひょっとしてW32\TeX{}ではこれを自動でやってくれる?)
  \end{itemize}
\end{frame}


