\documentclass[14pt,dvipdfmx,uplatex]{beamer}
\usetheme{Madrid} 
\setbeamertemplate{footline}[page number]{}
\beamertemplatenavigationsymbolsempty
\usepackage{mypresentation}
\AtBeginShipoutFirst{\special{pdf:tounicode UTF8-UCS2}}
\usepackage{shapepar}
\usepackage{tikz}
\usetikzlibrary{positioning}
\usepackage[noalphabet]{pxchfon}
\input{jpncolor}

\setgothicfont{migmix-2p-bold.ttf}
%\setgothicfont{YasashisaBold.ttf}
%\setminchofont{migmix-2p-bold.ttf} % 本文
\mathversion{bold}

\setbeamerfont{title}{size=\HUGE{28}{34},family={\yasagoth}}
\setbeamerfont{frametitle}{size=\HUGE{20}{28},series={\yasagoth}}
\setbeamerfont{frametext}{size=\HUGE{17}{24},series={\yasagoth}}
\setbeamertemplate{frametitle}[default][left]
\usefonttheme{professionalfonts}

\setbeamercolor{background}{bg=white}
\setbeamercolor{author}{fg=black}
\setbeamercolor{date}{fg=black}
\setbeamercolor{title}{fg=white, bg=kachi}
\setbeamercolor{frametitle}{fg=white}
\setbeamercolor{normal text}{fg=black}
\setbeamerfont{normal text}{family=\rmfamily, series=\bfseries}
\setbeamercolor{structure}{fg=black}

\makeatletter
\define@key{beamerframe}{t}[true]{% top
  \beamer@frametopskip=.2cm plus .5\paperheight\relax%
  \beamer@framebottomskip=0pt plus 1fill\relax%
  \beamer@frametopskipautobreak=\beamer@frametopskip\relax%
  \beamer@framebottomskipautobreak=\beamer@framebottomskip\relax%
  \def\beamer@initfirstlineunskip{}%
}
\def\header#1{\vskip.5\baselineskip{\large\sffamily #1}}
\makeatother

\setlength{\leftmargini}{12pt}
\setlength{\leftmarginii}{12pt}

\title{\TeX 原稿から\\ EPUBを作りたい}
\author{\sffamily 鹿野 桂一郎\\
\small\bfseries \email{k16.shikano@gmail.com} \\ 
\twitter{golden\_lucky} 
}
\date{\sffamily\footnotesize 2015年11月7日\\ \TeX{}ユーザの集い2015}

\begin{document}
\fontseries{ub}\selectfont

%{\usebackgroundtemplate{\includegraphics[height=1.1\paperheight]{skyrocket.jpg}}%
\frame{\titlepage}
%}

\setbeamertemplate{background canvas}[vertical shading][bottom=white,top=kachi!15]
\setbeamercolor{frametitle}{bg=kachi, fg=white}
\setbeamercolor{structure}{fg=kachi}

\begin{frame}[plain]
  \begin{center}
    \HUGE{34}{34}\color{kachi}\yasagoth
    EPUBの\\ なにがうれしいか
  \end{center}
\end{frame}

\begin{frame}[t]{\inhibitglue EPUBとは}
  \sffamily

  \begin{itemize}
    \item 電子書籍の標準フォーマットの規格
    \item 2007年9月にIDPF(\ruby{国際電子出版フォーラム}{International Digital Publishing Forum})が策定した
    \item 2011年10月に現行のメジャーバージョンであるEPUB3が策定された
  \end{itemize}
\end{frame}

\begin{frame}[t]{\inhibitglue EPUB には HTML が必要}
  \sffamily

  \begin{itemize}
    \item EPUB $\subset $ HTMLをZIPで固めたもの\\
    \begin{itemize}
      \item 実際には、ZIPの中にはHTML以外にどんなメディアをいれてもいい
      \item 規格上、テキストとして再生してもらえることになっているのは、HTMLとSVGのみ
    \end{itemize}
    \item レイアウトはCSSのみで指定する。JavaScriptは基本的に使えない
    \item EPUB3ではHTML5をサポートしているので、MathMLが使えることになっている\\
    \begin{itemize}
      \item \TeX{}ユーザにとっては朗報ですね
    \end{itemize}
  \end{itemize}
  
\end{frame}



\begin{frame}[t]{\inhibitglue EPUBといえばリフロー}
  \sffamily

  \begin{itemize}
    \item ユーザの手元で動的に作られる「ページ」
    \item 古き良き「ページ」というインターフェイスの\\ 再発明
  \end{itemize}
  \pause

  \begin{columns}[T]
    \begin{column}{.4551\textwidth}
      \begin{itemize}
        \item ページとその上の文字の物理的な制約が(主にユーザにとって)なくなる\\
        \begin{itemize}
          \item 小さい画面、巨大な画面、丸い画面、どんな読書端末で読んでもいい
        \end{itemize}
      \end{itemize}
      \vfill
    \end{column}
    \pause
    \begin{column}{.5\textwidth}
      \vskip\baselineskip
      \scriptsize\color{kachi}
      \shapepar{\circleshape}{\colorbox{yellow}{月世界上陸}\hfill  月世界の探険に於て、一番難所といわれるのは、無引力空間の通過だった。その空間は、丁度地球の引力と月の引力とが同じ強さのところであって、}
    \end{column}
  \end{columns}
      
      {\tiny \hfill\hbox{海野十三『月世界探険記』より}}

\end{frame} 

\begin{frame}[t]{\inhibitglue EPUBといえばリフロー}
  \sffamily

  \begin{itemize}
    \item ユーザの手元で動的に作られる「ページ」
    \item 古き良き「ページ」というインターフェイスの\\ 再発明
  \end{itemize}

  \begin{columns}[T]
    \begin{column}{.4551\textwidth}
      \begin{itemize}
        \item ページとその上の文字の物理的な制約が(主にユーザにとって)なくなる\\
        \begin{itemize}
          \item 小さい画面、巨大な画面、丸い画面、どんな読書端末で読んでもいい
        \end{itemize}
      \end{itemize}
      \vfill
    \end{column}
    \begin{column}{.5\textwidth}
      \vskip\baselineskip
      \scriptsize\color{kachi}
      \shapepar{\circleshape}{もしそこでまごまごしていたり、エンジンが\ruby{止}{とま}ったりすると、そこから先、月の方へゆくこともできず、さりとて地球の方へ引かえすことも出来ず}
    \end{column}
  \end{columns}
      
      {\tiny \hfill\hbox{海野十三『月世界探険記』より}}

\end{frame}

\begin{frame}[t]{\inhibitglue EPUBといえばリフロー}
  \sffamily

  \begin{itemize}
    \item ユーザの手元で動的に作られる「ページ」
    \item 古き良き「ページ」というインターフェイスの\\ 再発明
  \end{itemize}

  \begin{columns}[T]
    \begin{column}{.4551\textwidth}
      \begin{itemize}
        \item ページとその上の文字の物理的な制約が(主にユーザにとって)なくなる\\
        \begin{itemize}
          \item 小さい画面、巨大な画面、丸い画面、どんな読書端末で読んでもいい
        \end{itemize}
      \end{itemize}
      \vfill
    \end{column}
    \begin{column}{.5\textwidth}
      \vskip\baselineskip
      \scriptsize\color{kachi}
      \shapepar{\circleshape}{宙ぶらりんになってしまって、ただもう餓死を待つより外しかたがないという恐ろしい空間帯だった。\mbox{\hss}\hfill
\ruby{蜂谷艇長}{はちやていちょう}の巧みな指揮が、幸いにエンジン}
    \end{column}
  \end{columns}
      
      {\tiny \hfill\hbox{海野十三『月世界探険記』より}}

\end{frame}



\begin{frame}[t]{\inhibitglue 書籍に必要なのはコンテンツとメタ情報}
  \sffamily

  \begin{itemize}
    \item 大量のメタ情報が、リーダーで表示したり流通に乗せたりするために必要\\
    \begin{itemize}
      \item 書名、書名の読み、書名の種類、同一書名で別バージョンの本を区別する情報、シリーズ中の順番、原書名
      \item 著者名、著者名の別表記
      \item 発行日、改訂日、重刷日
      \item 出版社、製作所、コピーライト
      \item ISBNとかDOIとか出版社独自の番号のような識別子
    \end{itemize}
    \item EPUB3では、作り手がメタ情報を破綻なく拡張もできる仕組み(\texttt{<meta>}要素)がある
  \end{itemize}
\end{frame}

\begin{frame}[t]{\inhibitglue 文書の構造も一種のメタ情報}
  \sffamily

  \begin{itemize}
    \item 適切に意味付けされた構造は、アクセシビリティにとって重要\\
    \begin{itemize}
      \item そもそも再生に支障をきたす
      \item 読み上げとページめくりの同期とか、付加価値のある機能を提供するために必要
      \item コンテンツの再利用性は高いほうがいい
    \end{itemize}
    \item ナビゲーション(論理的な目次)がうまく機能するためには構造が必要
  \end{itemize}
\end{frame}

\begin{frame}[t]{\inhibitglue EPUB vs. PDF \\ $\approx$ 構造とメタ情報\, vs. 見た目}
  \sffamily

  \begin{itemize}
    \item EPUBは、多くの読者にとって、見た目と手軽さではPDFに劣る
    \item 流通やアクセシビリティを考えると、書籍としてのメリットが出始めている
  \end{itemize}
  \pause
  \begin{itemize}
    \item ただしPDF/UA(もしくはタグ付きPDF)のような規格もあるので「PDFはアクセシビリティだめ」というわけではない
  \end{itemize}
\end{frame}



\setbeamertemplate{background canvas}[vertical shading][bottom=white,top=yamabuki!15]
\setbeamercolor{frametitle}{bg=yamabuki, fg=black}
\setbeamercolor{structure}{fg=yamabuki}

\begin{frame}[plain]
  \begin{center}
    \HUGE{34}{34}\color{black}\yasagoth
    \TeX{}をHTMLに\\ 変換したい
  \end{center}
\end{frame}

\begin{frame}[t]{\inhibitglue \TeX{}→HTML変換の要件}
  \sffamily

  \begin{itemize}
    \item EPUBの仕様にかなったHTMLになること\\
    \begin{itemize}
      \item XML版のHTML5であること
      \item 要素のid属性や図に重複がないこと
    \end{itemize}
    \item 本の構造にかなったHTMLになること\\
    \begin{itemize}
      \item 本の中の位置に見合った連番
      \item 相互参照
    \end{itemize}
    \item 手作業でHTMLを編集しないで済むこと\\
    \begin{itemize}
      \item \TeX{}原稿から完全に機械的に変換したい
    \end{itemize}
  \end{itemize}
\end{frame}

\begin{frame}[t]{\inhibitglue 数式について}
  \sffamily
  \begin{itemize}
    \item EPUB3ではMathML(Presentational)が\\ サポートされているので、MathMLに変換すればよい
    \item Kindleでも出したかったら、SVGの画像にして\\ おくのがベストプラクティス
    \item MathJaxのことは忘れましょう\\
      \begin{itemize}
        \item JavaScriptが動くEPUB3リーダーはいまのところ一般的ではない
        \item 動いても、ネットワークもしくはmathjaxパッケージと\\ 数式用フォントの埋め込みが必要で、専用リーダーには\\ 非現実的
      \end{itemize}
    \pause
    \item いずれにせよ、ツールでなんとかなる時代になっています
  \end{itemize}
\end{frame}

\begin{frame}[t]{\inhibitglue \TeX{}からHTMLを手に入れる方法まとめ}
  \sffamily

  \begin{enumerate}
    \item[\sffamily\color{black}{1.}] \underline{テキストフィルタ型}\\
    \begin{itemize}
      \item \TeX{}原稿をテキストとしてパーズし、HTMLとして出力
      \item Pandoc、\LaTeX{}2htmlなど
      \item 気軽に使えるが、拡張性はほぼない
    \end{itemize}
    \item[\sffamily\color{black}{2.}] \underline{TeXエミュレート型}\\
    \begin{itemize}
      \item \TeX{}の処理(トークンを読み込んで箱を並べる)を模倣
      \item \LaTeX{}ml、plas\TeX{}、HeVeAなど
      \item 良好な結果がえられるが、独自のTeXマクロなどは自力で拡張が必要
    \end{itemize}
    \item[\sffamily\color{black}{3.}] \underline{DVIウェア型}\\
    \begin{itemize}
      \item \LaTeX{} にDVIを作らせて、それをHTMLにする
      \item \TeX{}4htなど
      \item ほぼ無敵(p\TeX{}を除く)だが、文書の構造が取れるわけではない
    \end{itemize}
  \end{enumerate}
\end{frame}

\begin{frame}[t]{\inhibitglue \TeX{}からHTMLを手に入れる方法まとめ}
  \sffamily

  \begin{enumerate}
    \item[\sffamily\color{black}{1.}] \underline{テキストフィルタ型}\\
    \begin{itemize}
      \item \TeX{}原稿をテキストとしてパーズし、HTMLとして出力
      \item Pandoc、\LaTeX{}2htmlなど
      \item 気軽に使えるが、拡張性はほぼない
    \end{itemize}
    \item[\sffamily\color{black}{2.}] \underline{TeXエミュレート型}\\
    \begin{itemize}
      \item \TeX{}の処理(トークンを読み込んで箱を並べる)を模倣
      \item {\color{shozyohi}\LaTeX{}ml}、plas\TeX{}、HeVeAなど
      \item 良好な結果がえられるが、独自のTeXマクロなどは自力で拡張が必要
    \end{itemize}
    \item[\sffamily\color{black}{3.}] \underline{DVIウェア型}\\
    \begin{itemize}
      \item \LaTeX{} にDVIを作らせて、それをHTMLにする
      \item {\color{shozyohi}\TeX{}4ht}など
      \item ほぼ無敵(p\TeX{}を除く)だが、文書の構造が取れるわけではない
    \end{itemize}
  \end{enumerate}
\end{frame}

\begin{frame}[containsverbatim, t]{\inhibitglue \LaTeX{}ml}
  \sffamily
  
  \begin{itemize}
    \item \TeX{}の消化の仕組みをPerlで模倣し、出力ルーチンの代わりにDOMを組み立てて生のXMLを吐き出すイメージ
    \item 吐き出したXMLをXSLTで後処理して使う
    \item 独自の\TeX{}マクロや\LaTeX{}コマンド・環境に対する処理を、Perlのモジュールを書いて追加できる\\
    \begin{alltt}\small
DefMacro('{\bslash}mybold\{\}','{\bslash}textbf\{#1\}');
    \end{alltt}
  \end{itemize}
\end{frame}

\begin{frame}[containsverbatim, t]{\inhibitglue \LaTeX{}ml(つづき)}
  \sffamily
  
  \begin{itemize}
    \item 精力的に開発されているようで、昨年の時点でTikZサポート率では\TeX{}4htを抜いたらしい
    \item 2014年5月には、直接EPUBを作ることも可能になっている\\
    \begin{alltt}\small
$ latexml --inputencoding=utf8 {\bslash}
          --dest=test.xhtml test.tex
$ latexmlpost --format=epub test.xhtml
    \end{alltt}
  \end{itemize}
\end{frame}

\begin{frame}[containsverbatim, t]{\inhibitglue \TeX{}4ht}
  \sffamily
  
  \begin{itemize}
    \item プロ向けにはよく使われているらしい
    \item 基本的な仕組みは、\\
    \begin{itemize}
\item[\color{black}1.] HTML構造用のヒントを\texttt{{\bslash}special}で埋め込んだ特別なdviを作るパッケージを読み込む\\
    \begin{alltt}\scriptsize
{\bslash}documentclass\{jsbook\}
...
{\bslash}usepackage[xhtml,mathml,charset=utf-8]\{tex4ht\}
...
{\bslash}begin\{document\}
    \end{alltt}
\item[\color{black}2.] 生成された特殊なdviを、\texttt{tex4ht}というコマンドで処理すると、HTMLができる
\item[\color{black}3.] さらに\texttt{t4ht}というコマンドで処理することでCSSを作る
    \end{itemize}
    \item \texttt{tex4ht}コマンドは、\texttt{platex}で処理されたdvi(に指定されている日本語用のjfm)を読めない!
  \end{itemize}
\end{frame}

\begin{frame}[containsverbatim, t]{\inhibitglue \TeX{}4htを日本語で使う方法、その1}
  \sffamily
  
  \begin{itemize}
    \item p\TeX{}を使わなければいい
    \item TeX4ebookというパッケージの機能を利用してfontspecを使う\\
    \begin{alltt}\scriptsize
{\bslash}documentclass\{book\} % jbook/jsbookはNG
{\bslash}usepackage\{alternative4ht\}
{\bslash}altusepackage\{fontspec\}
{\bslash}altusepackage\{xeCJK\}
{\bslash}altusepackage\{xunicode\}
{\bslash}setCJKmainfont\{IPAMincho\}
...
{\bslash}begin\{document\}
    \end{alltt}
    \item 実行にはLua\LaTeX{}が必要(\texttt{-l}オプション)\\
    \begin{alltt}\small
$ make4ht -l book.tex
    \end{alltt}
  \end{itemize}
\end{frame}

\begin{frame}[containsverbatim, t]{\inhibitglue \TeX{}4htを日本語で使う方法、その2}
  \sffamily
  
  \begin{itemize}
    \item 1つめの方法だとp\TeX{}のプリミティブが封じられる
    \item \texttt{platex}でコンパイルしたdviを\texttt{tex4ht}に読ませる手段はないか?
    \item \texttt{tex4ht}が読めないp\TeX{}由来のdvi命令\texttt{set2 \#N}を、\texttt{set\_char\_\#n}命令に変換できないか?
    \item 実はj\TeX{}は\texttt{set\_char\_\#n}命令だけで日本語の\\ 文字を印字している!
  \end{itemize}
\end{frame}

\begin{frame}[containsverbatim, t]{\inhibitglue p\TeX{}のdviをj\TeX{}のdviに変換}
  \sffamily
  
  \begin{itemize}
    \item 必要なもの:\\
    \begin{itemize}
      \item \texttt{dvi2dvi}
      \item \texttt{dvi2dvi}が使う仮想フォント{\scriptsize (Debianならdvi2ps-fontdata-a2n)}
      \item \texttt{jtex}用の\texttt{(dgj|dmj)*.tfm}{\scriptsize(Debianならjtex-base)}
      \item \texttt{tex4ht}が使う、\texttt{(dgj|dmj)*.tfm}からHTML用の文字への対応表{\scriptsize (このチートを開発した行木孝夫先生がむかし作ったものがW32\TeX{}に同梱されている)}
    \end{itemize}
    \item 実行手順:\\
    \begin{alltt}\scriptsize
$ platex book.dvi
$ dvi2dvi -F a2n -S book.dvi > book-ntt.dvi
$ tex4ht -i~/texmf/tex4ht/ht-fonts/ja/dnp {\bslash} 
         -cunihtf -utf8 book-ntt.dvi
    \end{alltt}
    \item (ひょっとしてW32\TeX{}ではこれを自動でやってくれる?)
  \end{itemize}
\end{frame}




\setbeamertemplate{background canvas}[vertical shading][bottom=white,top=kyara!15]
\setbeamercolor{frametitle}{bg=kyara, fg=white}
\setbeamercolor{structure}{fg=kyara}

\begin{frame}[plain]
  \begin{center}
    \HUGE{34}{34}\color{kachi}\yasagoth
    HTMLをEPUBにする
  \end{center}
\end{frame}

\begin{frame}[containsverbatim, t]{\inhibitglue CalibreかPandocでHTML→EPUB}
  \sffamily

  \begin{itemize}
    \item 電子書籍管理アプリケーションのCalibreには、各種フォーマットの変換コマンド\texttt{ebook-convert}が用意されている\\
    \begin{alltt}\small
$ ebook-convert --extra-css=book.css {\bslash}
                book.html book.epub
    \end{alltt}
    \item ただし、CalibreはEPUB2しか生成できないので、いまはPandocを使うのがよさそう\\
    \begin{alltt}\small
$ pandoc -t epub3 -o book.epub {\bslash}
         --epub-stylesheet=book.css {\bslash}
         book.html
    \end{alltt}
    \item 実際にはこんなオプションでは済まない!
  \end{itemize}

\end{frame}

\begin{frame}[containsverbatim, t]{\inhibitglue 自力でやるのも現実的}
  \sffamily

  \begin{itemize}
    \item HTMLをかき集めてメタ情報を用意すればいいので、難しくはない
    \item 実際に自作して使っている\\
    \begin{itemize}
      \item qnda (\url{https://github.com/k16shikano/qnda})
      \item \LaTeX{}が自動生成する情報(\texttt{{\bslash}ref}とか連番)を生成するのが面倒
      \item それでも、CalibreやPandocのオプションを調べるより楽だと思う(個人差があります)
    \end{itemize}
    \item いずれにせよ\texttt{epubcheck}を忘れずに\\
    \begin{alltt}\small
$ java -jar epubcheck.jar book.epub
    \end{alltt}
  \end{itemize}

\end{frame}



\setbeamertemplate{background canvas}[vertical shading][bottom=white,top=matsuba!15]
\setbeamercolor{frametitle}{bg=matsuba, fg=white}
\setbeamercolor{structure}{fg=matsuba}

\begin{frame}[plain]
  \begin{center}
    \HUGE{34}{34}\color{kachi}\yasagoth
    \TeX{}ユーザが\\ EPUBを気にする必要はあるのか?
  \end{center}
\end{frame}

\begin{frame}[t]{\inhibitglue \TeX はいかにもリフローと相性が悪そう}
  \sffamily

  \begin{itemize}
    \item EPUBは、デバイスに非依存で表示を保証することを考えていないメディア
    \item DVIは、デバイスに非依存で表示を保証することを目指していたメディア
  \end{itemize}

\end{frame}

\begin{frame}[t]{\inhibitglue \LaTeX{}で書かないという選択\\ (まとめのようなもの)}
  \sffamily

  \begin{itemize}
    \item EPUBが必要なら、\LaTeX{}をソースにすることをあきらめる\\
    \begin{itemize}
      \item PDFを作るつもりで書いた原稿をEPUBにしたい、というのは倒錯では?
    \end{itemize}
    \item EPUBにしやすい\LaTeX{}のサブセットを作る?
    \begin{itemize}
      \item \TeX{}っぽいシンタックスでEPUBを作りたい、というのは、やっぱり倒錯なような
    \end{itemize}
    \item タグ付きPDFや固定レイアウトでいいのかもしれない
  \end{itemize}
\end{frame} 




\setbeamertemplate{background canvas}[vertical shading][bottom=white,top=black!15]
\setbeamercolor{frametitle}{bg=black, fg=white}
\setbeamercolor{structure}{fg=black}

\begin{frame}[containsverbatim, t]{\inhibitglue 参考資料、URL}
  \sffamily
  \footnotesize

  \begin{itemize}
    \item Martin Bryan著, 山崎俊一監訳 ``SGML入門''(アスキー出版局, 1991)
    \item Brian Reid ``Scribe: A Document Specification Language and its Compiler''(Doctor Thesis, 1980)
    \item Leslie Lamport著, Edgar Cooke・倉沢良一監訳 ``文書処理システム\LaTeX ''(アスキー出版局, 1990)
    \item O'Reilly Media ``Using Web Standards in Print and Digital Book Workflows'' \url{http://www.w3.org/2012/12/global-publisher/slides/Day1/P0-witwer_adam.pdf}
  \end{itemize}

  \begin{itemize}
    \item pod:\url{http://perldoc.perl.org/perlpod.html}
    \item AsciiDoc:\url{http://www.methods.co.nz/asciidoc/}
    \item HTMLBook:\url{http://jagat-xml-publishing-study-group.github.io/HTMLBook-JA/}
    \item Philosophy of Markdown:\url{https://daringfireball.net/projects/markdown/syntax#philosophy}
  \end{itemize}

\end{frame}




\end{document}

