\begin{frame}[t]{\inhibitglue \TeX{}→HTML変換の要件}
  \sffamily

  \begin{itemize}
    \item EPUBの仕様にかなったHTMLになること\\
    \begin{itemize}
      \item XML版のHTML5であること
      \item 要素のid属性や図に重複がないこと
    \end{itemize}
    \item 本の構造にかなったHTMLになること\\
    \begin{itemize}
      \item 本の中の位置に見合った連番
      \item 相互参照
    \end{itemize}
    \item 手作業でHTMLを編集しないで済むこと\\
    \begin{itemize}
      \item \TeX{}原稿から完全に機械的に変換したい
    \end{itemize}
  \end{itemize}
\end{frame}

\begin{frame}[t]{\inhibitglue 数式について}
  \sffamily
  \begin{itemize}
    \item EPUB3ではMathML(Presentational)が\\ サポートされているので、MathMLに変換すればよい
    \item Kindleでも出したかったら、SVGの画像にして\\ おくのがベストプラクティス
    \item MathJaxのことは忘れましょう\\
      \begin{itemize}
        \item JavaScriptが動くEPUB3リーダーはいまのところ一般的ではない
        \item 動いても、ネットワークもしくはmathjaxパッケージと\\ 数式用フォントの埋め込みが必要で、専用リーダーには\\ 非現実的
      \end{itemize}
    \pause
    \item いずれにせよ、ツールでなんとかなる時代になっています
  \end{itemize}
\end{frame}

\begin{frame}[t]{\inhibitglue \TeX{}からHTMLを手に入れる方法まとめ}
  \sffamily

  \begin{enumerate}
    \item[\sffamily\color{black}{1.}] \underline{テキストフィルタ型}\\
    \begin{itemize}
      \item \TeX{}原稿をテキストとしてパーズし、HTMLとして出力
      \item Pandoc、\LaTeX{}2htmlなど
      \item 気軽に使えるが、拡張性はほぼない
    \end{itemize}
    \item[\sffamily\color{black}{2.}] \underline{TeXエミュレート型}\\
    \begin{itemize}
      \item \TeX{}の処理(トークンを読み込んで箱を並べる)を模倣
      \item \LaTeX{}ml、plas\TeX{}、HeVeAなど
      \item 良好な結果がえられるが、独自のTeXマクロなどは自力で拡張が必要
    \end{itemize}
    \item[\sffamily\color{black}{3.}] \underline{DVIウェア型}\\
    \begin{itemize}
      \item \LaTeX{} にDVIを作らせて、それをHTMLにする
      \item \TeX{}4htなど
      \item ほぼ無敵(p\TeX{}を除く)だが、文書の構造が取れるわけではない
    \end{itemize}
  \end{enumerate}
\end{frame}

\begin{frame}[t]{\inhibitglue \TeX{}からHTMLを手に入れる方法まとめ}
  \sffamily

  \begin{enumerate}
    \item[\sffamily\color{black}{1.}] \underline{テキストフィルタ型}\\
    \begin{itemize}
      \item \TeX{}原稿をテキストとしてパーズし、HTMLとして出力
      \item Pandoc、\LaTeX{}2htmlなど
      \item 気軽に使えるが、拡張性はほぼない
    \end{itemize}
    \item[\sffamily\color{black}{2.}] \underline{TeXエミュレート型}\\
    \begin{itemize}
      \item \TeX{}の処理(トークンを読み込んで箱を並べる)を模倣
      \item {\color{shozyohi}\LaTeX{}ml}、plas\TeX{}、HeVeAなど
      \item 良好な結果がえられるが、独自のTeXマクロなどは自力で拡張が必要
    \end{itemize}
    \item[\sffamily\color{black}{3.}] \underline{DVIウェア型}\\
    \begin{itemize}
      \item \LaTeX{} にDVIを作らせて、それをHTMLにする
      \item {\color{shozyohi}\TeX{}4ht}など
      \item ほぼ無敵(p\TeX{}を除く)だが、文書の構造が取れるわけではない
    \end{itemize}
  \end{enumerate}
\end{frame}
